\documentclass[12pt]{article}
\usepackage{amsmath}
\usepackage{amstext}
\usepackage{amssymb}
\usepackage{amsthm}
\makeatletter
\usepackage{graphicx,epsf}
\usepackage{times,float}
\usepackage{enumerate}
\usepackage[round,comma]{natbib}
\usepackage[colorlinks=true,citecolor=blue]{hyperref}
\usepackage{bm}
\usepackage{multirow}
%\usepackage{blkarray}
\usepackage{rotating}

\setlength{\textwidth}{6.4in} \setlength{\textheight}{8.5in}
\setlength{\topmargin}{-.2in} \setlength{\oddsidemargin}{.1in}
\renewcommand{\baselinestretch}{1.3}

\theoremstyle{definition}
\newtheorem{thm}{Theorem}
\newtheorem*{thm*}{Theorem}
\newtheorem{prop}{Proposition}
\newtheorem{cor}{Corollary}
\newtheorem{lem}{Lemma}
\newtheorem*{lem*}{Lemma}
\newtheorem{claim}{Claim}
\newtheorem{rem}{Remark}
\newtheorem{ex}{Example}
\newtheorem{fact}{Fact}
\newtheorem*{fact*}{Fact}
\newtheorem{remark}{Remark}


\newcommand{\rR}{\mathrel{R}}
\newcommand{\rP}{\mathrel{P}}
\newcommand{\real}{\mathbb{R}}
\newcommand{\norev}{\medskip \centerline{\textbf{No Revisions Below}} \medskip}
\renewcommand{\Re}{\mathbb{R}}
\newcommand{\In}{\mathbb{Z}}

\newcommand{\bare}{\overline{e}}
\newcommand{\barl}{\overline{l}}

\newcommand{\bq}{\mathbf{q}}

\newcommand{\cE}{\mathcal{E}}
\newcommand{\cH}{\mathcal{H}}
\newcommand{\cM}{\mathcal{M}}
\newcommand{\cP}{\mathcal{P}}
\newcommand{\cX}{\mathcal{X}}

\newcommand{\dr}{{\dot r}}
\newcommand{\dq}{{\dot q}}
\newcommand{\dg}{{\dot g}}
\newcommand{\ddp}{{\dot p}}

\newcommand{\hA}{{\hat A}}
\newcommand{\hO}{{\hat O}}

\newcommand{\halpha}{{\hat \alpha}}

\newcommand{\ta}{{\tilde a}}
\newcommand{\te}{{\tilde e}}
\newcommand{\tn}{{\tilde n}}

\newcommand{\tB}{{\tilde B}}

\newcommand{\bark}{{\overline k}}
\newcommand{\bart}{{\overline t}}

\newcommand{\varep}{\varepsilon}

\newcommand{\bone}{\mathbf{1}}

\begin{document}

\title{GCPS Schools: A User's Guide}

\author{Andrew McLennan\footnote{School of Economics, University of
    Queensland, {\tt a.mclennan@economics.uq.edu.au}}}

\date{\today}

\maketitle

\begin{abstract}
This document provides a brief introduction to the software package GCPS Schools.
\end{abstract}

% \pagebreak

\section{Introduction}

In the paper ``An Efficient School Choice Mechanism Based on a
Generalization of Hall's Marriage Theorem'' (joint with Shino Takayama
and Yuki Tamura) we describe a new algorithm for school choice, along
with its theoretical foundations.  This algorithm has been implemented
in the software package \emph{GCPS Schools} as an executable
\texttt{gcps}, which passes from a school choice problem (as described
below) to a matrix specifying, for each student-school pair, the
probability that the student is assigned to the school.  The software
package also contains two other executables \texttt{purify} and
\texttt{make\_ex}.  The first of these passes from a matrix of
assignment probabilities to a random pure assignment whose probability
distribution averages to the given matrix of probabilities.  The
second program generates example school choice problems of the sort
that might occur in large school districts.  These programs provide
the basic computational resources required to apply our mechanism, and
perhaps in some cases they will suffice.  However, our primary hope is
that the underlying code will be a useful starting point for further
software development.

This document describes these programs, from the point of view of a
user.  It doesn't assume that the reader has already read our paper,
but of course we are leaving out lots of relevant information.

\section{\texttt{gcps}}

To begin with we describe a simple example of an input file, which the
application \texttt{gcps} expects to find in a file called
\texttt{schools.scp} in the current directory.  (If there is no such
file \texttt{gcps} simply complains and quits.)

\begin{obeylines}\texttt{
/* This is a sample introductory comment. */
There are 4 students and 3 schools
The vector of quotas is (1,2,1)
The priority matrix is
     1     1     1
     1     0     1
     1     1     1
     1     1     1
The students numbers of ranked schools are (3,2,3,3)
The preferences of the students are
1:  1  2  3  
2:  1  3  
3:  1  2  3  
4:  1  2  3  
The priority thresholds of the schools are
1   1   1   
  }
\end{obeylines}

\medskip

Our input files begin with a comment between \texttt{/*} and
\texttt{*/}.  This is purely for your convenience, and the comment can
be of any length, and provide whatever information is useful to you,
but it is mandatory insofar as the computer will insist that the first
two characters of the file are \texttt{/*} and will only start
extracting information after it sees the \texttt{*/}.  The computer
divides the remainder into ``generalized white space'' (in addition to
spaces, tabs, and new lines, `\texttt{(}', `\texttt{)}', and
`\texttt{,}' are treated as white space) and ``tokens,'' which are
sequences of characters without any of the generalized white space
characters.  Tokens are either prescribed words, numbers, or student
tags (a student number followed by `\texttt{:}') and everything must
be more or less exactly as shown below, modulo white space, so, for
example, the first line must not be \texttt{There are 3 students and 1
  school}, but it could be \texttt{There are 3 students and \ \ 1
  schools}.

The next line gives the quotas (i.e., the capacities) of the schools,
so school 2 has two seats, and the other two schools each have one
seat.  Here we see the convenience of making `\texttt{(}',
`\texttt{)}', and `\texttt{,}' white space characters: otherwise we
would have had to write \texttt{The vector of quotas is 1 2 1}.

Our treatment of priorities is somewhat different from what is typical
in the school choice literature, where the priority is thought of as
the ``utility'' the school gets from a student, and is often required
to come from a strict ranking of the students.  At this stage a
student's priority at a school is either 1 if she is allowed to attend
the school, and may be assigned a seat there, and otherwise it is 0.
(We'll talk about more complicated priorities later.)  A student's
priority at a school may be 0 because she is not qualified (it is a
single sex school for boys, or her test scores are too low) or it may
be 0 because the student prefers a seat at her ``safe school'' (as we
explain below) and can insist on receiving a seat at a school that is
no worse for her than that.

The next line provides information (for each student, the number of
schools for which she has priority 1) that the computer could figure
out for itself, but we prefer to confirm that whatever person or
software prepared the input knew what they were doing.  After that
come the students' preferences: for each student, that student's tag
followed by the schools she might attend, listed from best to worst.
Finally, there are the schools' minimum priorities for admission,
which in this context are all 1: a student is good enough to admit to
a school if her priority for that school is 1 and not
otherwise. (Again, we'll talk about more complex situations later.)
The collection of information provided by an input file is a
\emph{school choice problem}.

What the software does (primarily) is compute a matrix of assignment
probabilities.  For our particular example \texttt{gcps} gives the
following output:
\medskip
\begin{obeylines}\texttt{
The allocation is:
\ \ \ \ \ 1:    \    2:  \      3:
1:      0.25     0.67     0.08
2:      0.25     0.00     0.75
3:      0.25     0.67     0.08
4:      0.25     0.67     0.08
}
\end{obeylines} \noindent

\smallskip
Note that the sum of the entries in each row is 1 and the sum of the
entries in each school's column is that school's quota.  In general
the sum of the quotas may exceed the number of students, in which case
we require that the sum of the entries in each school's column does
not exceed that school's quota. An assignment of probabilities with
these properties --- each student's total assignment is 1 and no
school is overassigned --- is a \emph{feasible allocation}.  

Our mechanism is based on the ``simultaneous eating'' algorithm of
\cite{bm01} for probabilistic allocation of objects, as generalized by
\cite{balbuzanov22jet}.  In our example each student consumes
probability of a seat in her favorite school (school 1) until that
resource is exhausted at time 0.25, at which point each student
switches to the next best thing.  This continues until school 2 is
also exhausted, after which all finish up by consuming probability of
a seat in school 3.

This makes good sense if the schools simply fill up one by one, as in
this example, but is that always what happens?  Actually, no.  To help
understand this we first introduce a new concept, the ``safe school.''
The idea is that each student has one school, say the closest school
or the school that a sibling attends, to which she is guaranteed
admission if she insists.  (It is a major advantage of our mechanism
that we can provide such a guarantee.)  Each student submits a ranking
of schools that she is eligible for and (weakly) prefers to the safe
school, and her priority is 1 at those schools and 0 at all others.

Now suppose that there are two schools, say 1 and 2, that are quite
popular.  Some students have school 1 as their safe school, but prefer
2, and some students have school 2 as their safe school, but prefer 1.
There are also some students who have other safe schools, but prefer
either 1 or 2, or both.  As the students consume probability at their
favorite schools, there can come a time at which schools 1 and 2
together only have enough remaining capacity to serve the students who
can insist on going to one of these two schools, even though school 1
still has excess capacity if it can ignore the students who have 1 as
their safe school but prefer 2 and the students who have 2 as their
safe school but prefer 1, and similarly for school 2.  When this
happens we say that the set of schools $P = \{1,2\}$ has become
\emph{critical}.

At this time further consumption of capacity at schools 1 and 2 is
restricted to those students who cannot be assigned to other schools,
so further consumption of these schools is denied to students who do
not have 1 or 2 as their safe school, and also to students who have 1
or 2 as their safe school but prefer some third school that is still
available.  For each of the latter students the least preferred of the
schools she is willing to attend that is still available becomes the
new safe school.

More generally, let $P$ be a set of schools, and let $J_P$ be the set
of students whose priorities for all schools outside of $P$ are 0.
For any $i \in J_P$, a feasible allocation must assign probability 1
to student $i$ receiving a seat in $P$, so a necessary condition for
the existence of a feasible allocation is that the total capacity of
the schools in $P$ is not less than the number of students in $J_P$.
In fact this condition is sufficient for the existence of a feasible
allocation: if, for each set of schools $P$, the total capacity of the
schools in $P$ is not less than the number of students in $J_P$, then
a feasible allocation exists.  This is not an obvious or trivial
result, and a somewhat more general version of it is one of the main
points of our paper.  This result extends to situations where the
resources have already been partially allocated: if, for each set of
schools $P$, the total remaining capacity of the schools in $P$ is not
less than the total remaining demand of students in $J_P$ (where this
set is defined in relation to the current partial allocation) then
there is an allocation of the remaining resources that gives a
feasible allocation.

We can now describe the algorithm in a bit more detail.  At each time
each student is consuming probability of a seat at the favorite school
among those that are still available to her.  This continues until the
first time that there is a set of schools $P$ such that the remaining
capacity is just sufficient to meet the needs of the students in the
set $J_P$ of students who no longer have access to any schools outside
of $P$.  At this point the problem divides into two subproblems, one
corresponding to the sets $P$ and $J_P$ and the other corresponding to
the complements of these sets.  These problems have the same form as
the original problem, and can be treated algorithmically in the same
way, so the algorithm can descend recursively to smaller and smaller
subproblems until a feasible allocation has been fully computed.


\section{\texttt{purify}} \label{sec:Implementation}

The output of \texttt{gcps} is a matrix of assignment probabilities, as shown in the example below.

\begin{obeylines}\texttt{
\ \ \ \ \ $A$:    \     $B$:  \      $C$:
1:      0.25     0.67     0.08
2:      0.25     0.00     0.75
3:      0.25     0.67     0.08
4:      0.25     0.67     0.08
}
\end{obeylines} \noindent
(Now, to avoid confusion, the schools are $A$, $B$, and
$C$.)

\medskip
Generating a random deterministic assignment with a probability
distribution that averages to this matrix is called
\emph{implementation} by \cite{bckm13aer}.  The executable
\texttt{purify} reads a feasible matrix $m$ of assignment
probabilities from a file \texttt{allocate.mat}, which must have the
form of the output of \texttt{gcps}, and it produces a random
deterministic allocation with a suitable distribution, using an
algorithm of \cite{bckm13aer}, as it applies to our somewhat simpler
framework.

We can illustrate the algorithm using the feasible allocation shown
above.  We consider a cyclic path alternating between students and
schools, say $1 \to C \to 3 \to A \to 1$, such that the entries of the
matrix for $(1,C)$, $(3,C)$, $(3,A)$, and $(1,A)$ are all strictly
between $0$ and $1$.  If we add $0.08$ to the entries for $(1,C)$ and
$(3,A)$ while subtracting $0.08$ from the entries for $(3,C)$ and
$(1,A)$, we obtain
\begin{obeylines}\texttt{
\ \ \ \ \ $A$:    \     $B$:  \  $C$:
1:      0.17     0.67     0.17
2:      0.25     0.00     0.75
3:      0.33     0.67     0.00
4:      0.25     0.67     0.08
}
\end{obeylines} \noindent
(Recall that $0.08$, $0.17$, and $0.33$ are really $\tfrac{1}{12}$,
$\tfrac16$ and $\tfrac13$.)  This is also a feasible allocation.  We
could also subtract $0.08$ from the entries for $(1,C)$ and $(3,A)$
while adding $0.08$ the entries for $(3,C)$ and $(1,A)$, thereby
obtaining the feasible allocation
\begin{obeylines}\texttt{
\ \ \ \ \ $A$:    \     $B$:  \      $C$:
1:      0.33     0.67     0.00
2:      0.25     0.00     0.75
3:      0.17     0.67     0.17
4:      0.25     0.67     0.08
}
\end{obeylines} \noindent
Supposing that we decide which of these to do by flipping a coin, we
could then look for another cycle and repeat the process.  Each step
reduces the number of entries that are strictly between $0$ and $1$,
so we would eventually arrive at a deterministic assignment.

We now give a more formal explanation of the algorithm, as implemented
in \texttt{implement.h}, \texttt{implement.c}, and \texttt{purify.c}.
There is a directed graph whose nodes are the students, the schools,
and a \emph{sink}.  The graph has an arc from each student to each
school, and an arc from each school to the sink.  A \emph{flow} is an
assignment of a positive number to each arc such that for each
student, the sum of the flows to all schools is $1$, and for each
school the sum of the flows from all students is equal to the flow
from that school to the sink.  A matrix of assignment probabilities
$m$ has an associated flow $f$ in which the flow from each student to
each school is the probability that the student receives a seat in the
school, and the flow from the school to the sink is the sum of the
school's assignment probabilities.

There is a subgraph consisting of all arcs whose flows are not
integers.  A key point is that for any node that is an endpoint of one
of the arcs in the subgraph, there is another node in the subgraph
that also has that node as an endpoint.  For each student, this is
obvious because the sum of the student's assignment probabilities is
one.  If the sum of the flows into a school is an integer, and one of
these flows is not an integer, then there must be another flow into
the school that is not an integer.  If the sum of the flows into a
school is not an integer, then one of the flows into the school is not
an integer, and the flow from the school to the sink is not an
integer.  The sum of the flows into the sink is the sum of the flows
out of the students, which is the number of students, hence an
integer, so if the flow from one of the schools to the sink is not an
integer, there must be another such school.

Consequently the subgraph has a \emph{cycle}, which is a sequence of
distinct nodes $n_1, \ldots, n_k$ such that $k > 2$ and, for each $i =
1, \ldots, k$, $n_i$ and $n_{i+1}$ are the endpoints of an arc in the
subgraph.  (We are treating the indices as integers mod $k$, so $k + 1
= 1$.)  The algorithm for finding a cycle (whose correctness is the
proof of the existence of a cycle) works in an obvious manner.
Beginning with $n_1$ and $n_2$ that are the endpoints of an arc in the
subgraph, it finds $n_3 \ne n_1$ such that $n_2$ and $n_3$ are the
endpoints of an arc in the subgraph.  In general, after finding $n_i$
such that $n_{i-1}$ and $n_i$ are the endpoints of an arc in the
subgraph, the algorithm asks whether there is $j = 1, \ldots, i-2$
such that $n_i = n_j$, in which case $n_j, \ldots, n_{i-1}$ is the
desired cycle, and otherwise it finds $n_{i+1} \ne n_{i-1}$ such that
$n_i$ and $n_{i+1}$ are the endpoints of an arc in the subgraph.
Since there are finitely many nodes, the process must eventually halt.

Given a cycle $n_1, \ldots, n_k$, for each $i = 1, \ldots, k$ we say
that $n_in_{i+1}$ is a \emph{forward arc} if $n_i$ is a student and
$n_{i+1}$ is a school, or if $n_i$ is a school and $n_{i+1}$ is the
sink, and otherwise we say that $n_{i+1}n_i$ is a \emph{backward arc}.
For any real number $\delta$, if we modify $f$ by adding a constant
$\delta$ to the flow of each forward arc while subtracting $\delta$
from the flow of each backward arc, the result $f^\delta$ is a new
flow, because for each student the sum of outward flows is $1$, and
for each school the sum of flows from students to the school is the
flow from the school to the sink.

Let $\alpha$ be the smallest positive number such that $f^\alpha$ has
at least one more integer entry than $f$, and let $\beta$ be the
smallest positive number such that $f^{-\beta}$ has at least one more
integer entry than $f$.  Then $f = \tfrac{\beta}{\alpha +
  \beta}f^\alpha + \tfrac{\alpha}{\alpha + \beta}f^{-\beta}$.  Let
$m^\alpha$ and $m^{-\beta}$ be the restrictions of $f^\alpha$ and
$f^{-\beta}$ to the arcs from students to schools.  It is easy to see
that $m^\alpha$ and $m^{-\beta}$ are feasible allocations: their
entries lie in $[0,1]$, and the sums of the entries for each student
and each school are the corresponding sums for $m$.  The algorithm
passes from $m$ and $f$ to $m^\alpha$ and $f^{\alpha}$ with
probability $\tfrac{\beta}{\alpha + \beta}$, and to $m^{-\beta}$ and
$f^{-\beta}$ with probability $\tfrac{\alpha}{\alpha + \beta}$.
Whichever of $m^\alpha$ and $m^{-\beta}$ is chosen, if it is not a
deterministic assignment, then the process is repeated.

The code for the algorithm described above is in \texttt{implement.c},
which has the associated header file \texttt{implement.h}.  The file
\texttt{purify.c} contains a high level sequence of commands that
execute the algorithm.

\section{\texttt{make\_ex}}

Development of this sort of software requires testing under at least
somewhat realistic conditions.  The utility \texttt{make\_ex} produces
examples of input files for \texttt{gcps} that reflect the
geographical dispersion of school districts with many schools, and the
idiosyncratic nature of school quality and student preferences.

One of the files produced by \texttt{make\_ex} begins as follows:
\begin{obeylines}\texttt{
    /* This file was generated by make\_ex with 20 schools,
    4 students per school, capacity 5 for all schools,
    school valence standard deviation 1.00,
    and idiosyncratic standard deviation 1.00. */
}
\end{obeylines} \noindent
In this example there are 20 schools that are spaced evenly around a
circle of circumference 20.  Since there are 4 students per school,
there are 80 students whose homes are spaced evenly around the circle.
Each student's safe school is the school closest to her home.  A
student's utility for a school is the sum of the school's valence and
an idiosyncratic shock, minus the distance from the student's home to
the school.  Each school's valence is a normally distributed random
variable with mean 0.0 and standard deviation 1.0, and for each
student-school pair the idiosyncratic shock is a normally distributed
random variable with mean 0.0 and standard deviation 1.0.  All of
these random variables are independent.  The program passes from the
utilities to an input for \texttt{gcps} by finding the ranking, for
each student, of the schools for which the student's utility is at
least as large as the utility of the safe school.

Near the beginning of the file \texttt{example.c} there are the following lines:
\begin{obeylines}\texttt{
  int no\_schools = 20;
  int no\_students\_per\_school = 4;
  int school\_capacity = 5;
  double school\_valence\_std\_dev = 1.0;
  double idiosyncratic\_std\_dev = 1.0;
}
\end{obeylines} \noindent
Even for someone who knows nothing about the C programming language,
this is pretty easy to understand.  The keywords \texttt{int} and
\texttt{double} are data types for integers and floating point
numbers.  Thus \texttt{no\_schools},
\texttt{no\_students\_per\_school}, and \texttt{school\_capacity} are
integers, while \texttt{school\_valence\_std\_dev} and
\texttt{idiosyncratic\_std\_dev} are floating point numbers.  Each
line assigns a value to some variable.  If you would like to generate
examples with different parameters, the way to do that is to change
the parameters by editing \texttt{example.c}, run \texttt{make}, and
then issue a command like \texttt{make\_ex > my\_file.scp}.  For
example, to diminish the relative importance of travel costs one can
increase \texttt{school\_valence\_std\_dev} and
\texttt{idiosyncratic\_std\_dev}.

This illustrates an important point concerning the relationship
between this software and its users.  Most softwares have interfaces
with the user that neither require nor allow the user to edit the
source code, but to create such an interface here would be
counterproductive. It would add complexity to the source code that had
nothing to do with the underlying algorithms.  More importantly, the
main purpose of this software is to provide a starting point for the
user's own programming effort in adapting it to the particular
requirements and idiosyncratic features of the user's school choice
setting.  Our algorithms are not very complicated, and someone
familiar with C should hopefully not have a great deal of difficulty
figuring out what is going on and then bending it to her purposes.
Starting to look at and edit the source code as soon as possible is a
first step down that road.


\section{Finer Priorities} \label{sec:Priorities}

To appreciate the issue discussed in this section one needs to
understand some of the history of other school choice mechanisms.
Instead of matching students to seats in schools, it is perhaps more
intuitive to consider matching a finite set of boys with a finite set
of girls, who each have strict preferences over potential partners and
remaining single.

The boy-proposes version of the famous deferred acceptance algorithm
begins with each boy proposing to his favorite girl, if there is one
he prefers to being alone.  Each girl rejects all proposals that are
less attractive than being alone, and if she has received more than
one acceptable proposal, she holds on to her favorite and rejects all
the others.  In each subsequent round, each boy who was rejected in
the previous round proposes to his favorite among the girls who have
not yet rejected him, if one of these is acceptable. Each girl now has
a number of new proposals, and possibly the proposal she brought
forward from the previous round.  She retains her favorite of these,
if it is acceptable, rejecting all others.  This procedure is repeated
until there is a round with no rejections, at which point each girl
holding a proposal pairs up with the boy whose proposal she is
holding. This mechanism was first proposed in the academic literature
by \cite{GaSh62}, but it turned out that it had already been used for
several years to match new graduates of medical schools with
residencies.  For almost twenty years it has been used in school
matching, with the students proposing and the seats in the various
schools rejecting, and it is now in widespread use around the world.

The key point for us is that this mechanism is not well defined unless
both sides have strict preferences.  In the context of school
matching, the schools' preferences are called \emph{priorities}.  If
these priorities are not actual reflections of society's values, this
can result in inefficiency.  For example, if Carol School's priorities
rank Bob above Ted while Alice School's priorities rank Ted above Bob,
then we could have an assignment in which Bob envies Ted's seat at
Alice School while Ted envies Bob's seat at Carol School.  This sort
of inefficiency can be quantitatively important, and a major advantage
of our mechanism is that it is efficient, in an even stronger sense
than not allowing outcomes in which improving trades are possible.

However, there are cases in which the schools' priorities do reflect
actual values.  In China, for example, each student's priority at all
schools is the score on a standardized test.  A consequence of this,
under deferred acceptance, is that, in effect, each school has an exam
score cutoff, accepting all students above the cutoff, rejecting all
students below the cutoff, and randomizing (roughly speaking) over
students right at the cutoff.  Our main concern in this section is to
explain how our mechanism can achieve similar outcomes.

The first point is that our input files can have a richer structure
than our original example suggests, as illustrated by the input on the
next page.  The priorities can be arbitrary nonnegative integers.  A
student having a priority of 0 at a particular school is understood as
indicating that the student cannot be assigned there, either because
she is not qualified or because she prefers her safe school.  A
student's safe school can be indicated by giving the student the
highest possible priority at that school.  The computer passes from
this input to a school choice problem in which the priority of a
student at a school is 1 if her priority in the input is not less than
the school's priority threshold, and it is 0 otherwise, each school's
priority threshold is set to 1, and each student's preference is
truncated by eliminating schools she is not eligible for.  Applying
this procedure to the input below gives our original example.

It is possible to repeatedly adjust the schools' priority thresholds
to achieve a desired effect.  For example, suppose there are two
selective schools, and the school district would like it to be the
case that a well qualified student is almost certain to receive a seat
in one of them if that is what she wants, and at the same time these
schools do not have more than a small amount of unused capacity.  One
may raise the priority threshold of a school if many students are
receiving some probability of admission and lower the threshold if its
seats are not being filled.  Of course changing the priority threshold
at one of the schools will effect demand for the other school, so
repeated adjustment of the priority thresholds of all the schools may
be required to achieve a desirable result.  (Automating this iterative
adjustment process may require the development of a version of
\texttt{gcps} that can accept parameter inputs, without editing the
source code.  This should be a simple task for an experienced C
programmer.)

\begin{obeylines}\texttt{
/* This is a sample introductory comment. */
There are 4 students and 3 schools
The vector of quotas is (1,2,1)
The priority matrix is
     5     6     9
     2     2     9
     5     4     9
     3     4     9
The students numbers of ranked schools are (3,3,3,3)
The preferences of the students are
1:  1  2  3  
2:  1  2  3  
3:  1  2  3  
4:  1  2  3  
The priority thresholds of the schools are
1   3   5   
  }
\end{obeylines}

Whether it is a good idea to use priorities as the Chinese do is an
extremely complex question.  On the one hand there is an obvious sense
in which it is desirable to provide the best resources to those who
can extract the greatest benefit.  On the other side, the Chinese
system intensifies the intergenerational transmission of advantage,
and there is some education research suggesting that average students
benefit from having talented peers while talented students are not
disadvantaged by having some peers who are ordinary. One could easily
list numerous additional issues.  Balancing various concerns in
practice requires information concerning what would actually happen
under various policy alternatives.  Our mechanism provides a wide
range of alternatives, for which outcomes from existing data can be
easily computed.

\section{What If There Are Many Schools?} \label{sec:ManySchools}

As the algorithm was described above, it looked ahead, for each
nonempty set of schools $P$, to determine the time at which it would
become necessary to restrict further consumption of schools in $P$ to
students in $J_P$.  This is not unduly burdensome if there are a
moderate number of schools.  (For a ``toy'' example with 20 schools,
hence over one million sets of schools, this form of the algorithm
finishes in about 10 seconds.)  But some school choice problems have
several dozen or even hundreds of schools, and will overwhelm the
naive version of the algorithm described above.  There are several
things that can be done about this.

To help things along a bit, the computer looks for schools whose
capacity will not be exceeded even if every student who ranks it ends
up receiving a seat.  Such a school is said to be \emph{unpopular}.
An unpopular school will never be an element of a minimal critical
set. A school is \emph{popular} if it is not unpopular

Two schools are \emph{related} if there is a student who can attend
either one.  The computer computes a square matrix \texttt{related}
whose rows and columns are indexed by the schools, such that
\texttt{related[j][k]} is one if either $j = k$ or $j$ and $k$ are
both popular and $\alpha_j \cap \alpha_k \ne \emptyset$, and otherwise
\texttt{related[j][k]} is zero.  We think of \texttt{related} as
encoding an undirected graph whose nodes are the schools, with an edge
connecting $j$ and $k$ if and only if \texttt{related[j][k]} and
\texttt{related[k][j]} are both one.  For any set of schools $P$ there
is an induced subgraph whose set of nodes is $P$ and whose edges are
the edges of the graph whose endpoints are both in $P$.

An undirected graph is \emph{connected} if, for any pair of nodes $j$
and $k$, there is a sequence of edges leading from $j$ to $k$.
Suppose that $P$ is a set of schools such that the induced subgraph of
\texttt{related} is not connected.  Then $P = P_1 \cup P_2$ where
$P_1$ and $P_2$ are nonempty, $P_1 \cup P_2 = \emptyset$, and there is
no path of edges leading from a school in $P_1$ to a school in $P_2$.
If $P$ is critical, then both $P_1$ and $P_2$ are critical, so $P$
cannot be a minimal critical set.

At this point we have seen that it is only necessary to consider
subsets $P$ of the set of popular schools such that the subgraph of
\texttt{related} induced by $P$ is connected.  For large school choice
problems the number of such sets can still be overwhelming.

The trick is to let the computational process itself tell us which
sets are relevant.  We maintain an assignment \texttt{subset\_sizes}
of a nonnegative integer to each school.  At the beginning of the
computation \texttt{subset\_sizes[j]} is zero if $j$ is unpopular and
one if $j$ is popular.  This corresponds to simply allocating each
school until it is exhausted, and (as in our initial example) it may
be the case that this computes a feasible allocation.  If it does not,
there will be a time prior to $1$ at which some student $i$ no longer
has any schools she can consume.  For each school $j$ in $\alpha_i$ we
increase \texttt{subset\_sizes[j]} from $1$ to $2$.  We now repeat the
allocation processs, this time looking for criticality of sets of
schools whose elements are a school in $\alpha_j$ and another school
that is related to it.  In general, the allocation process looks for
criticality of each connected set of schools $P$ such that there is a school
$j$ such that $j \in P$ and the number of elements of $P$ is not
greater than \texttt{subset\_sizes[j]}.  If the allocation process
does not succeed there is a student $i$ who runs out of schools before
time $1$, and we increase \texttt{subset\_sizes[j]} by $1$ if $j \in
\alpha_i$ and \texttt{subset\_sizes[j]} is minimal for $j \in
\alpha_i$.

In this way there arises the need to efficiently enumerate all of the
relevant $P$, which is provided by the function \texttt{next\_subset}
in \texttt{subset.c}. For any \texttt{related}-connected $r$-element
set $P$ and any $j \in P$ there is a unique ordering $j_1, \ldots,
j_r$ of the elements of $P$ such that $j_1 = j$ and, for each $h = 2,
\ldots, r$, $j_h$ is the smallest number among $j_h, \ldots, j_r$ that
is \texttt{related} to one of the schools $j_1, \ldots, j_{h-1}$.  We
endow these sets with a particular kind of lexicograpical order,
namely that $\{j_1, \ldots, j_r\} < \{j_1', \ldots, j_{r'}'\}$ if any
of the following hold:
\begin{enumerate}
  \item[(a)] $\{j_1, \ldots, j_r\} = \emptyset$.
  \item[(b)] $j_1 < j_1'$.
  \item[(c)] $j_1 = j_1'$ and $r < r'$.
  \item[(d)] $j_1 = j_1'$, $r = r'$, and there is $h$ such that $2 \le
    h \le r$, $j_1 = j_1', \ldots, j_{h-1} = j_{h-1}'$, and $j_h <
    j_h'$.
\end{enumerate}

For a given set $\{j_1, \ldots, j_r\}$ with the elements ordered as
above, the function \texttt{next\_subset} finds the next set in the
ordering.  If $\{j_1, \ldots, j_{r-1},j_r'\}$ is the next set, then
$j_r' > j_r$, and if $h$ is the smallest index such that $j_h$ and
$j_r'$ are \texttt{related}, then $j_{h+1}, \ldots, j_{r-1}$ are all
smaller than $j_r'$, because if $h+1 \le g \le r$ and $j_g \ge j_r'$,
then $j_g > j_r$ and $j_g$ is not the smallest element of $j_g,
\ldots, j_r$ that is \text{related} to some element of $j_1, \ldots
j_h$.  Conversely, if $j_r'$ is the smallest number such that $j_r'$
is related to one of $j_1, \ldots, j_{r-1}$ and $j_r' > \max
\{j_{h+1}, \ldots, j_r\}$, where $h$ is the smallest index such that
$j_h$ and $j_r'$ \texttt{related}, then $\{j_1, \ldots,
j_{r-1},j_r'\}$ is the next set in the ordering, and
\texttt{next\_subset} tests the numbers greater than $j_r$, in order,
for these conditions.

If there is no such $j_r'$, then \texttt{next\_subset} tests the
numbers $j_{r-1}' > j_{r-1}$, in order, for being \texttt{related} to
some $j_h$ with $1 \le h \le r-2$ and satisfying $j_{r-1}' > j_{h+1},
\ldots, j_{r-1}$, where $h$ is the smallest such index.  Having found
such a $j_{r-1}'$, it looks for the smallest $j_r'$ such that $\{j_1,
\ldots, j_{r-2},j_{r-1}', j_r'\}$ is satisfactory, either finding such
a $j_r'$ or moving on to the next $j_{r-1}'$.  If there is no
satisfactory $j_{r-1}'$, then \texttt{next\_subset} considers possible
$j_{r-2}' > j_{r-2}$, and so forth.  If there is no successor of
$\{j_1, \ldots, j_r\}$ of the form $\{j_1, \ldots, j_r\}$, then
\texttt{next\_subset} looks for the first successor of the form
$\{j_1,j_2', \ldots, j_{r'}'\}$ where $r' > r$, and if there is no
such successor, then it moves on to the enumeration of subsets
containing $j_r+1$.

%%%------------------------------------------------------------------------------------
%%%------------------------------------------------------------------------------------
\bibliographystyle{agsm}
\bibliography{pa_ref}

\newpage

\begin{appendix}

\section{About the Code} \label{app:Code}

As we have mentioned earlier, we hope that our code provides a useful
starting point for others, either contributing to the repository at
Github, or for applications to districts with particular features.
For this reason we have kept things as simple as possible, even if
that entails less convenience for the user.  In particular, the input
and output formats are inflexible, and some users will probably want
to develop a more sophisticated interface.

In this Appendix we provide an overview of the code, passing from the
simpler and more basic files to increasingly higher levels, in each
case describing those features that might not be so obvious.  Our hope
is to ease the process of learning about the code by providing a level
of explanation in which the objects in the code are described in human
terms, and in relation to the earlier descriptions of the algorithms.

Before diving into details, here are some general remarks.  First,
although we have used C rather than C++ (for a project as small as
this, the various advantages of C++ seem not worth the additional
complexity of that language) the code is object oriented in spirit,
being organized as interactions of objects given by \texttt{struct}s.
With perhaps one or two exceptions, each object has a destroyer, which
deallocates the memory that stores the object's data, and for many
objects there is a way of printing the object, primarily for
debugging.  In all cases the code for these functions is simple and
straightforward, and printing and destroyer functions will not be
mentioned below.  When studying the code, the reader should ignore
calls to destroyers, trusting that the allocation and freeing of
memory is being handled correctly.

In the C programming language, an $n$ element array is indexed by the
integers $0, \ldots, n-1$.  We always think of it as indexed by the
integers $1, \ldots, n$, so the $j^{\text{th}}$ component of
\texttt{vec} is \texttt{vec[j-1]}.  Similarly, the $(i,j)$ component
of a matrix \texttt{mat} is \texttt{mat[i-1][j-1]}.  While this is
perhaps not one of the most appealing features of C, and it certainly
adds bulk to the code, once you get used to it, in a curious way it
seems to enhance the readability of the code.

\subsection{\texttt{normal.h} and \texttt{normal.c}}

The function \texttt{min} computes the minimum of two doubles.  The
function \texttt{is\_integer} returns 1 (true) if the given double is
within one one millionth of an integer and 0 (false) otherwise.
Incidently, the reason that the numbers in the output of \text{gcps}
have many digits is that an output of \texttt{gcps} must be an
accurate input for \texttt{purify}, so \texttt{gcps} shouldn't (for
example) print 0.99 instead of 0.99999999. The functions
\texttt{uniform} and \texttt{normal} provided uniformly distributed
(in $[0,1]$) and normally distributed (for mean 0 and standard
deviation 1) pseudorandom numbers.

\subsection{\texttt{parser.h} and \texttt{parser.c}}

Two parsing functions \texttt{sch\_ch\_prob\_from\_file} and
\texttt{allocation\_from\_file} are declared in \texttt{parser.h}.  As
their names suggest, these functions read data from files,
constructing, respectively, a school choice problem
(\texttt{sch\_ch\_prob}) and an allocation (\texttt{partial\_alloc}).
A valid input file has an opening comment, which begins with
\texttt{/*} and ends with \texttt{*/}, and a body.  In the body, in
addition to the usual white space characters (space, tab, and newline)
the characters `\texttt{(}', `\texttt{)}', and `\texttt{,}' are
treated as white space.  The body is divided into whitespace and
tokens, which are sequences of characters without any white space.

Everything in \texttt{parser.c} is easy to understand.  There are
numerous functions checking that the verbal tokens are the ones that
are expected, and quitting with an error message if one of them
isn't. This makes the code extremely verbose and thoroughly
amateurish.  If the reader kindly refrains from looking in
\texttt{parser.c}, this author will be spared considerable
embarrassment.

\subsection{\texttt{subset.h} and \texttt{subset.c}}

One may represent a subset of $\{1, \ldots, n\}$ as an $n$-tuple of
0's and 1's, or as a list of its elements.  The first of these is
given by \texttt{subset}, which, in addition to the $n$-tuple
\texttt{indicator} of elements of $\{0,1\}$, keeps track of the number
of elements of the subset and the number of elements of the set it is
a subset of.  The second representation is given by \texttt{index}, in
which \texttt{no\_elements} is the number of elements of the subset
(not the containing set) and \texttt{indices} is a strictly increasing
\texttt{no\_elements}-tuple of elements of $\{1, \ldots,
\text{\texttt{large\_set\_size}}\}$.  The function
\texttt{index\_of\_subset} passes from the first representation to the
second.

A \texttt{square\_matrix} is a $\text{\texttt{dimension}} \times
\texttt{dimension}$ matrix whose $(i,j)$ entry is an integer
\texttt{entries[i-1][j-1]}.  The most important use of this notion is
to represent an undirected graph with
$$\text{\texttt{entries[i-1][j-1]}} = 1 =
\text{\texttt{entries[j-1][i-1]}}$$ if $i = j$ or the graph has an
edge with endpoints $i$ and $j$, and otherwise
$$\text{\texttt{entries[i-1][j-1]}} = 0 =
\text{\texttt{entries[j-1][i-1]}}.$$

The coding of \texttt{next\_subset} has already been described in Section \ref{sec:ManySchools}.

\subsection{\texttt{cee.h} and \texttt{cee.c}}

A school choice \emph{communal endowment economy} (CEE) consists of
\texttt{no\_students} students, \texttt{no\_schools} schools, a
specification of \texttt{quotas} (i.e., capacities) for the schools,
and a matrix \texttt{priority} specifying a nonnegative integer
\texttt{priority[i-1][j-1]} for each student \texttt{i} and each
school \texttt{j}.  When a CEE occurs as a part of an input, the
\texttt{quotas} are usually integers, but partially allocated CEE's
are used in the computations, when the remaining unallocated
\texttt{quotas} are floating point numbers.  For this reason there are
\texttt{int\_cee}'s and \texttt{double\_cee}'s.  Some of the more
advanced functions in \texttt{cee.h} are specific to priorities that
are either $0$ or $1$; in Section \ref{sec:Priorities} we explained
how to pass from more complicated priorities to binary priorities
using priority thresholds.

The computations of \texttt{popular\_schools} and
\texttt{relatedness\_matrix} are straightforward, and were explained
in Section \ref{sec:ManySchools}.  The function
\texttt{increase\_subset\_sizes} increases the components of
\texttt{subset\_sizes} for schools that
\texttt{underallocated\_student} is eligible for and for which the
current value is minimal.

The function \texttt{minimum\_gmc\_inequality} takes a
\texttt{double\_cee} and a (pointer to) a set \texttt{school\_subset}
as arguments, computes the set of students that cannot attend any
school outside \texttt{school\_subset}, returns 1 (true) if the GMC
inequality for \texttt{school\_subset} and this set of students holds,
and returns 0 (false) otherwise. The function \texttt{gmc\_holds}
decides whether the argument satisfies
\texttt{minimum\_gmc\_inequality} for every subset of the set of
schools.  The running time is proportional to 2 raised to the power of
the number of schools, which can be practical when the number of
schools is moderate, say around 20, but of course it blows up rapidly
after that.

\subsection{\texttt{schchprob.h} and \texttt{schchprob.c}}

A \emph{school choice problem} combines a CEE, which may be thought of
as describing the outcomes that are physically possible, with
preferences for the students and priority thresholds for the schools.
A student is \emph{eligible} for a school if her priority at that
school is at or above the school's priority threshold. A student's
(strict) preference is the list of the schools she is eligible for,
going from best to worst.  For convenience we keep track of each
student's number of eligible schools.

The underlying CEE may be either an \texttt{int\_cee} or a
\texttt{double\_cee}.  Typically the input school choice problem has
an \texttt{int\_cee}, and \texttt{double\_cee}'s are used in computing
an allocation, so there are \texttt{input\_sch\_ch\_prob}'s, which
have \texttt{int\_cee}'s, and \texttt{sch\_ch\_prob}'s, which have
\texttt{double\_cee}'s.  A \texttt{sch\_ch\_prob} is typically what
remains to be allocated after a certain time, so it has a member
\texttt{time\_remaining}.

During the allocation process, when a GMC inequality for a set $P$ of
schools is encountered, there is a smaller allocation problem for $P$
and the set $J_P$ of students who, at that point in the process, are
not eligible for any schools outside of $P$.  There is a similar
allocation problem for the complements $P^c$ and $J_P^c$ of $P$ and
$J_P$, and the continuation of the allocation process is the sum of
the allocation processes for these subproblems.  The function
\texttt{sub\_sch\_ch\_prob} constructs the subproblem for $J_P =
\text{\texttt{stu\_subset}}$ and $P = \text{\texttt{sch\_subset}}$.
This function has a new argument \texttt{underallocated\_student},
which is a pointer to an integer.  This interger is $0$ until there is
a subproblem for which some student has no eligible schools, at which
point it becomes that student's number.  The general idea is that when
this happens, the current attempt at an allocation is abandoned as
quickly as possible, and another attempt is made after increasing the
minimal \texttt{subset\_sizes} for the schools that the student is
initially eligible for.

The function \texttt{time\_remaining\_of\_gmc\_eq} computes the time
that remains if the allocation process continues until the GMC
inequality for \texttt{school\_subset} and \texttt{captive\_students}
holds with equality or the unit interval of time is exhausted,
ignoring all other constraints.  The function
\texttt{time\_remaining\_after\_first\_gmc\_eq} considers all the
school subsets generated by \texttt{next\_subset} with the parameters
\texttt{related} and \texttt{subset\_sizes}.  For each such set of
schools \texttt{time\_remaining\_of\_gmc\_eq} is applied to that set
and the set of students who cannot be assigned further probability in
schools outside that set.  It returns the maximum of these quantities
while setting the pointees of \texttt{crit\_stu\_subset} and
\texttt{crit\_sch\_subset} to subsets that attain the maximum.

\subsection{\texttt{partalloc.h} and \texttt{partalloc.c}}

In a \texttt{partial\_alloc} for \texttt{no\_students} students and
\texttt{no\_schools} schools, \texttt{allocations} is a matrix that
specifies an allocation \texttt{allocations[i-1][j-1]} of school
\texttt{j} to student \texttt{i} for each \texttt{i} and \texttt{j}.
A \texttt{pure\_alloc} has the same structure, but now
\texttt{allocations[i-1][j-1]} is an integer that should be zero or
one, and for each student \texttt{i} there should be exactly one
school \texttt{j} such that \texttt{allocations[i-1][j-1]} is one.

The function \texttt{increment\_partial\_alloc} increases a
\texttt{base} partial allocation by adding an \texttt{increment}
partial allocation to it.  The function
\texttt{allocate\_until\_new\_time} creates a \texttt{partial\_alloc}
in which each student receives
$$\text{\texttt{my\_scp->time\_remaining}} -
\text{\texttt{new\_time\_remaining}}$$ units of her favorite school
and none of any other school.  The function \texttt{school\_sums}
returns an array of \texttt{double} that specifies, for each school,
the total amount of it that has been allocated in \texttt{my\_alloc}.

\subsection{\texttt{solver.h} and \texttt{solver.c}}

The function \texttt{GCPS\_schools\_solver\_top\_level} creates a
\texttt{copy} of the input \texttt{my\_scp}, the vector
\texttt{subset\_sizes}, which is initially just the indicator function
of the set of popular schools, the matrix \texttt{related}, and the
integer pointer \texttt{underallocated\_student}.  It then asks
\texttt{GCPS\_schools\_solver} to try to find the GCPS allocation.  If
\texttt{GCPS\_schools\_solver} comes back with
\texttt{underallocated\_student != 0}, then \texttt{subset\_sizes} is
increased by applying \texttt{increase\_subset\_sizes} and the process
is repeated, again and again, until \texttt{underallocated\_student ==
  0} and the computation is complete.

The function \texttt{GCPS\_schools\_solver} begins by computing the
time that will remain after the allocation has proceeded to the first
GMC equality, and \texttt{stu\_subset} and \texttt{sch\_subset} are
set to be the student and school subsets of this equality.  The
complements \texttt{stu\_compl} and \texttt{sch\_compl} of these sets
are computed, and \texttt{first\_alloc} is the result of giving each
student her favorite school, at unit speed, until \texttt{end\_time}.
At \texttt{end\_time} the problem splits into two subproblems
(possibly one of them is null because it has no students) and in the
process of constructing them an \texttt{underallocated\_student} may
be found.  If this happens, then \texttt{GCPS\_schools\_solver}
aborts, sending \texttt{first\_alloc} (which is of no consequence) and
the \texttt{underallocated\_student} back to the function that called
\texttt{GCPS\_schools\_solver}, which may be either
\texttt{GCPS\_schools\_solver} itself or
\texttt{GCPS\_schools\_solver\_top\_level}.

If the construction of the subproblems does not turn up an
\texttt{underallocated\_student}, then \texttt{GCPS\_schools\_solver}
is called on each of them. (In this sense the algorithm is recursive.)
Of course these calls my find an \texttt{underallocated\_student},
whose number in the smaller problem must be translated into a number
in the current problem.  The calls for the two problems produce
\texttt{left\_alloc} and \texttt{right\_alloc}, which are
\texttt{partial\_alloc}'s, and \texttt{increment\_partial\_allocation}
is used to add these to \texttt{first\_alloc}, which is now the return
value of the current call to \texttt{GCPS\_schools\_solver}.

\subsection{\texttt{implement.h} and \texttt{implement.c}}

The code of the algorithm going from a fractional allocation to a
random pure allocation whose distribution has the given algorithm as
its average follows the description in Section
\ref{sec:Implementation}.  The \texttt{nonintegral\_graph} derived
from the given allocation is an undirected graph with an edge between
a student and a school if the student's allocation of the school is
between zero and one, and an edge between a school and the sink if the
total allocation of the school is not an integer.  The function
\texttt{graph\_from\_alloc} has the given allocation as its input, and
its output is the derived \texttt{nonintegral\_graph}.

Especially for large school choice problems, we expect the
\texttt{nonintegral\_graph} to be quite sparse, so it can be
represented more compactly, and be easier to work with, if we encode
it by listing the neighbors of each node.  The \texttt{stu\_sch\_nbrs}
member of \texttt{neighbor\_lists} is a list of \texttt{no\_students}
lists, where the \texttt{stu\_sch\_nbrs[i-1]} are arrays of varying
dimension. We set \texttt{stu\_sch\_nbrs[i-1][0] = 0} in order to have
a place holder that allows us to not have an array with no entries
when \texttt{i} has no neighbors.  The actual neighbors of \texttt{i}
are
$$\text{\texttt{stu\_sch\_nbrs[i-1][1],...,stu\_sch\_nbrs[i-1][stu\_no\_nbrs[i-1]]}}.$$
The members \texttt{sch\_no\_nbrs} and \texttt{sink\_sch\_nbrs} follow
this pattern, except that in the latter case there is just a single
list.  The member \texttt{sch\_sink\_nbrs} is a
\texttt{no\_schools}-dimensional array of integers with
\texttt{sch\_sink\_nbrs[j-1] = 1} if there is an edge connecting
\texttt{j} and the \texttt{sink} and \texttt{sch\_sink\_nbrs[j-1] = 0}
otherwise.  To pass from a \texttt{nonintegral\_graph} to its
representation as a \texttt{neighbor\_lists} we apply
\texttt{neighbor\_lists\_from\_graph}.

A cycle in the \texttt{nonintegral\_graph} is a linked list of
\texttt{path\_node}'s.  The function \texttt{find\_cyclic\_path}
implements the algorithm for finding a cycle that we described in
Section \ref{sec:Implementation}.  Given a cycle,
\texttt{bound\_of\_cycle} computes the smallest ``alternating
perturbation,'' in one direction or the other, of the entries of (the
pointee of) \texttt{my\_alloc} that turns some component of the
allocation, or some total allocation of a school, into an integer.
For such an \texttt{adjustment} the function
\texttt{cyclic\_adjustment} updates the allocation, and it calls the
functions \texttt{student\_edge\_removal} and
\texttt{sink\_edge\_removal} to update the \texttt{neighbor\_lists}.
When \texttt{graph\_is\_nonempty(my\_lists) = 0} (false) the entries
of \texttt{my\_alloc} are doubles that are all very close to integers,
and the function \texttt{pure\_allocation\_from\_partial} passes to
the associated \texttt{pure\_alloc}.  The function
\texttt{random\_pure\_allocation} is the master function that
supervises the whole process.

\subsection{\texttt{solve.c},\  \texttt{purify.c}, and \texttt{example.c}}

The files \texttt{solve.c}, \texttt{purify.c}, and \texttt{example.c}
contain the \texttt{main} functions of the executables \texttt{gcps},
\texttt{purify}, and \texttt{make\_ex}, respectively.  The
\texttt{main} functions in \texttt{solve.c} and \texttt{purify.c} are
simple and straightforward, but the \texttt{main} function in
\texttt{example.c} contains all of the code that is involved in
generating an example.  Although the code is somewhat lengthy, the
process is a straight line:
\begin{enumerate}
  \item[(a)] Locate the schools and students around the circle.
  \item[(b)] Compute the matrix of distances between students and schools.
  \item[(c)] Generate normally distributed random valences for the schools.
  \item[(d)] The utility of student \texttt{i} for school \texttt{j}
    is the valence of \texttt{j} plus a normally distributed
    \texttt{(i,j)}-idiosyncratic shock minus the distance from
    \texttt{i} to \texttt{j}.
  \item[(e)] Each student's safe school is (roughly) the one that is closest.
  \item[(f)] Student \texttt{i}'s priority at school \texttt{j} is one
    if its utility is not less than the utility of \texttt{i}'s safe
    school, and otherwise it is zero.
  \item[(g)] The preference of student \texttt{i} is the list of
    schools of priority one in order of decreasing utility.
\end{enumerate}

\end{appendix}

\end{document}

