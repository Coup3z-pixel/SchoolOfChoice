\documentclass[12pt, A4paper]{article}
\usepackage{amsmath}
\usepackage{amstext}
\usepackage{amssymb}
\usepackage{amsthm}
\makeatletter
\usepackage{graphicx,epsf}
\usepackage{times,float}
\usepackage{enumerate}
\usepackage[round,comma]{natbib}
\usepackage[colorlinks=true,citecolor=blue]{hyperref}
\usepackage{bm}
\usepackage{multirow}
%\usepackage{blkarray}
\usepackage{rotating}
\usepackage{setspace}

%\setlength{\textwidth}{6.4in} \setlength{\textheight}{8.5in}
%\setlength{\topmargin}{-.2in} \setlength{\oddsidemargin}{.1in}

%\renewcommand{\baselinestretch}{1.15}
\onehalfspacing

\usepackage{geometry}
\geometry{margin=1.25in}

\theoremstyle{definition}
\newtheorem{thm}{Theorem}
\newtheorem*{thm*}{Theorem}
\newtheorem{prop}{Proposition}
\newtheorem{cor}{Corollary}
\newtheorem{lem}{Lemma}
\newtheorem*{lem*}{Lemma}
\newtheorem{claim}{Claim}
\newtheorem{rem}{Remark}
\newtheorem{ex}{Example}
\newtheorem{fact}{Fact}
\newtheorem*{fact*}{Fact}
\newtheorem{remark}{Remark}


\newcommand{\rR}{\mathrel{R}}
\newcommand{\rP}{\mathrel{P}}
\newcommand{\real}{\mathbb{R}}
\newcommand{\norev}{\medskip \centerline{\textbf{No Revisions Below}} \medskip}
\renewcommand{\Re}{\mathbb{R}}
\newcommand{\In}{\mathbb{Z}}

\newcommand{\bq}{\mathbf{q}}

\newcommand{\cE}{\mathcal{E}}
\newcommand{\cG}{\mathcal{G}}
\newcommand{\cH}{\mathcal{H}}
\newcommand{\cI}{\mathcal{I}}
\newcommand{\cJ}{\mathcal{J}}
\newcommand{\cK}{\mathcal{K}}
\newcommand{\cL}{\mathcal{L}}
\newcommand{\cM}{\mathcal{M}}
\newcommand{\cO}{\mathcal{O}}
\newcommand{\cP}{\mathcal{P}}
\newcommand{\cQ}{\mathcal{Q}}
\newcommand{\cV}{\mathcal{V}}
\newcommand{\cX}{\mathcal{X}}

\newcommand{\dr}{{\dot r}}
\newcommand{\dq}{{\dot q}}
\newcommand{\dg}{{\dot g}}
\newcommand{\ddp}{{\dot p}}

\newcommand{\ho}{{\hat o}}

\newcommand{\hA}{{\hat A}}
\newcommand{\hO}{{\hat O}}

\newcommand{\halpha}{{\hat \alpha}}

\newcommand{\ta}{{\tilde a}}
\newcommand{\te}{{\tilde e}}
\newcommand{\tm}{{\tilde m}}
\newcommand{\tn}{{\tilde n}}

\newcommand{\tB}{{\tilde B}}

\newcommand{\bare}{\overline{e}}
\newcommand{\bark}{{\overline k}}
\newcommand{\barp}{\overline{p}}
\newcommand{\bart}{{\overline t}}
\newcommand{\bartheta}{{\overline \theta}}

\newcommand{\varep}{\varepsilon}

\newcommand{\bone}{\mathbf{1}}

\usepackage{titlesec}
\titlespacing*{\section}{0pt}{2ex plus 1ex minus 2ex}{2 ex plus .2ex minus 2ex}
\titlespacing*{\subsection}{0pt}{2ex plus 1ex minus 2ex}{2 ex plus .2ex minus 2ex}

\begin{document}

\title{The Push-Relabel Algorithm for a Communal Endowment Economy}

\author{Andrew McLennan\footnote{School of Economics, University of
    Queensland, {\tt a.mclennan@economics.uq.edu.au}}}

\date{\today}

\maketitle

\begin{abstract}
The central result of \cite{MTT23} is stated, and its proof using the theory of network flows is reviewed.  The push-relabel algorithm of \cite{GoTa88} is described, along with the proof of its validity and the analysis of its complexity in its application to the the particular networks of interest.
\smallskip
\\%[2ex]
\noindent \textbf{Keywords:} Maximal Flow, Push-Relabel Algorithm, Communal Endowment Economy.
%We introduce a generalized version of Hall's marriage theorem, providing a new algorithm for implementing the Generalized Constrained Probabilistic Serial (GCPS) mechanism by \cite{balbuzanov22jet}. This extends \cite{bm01}'s probabilistic serial mechanism to various constraint problems. When the agent or object numbers are not too large, our algorithm is computationally feasible, making it applicable to school choice problems with moderate school quantities. In our context, the GCPS mechanism demonstrates strong efficiency without justified envy. Furthermore, we present a variation of \cite{km10jet} result, highlighting the mechanism's strategy-proof nature in scenarios with a large number of agents competing for each object.
\end{abstract}

\medskip

\section{A Generalized Hall's Marriage Theorem} \label{sec:GenHall}

The primary purpose of this brief note (which is not intended for publication) is to explain the push-relabel algorithm of \cite{GoTa88} as it applies to the particular networks arising in \cite{MTT23}.  We first review the main theoretical result of that paper, and explain how the theory of maximal flow in a network is used to prove it.  In the second section we describe the push-relabel algorithm and analyze its complexity in this context.  The complexity analysis is the only original element of this note.

A \emph{communal endowment economy} (CEE) is a
quintuple $E = (I,O,r,q,g)$ in which $I$ is a nonempty finite set of \emph{agents}, $O$ is a nonempty finite set of \emph{objects}, $r \in \Re_+^I$, $q \in \Re_+^O$, and $g \in \Re_+^{I \times O}$.  We say that $r_i$ is
$i$'s \emph{requirement}, that $q_o$ is the \emph{quota} of $o$, and that $g_{io}$ is \emph{$i$'s $o$-max}.  
An \emph{allocation} for $I$ and $O$ is a
matrix $p \in \Re_+^{I \times O}$.  A \emph{partial allocation} for $E$ is an allocation $p$ such that $\sum_o p_{io} \le r_i$ for all $i$, $\sum_i p_{io} \le q_o$ for all $o$, and $p_{io} \le g_{io}$ for all $i$ and $o$. A \emph{feasible allocation} is a partial allocation $m$ such that $\sum_o m_{io} = r_i$ for all $i$. 

For $J \subset I$ and $P \subset O$ let $J^c = I \setminus J$ and $P^c = O \setminus P$ be the complements.  We say that $E$ satisfies the \emph{generalized marriage condition}
(GMC) if,  for every $J \subset I$ and $P \subset O$,
$$\sum_{i \in J} r_i \le \sum_{i \in J} \sum_{o \in P^c} g_{io} + \sum_{o \in P} q_o.$$  We will refer to this relation as the \emph{GMC inequality} for $(J,P)$.  The GMC is obviously necessary for the existence of a feasible allocation.  The central result of \cite{MTT23} is that this condition is also sufficient.

\begin{thm} \label{th:MultiHall}
  The CEE $E$ has a feasible allocation if and only if it satisfies the GMC.
\end{thm}

Our proof of Theorem \ref{th:MultiHall} is a simple application of the method of network flows.  (\cite{AhMaOr93} provides a general introduction and overview.)  Let $(N,A)$ be a directed graph ($N$ is a finite set of \emph{nodes} and $A \subset N \times N$ is a set of \emph{arcs}) with distinct distinguished nodes $s$ and $t$, called the \emph{source} and \emph{sink} respectively.  We assume that $(n,s), (t, n) \notin A$ for all $n \in N$.

A \emph{preflow} is a function $f \colon N \times N \to \Re$ such that:
\begin{enumerate}
  \item[(a)] for all $n$ and $n'$,  if $(n,n') \notin A$, then $f(n,n') \le 0$.
  \item[(b)] for all $n$ and $n'$,  $f(n,n') = - f(n',n)$ (antisymmetry); 
  \item[(c)] $\sum_{n' \in N} f(n',n) \ge 0$ for all $n \in N \setminus \{s,t\}$. 
\end{enumerate}
If neither $(n,n')$ nor $(n',n)$ is in $A$, then (a) and (b) imply that $f(n,n') = 0$.  Note that $f(s,n), f(n,t) \ge 0$ for all $n \in N$.
In conjunction with the other requirements, (c) can be understood as saying that for each $n$ other than $s$ and $t$, the total flow into $n$ is greater than or equal to the total flow out. 

A preflow $f$ is a \emph{flow} if $\sum_{n' \in N} f(n,n') = 0$ for all $n \in N \setminus \{s,t\}$.  In this case antisymmetry and this condition imply that
$$0 = \sum_{n' \in N}\sum_{n \in N} f(n,n') = \sum_{n \in N} f(n,s) + \sum_{n' \in N} f(n,t),$$
so we may define \emph{value} of $f$ to be
$$|f| = \sum_{n \in N} f(s,n) = \sum_{n \in N} f(n,t).$$

A \emph{capacity} is a function $c \colon N \times N \to [0,\infty]$ such that $c(n,n') = 0$ whenever $(n,n') \notin A$.   A preflow $f$ is \emph{bounded} by a capacity $c$ if $f(n,n') \le c(n,n')$ for all $(n,n')$.   A \emph{cut} is a set $S \subset N$ such that $s \in S$ and $t \in S^c$ where $S^c = N \setminus S$ is the complement.  For a capacity $c$, the \emph{capacity} of $S$ is
$$c(S) = \sum_{(n,n') \in S \times S^c} c(n,n').$$  

We say that a flow $f$ is \emph{circular} if $|f| = 0$.  For such an $f$ a path of arcs on which $f$ is positive eventually returns to a previously visited node, so repeatedly subtracting off the maximal flow along loops decomposes $f$ into a finite sum of flows along loops.  It follows that if $S$ is a cut, then $\sum_{(n,n') \in S \times S^c} f(n,n') = 0$.

For a flow $f$ with $|f| > 0$ there is a path of arcs on which $f$ is positive that goes from $s$ to $t$, and by cutting out loops one can find such a path  that does not visit any node twice.  By repeatedly subtracting off the maximal flows along such paths one can decompose $f$ into the sum of a circular part and a finite sum of flows along such paths.  It follows that if $S$ is a cut, then $|f| \le \sum_{(n,n') \in S \times S^c} f(n,n')$.  If $f$ is bounded by $c$, then this inequality implies that $|f| \le c(S)$, so the maximum value of flows bounded by $c$ is not greater than the minimum capacity of a cut for $c$.  

The max-flow min-cut theorem \citep{FoFu56} asserts that these two quantities are equal.  A consequence of this is that if $f$ is a maximal flow for $c$, then a cut $S$ has minimal capacity if and only if $|f| = \sum_{(n,n') \in S \times S^c} f(n,n') = c(S)$.  Since $c$ bounds $f$, this holds if and only if $f(n,n') = c(n,n')$ for all $(n,n') \in S \times S^c$. 

We define a particular directed graph $(N_E,A_E)$ in which the set of nodes is
$N_E = \{s\} \cup I \cup  O \cup \{t\}$.
For $i \in I$ and $o \in O$ let  $a_i = (s,i)$, $a_{io} = (i,o)$, and $a_o = (o,t)$, and let
$$A_E = \{\, a_i : i \in I \,\} \cup \{\, a_{io} : i \in I, o \in O \,\} \cup \{\, a_o : o \in O \,\}.$$
Let $c_E$ be the capacity in which 
$c_E(a_i) = r_i$, $c_E(a_{io}) = g_{io}$, and $c_E(a_o) = q_o$.  
If $p$ is an allocation, there is a unique flow $f_p$ such that $f_p(a_{io}) = p_{io}$ for all $i$ and $o$ that has $f_p(a_i) = \sum_o p_{io}$ for all $i$ and $f_p(a_o) = \sum_i p_{io}$ for all $o$. Evidently $p$ is a partial allocation if and only if $f_p$ is bounded by $c_E$, and it is a feasible allocation if and only if, in addition, $f_p(a_i) = r_i$ for all $i$, which is the case if and only if $|f_p| = \sum_i r_i$.  Conversely, if $f$ is a flow bounded by $c_E$ with $|f| = \sum_i r_i$ and thus $f(a_i) = r_i$ for all $i$, then setting $p_{io} = f(a_{io})$ gives a feasible allocation $p$.

Thus there is a feasible allocation if and only if the maximum value of a flow bounded by $c_E$ is $\sum_i r_i$.  The max flow-min cut theorem implies that this is the case if and only if the minimum capacity of a cut for $c_E$ is $\sum_i r_i$.  
For $J \subset I$ and $P \subset O$ let $S_{(J,P)} = \{s\} \cup J \cup P$.  This is a cut, and if $S$ is a cut, then $S = S_{(J,P)}$ where  $J = S \cap I$ and $P = S \cap O$.  An arc can go from a node in $S_{(J,P)}$ to a node in $S_{(J,P)}^c$ by going from $s$ to a node in $J^c$, by going from a node in $J$ to a node in $P^c$, and by going from a node in $P$ to $t$, so
$$c_E(S_{(J,P)}) = \sum_{i \in J^c} r_i + \sum_{i \in J} \sum_{o \in P^c} g_{io} + \sum_{o \in P} q_o.$$
Thus there is a feasible allocation if and only if $\sum_i r_i \le c_E(S_{(J,P)})$ for all $J \subset I$ and $P \subset O$, and this inequality is equivalent to the GMC inequality for $J$ and $P$.

\section{The Specialized Goldberg-Tarjan Algorithm}

The computational problems of finding the maximum flow or a minimal cut for a network $(N,A)$ and a capacity $c$ are very well studied, and many algorithms have been developed. 
The push-relabel algorithm of \cite{GoTa88} is relatively simple.  They show that it has worst case complexity of $\cO(|N|^2|A|)$ or $\cO(|N|^3)$, and \cite{ChMa05} improve this to $\cO(|N|^2\sqrt{|A|})$.  
In connection with the particular problem we study, it is possible to improve these bounds, as we explain below.  The literature continues to advance, and algorithms (e.g., \cite{CKLGS22}) with better asymptotic worst case bounds have been developed.
  
Let $f \colon N \times N \to \Re$ be a preflow that is bounded by $c$.  The \emph{excess} of $f$ at $n$ is $e_f(n) = \sum_{n' \in N} f(n',n)$.  Of course $f$ is a flow if and only if $e_f(n) = 0$ for all $n \in N \setminus \{s,t\}$.  The \emph{residual capacity} of $(n,n')$ is $r_f(n,n') = c(n,n') - f(n,n')$.  We say that $(n,n')$ is a \emph{residual edge} if $r_f(n,n') > 0$.  This can happen either because $c(n,n') > f(n,n') \ge 0$ or because $f(n,n') < 0$.  The \emph{residual graph} is the directed graph $G_f = (N,A_f)$ where $A_f$ is the set of residual edges.

A \emph{valid labelling} for $f$ and $c$ is a function $d \colon N \to \{0,1,2,\ldots\} \cup \{\infty\}$ such that $d(t) = 0$ and $d(n) \le d(n') + 1$ whenever $(n,n') \in A_f$.  We say that $n \in N \setminus \{s,t\}$ is \emph{active} for $f$ and $d$ if $d(n) < \infty$ and $e_f(n) > 0$.
The algorithm consists of repeatedly applying the following two \emph{elementary operations}, in any order, until there is no longer any valid application of them:
\begin{enumerate}
  \item[(a)] $\mathrm{Push}(n,n')$ is valid if $n$ is active, $(n,n') \in A_f$ and $d(n') = d(n) - 1$.  The operation resets $f(n,n')$ to $f(n,n') + \delta$ and $f(n',n)$ to $f(n',n) - \delta$ where $\delta = \min\{e_f(n),r_f(n,n')\}$.
  \item[(b)] $\mathrm{Relabel}(n)$ is valid if $n$ is active and $d(n) \le d(n')$ for all $n'$ such that $(n,n') \in A_f$.  The operation resets $d(n)$ to $\infty$ if there is no $n'$ such that $(n,n') \in A_f$, and otherwise it resets $d(n)$ to $1 + \min_{n' : (n,n') \in A_f} d(n')$.
\end{enumerate}
One intuitive understanding of the algorithm is that we imagine excess as water flowing downhill, so that $d(n)$ can be thought of as a height, (Goldberg and Tarjan offer a somewhat different intuition in which $d$ is a measure of distance.) We think of $\mathrm{Push}(n,n')$ as moving $\delta$ units of excess from a node $n$ to an adjacent node $n'$ that is one step lower.  The operation $\mathrm{Relabel}(n)$ is valid when there is excess ``trapped'' at $n$, and this operation increases $d(n)$ to the largest value allowed by the definition of a valid labelling, which is the smallest value such that there is a neighboring node the excess can flow to.

\begin{lem}
  If $f$ is a preflow and $d$ is a valid labelling for $f$, then after any valid elementary operation it is still the case that $d$ is a valid labelling for $f$.
\end{lem}

\begin{proof}
  The operation $\mathrm{Push}(n,n')$ may have no effect on $A_f$, it may remove $(n,n')$ from $A_f$, or it may add $(n',n)$ to $A_f$.  In the first two cases $d$ is valid after the operation because it was valid before, and in the third case it is valid because $d(n') = d(n) - 1$.
  
  The only way that $\mathrm{Relabel}(n)$ might make $d$ invalid is that after it, $d(n) > d(n') + 1$ for some $n'$ such that $(n,n') \in A_f$ or $d(n') > d(n) + 1$ for some $n'$ such that $(n',n) \in A_f$.  The first of these is obviously impossible, and the second is impossible because $\mathrm{Relabel}(n)$ increases $d(n)$ and $d$ was valid before the operation.  
\end{proof}

The algorithm cannot halt when there is an active node.

\begin{lem}
  If $f$ is a preflow, $d$ is a valid labelling for $f$, and $n$ is an active vertex, then either $\mathrm{Relabel}(n)$ is valid or there is an $n'$ such that $\mathrm{Push}(n,n')$ is valid.
\end{lem}

\begin{proof}
  Since $d$ is valid, $d(n) \le d(n') + 1$ for any $n'$ such that $(n,n') \in A_f$.   If this holds with equality for some $n'$ such that $(n,n')$ is residual, then $\mathrm{Push}(n,n')$ is valid, and otherwise $\mathrm{Relabel}(n)$ is valid.
\end{proof}

For the complexity analysis below a key parameter is the maximum length of a simple (no repeated nodes) path in $G_f$.  Let $G = (N,E)$ be the undirected graph in which an unordered pair $\{n,n'\}$ is an element of $E$ if either $(n,n') \in A$ or $(n',n) \in A$.  For any preflow $f$, any simple path in $G_f$ is a simple path in $G$.  

We now assume that $N = \{s\} \cup I \cup  O \cup \{t\}$ and 
$$A = \{\, a_i : i \in I \,\} \cup \{\, a_{io} : i \in I, o \in O \,\} \cup \{\, a_o : o \in O \,\}$$
where, for $i \in I$ and $o \in O$,  $a_i = (s,i)$, $a_{io} = (i,o)$, and $a_o = (o,t)$.  We are primarily interested in school choice applications in which $I$ is a set of students, $O$ is a set of schools, and there are many more students than schools, so we assume that $|O| < |I|$.  A simple path in $G$ from $s$ to $t$ alternates between nodes in $I$ and nodes in $O$, so the the total number of nodes it traverses is at most $2|O| + 2$.  

The algorithm is initialized as follows.  
We set $f(s,i) = -f(i,s) = c(s,i)$ for all $i$, $f(i,o) = 0$ for all $i$ and $o$, and $f(o,t) = 0$ for all $o$.
We set $d(s) = 2|O| + 2$ and $d(n) = 0$ for all $n \in N \setminus \{s\}$.  Since $s$ is never an active node, $d(s)$ cannot change during the run of the algorithm, and henceforth a valid labelling is a $d$ satisfying our original definition and also $d(s) = 2|O| + 2$.  After initialization the set of active nodes is $I$, and $A_f = \{\, (i,o) : c(i,o) > 0 \,\}$, so $d$ is valid.

An \emph{augmenting path} is a simple path in $G_f$ from $s$ to $t$.  A classical result of \cite{FoFu56} asserts that a flow $f$ is maximal if and only if $G_f$ does not have an augmenting path.  The next result implies that if the algorithm arrives at a state in which $f$ is a flow, then $f$ is maximal.

\begin{lem}
  If $f$ is a preflow and $d$ is a valid labelling for $f$, then $G_f$ does not have an augmenting path.
\end{lem}

\begin{proof}
  If $s = n_0, \ldots, n_k = t$ is an augmenting path, then $k < 2|O| + 2$, $d(s) = 2|O| + 2$,  $d(n_{h+1}) \ge d(n_h) - 1$ for $h = 0, \ldots, k-1$, and $d(t) = 0$, which is impossible.
\end{proof}

\begin{lem}
  If $f$ is a preflow and $s$ is not reachable from $n_0$ in $G_f$, then $e_f(n_0) = 0$.
\end{lem}

\begin{proof}
  Let $S$ be the set of nodes that are reachable from $n_0$.  We have 
  $$\sum_{n \in S} e_f(n) = \sum_{n \in S} \sum_{n' \in N} f(n',n) = \sum_{n \in S} \sum_{n' \in S^c} f(n',n)$$ because $\sum_{n \in S} \sum_{n' \in S} f(n',n) = 0$ by antisymmetry.  The definition of $S$ implies that $f(n',n) \le 0$ for all $n \in S$ and $n' \in S^c$, so $\sum_{n \in S} e_f(n) \le 0$.  Since $f$ is a preflow, $e_f(n) \ge 0$ for all $n \in N \setminus \{s\}$, so $e_f(n) = 0$ for each $n \in S$, including $n_0$,  if $s \notin S$.
\end{proof}

\begin{lem}
  For any $n \in N$ and any time during the course of the algorithm, $d(n) \le 4|O| + 3$.
\end{lem}

\begin{proof}
  The result can only fail if there is a $\mathrm{Relabel}(n)$ operation that leads to a violation of the inequality.  Suppose that $n$ is active.  The last result gives a simple path $n = n_0, \ldots, n_k = s$ in $G_f$.  We have $k < 2|O| + 2$ and $d(n_h) \le d(n_{h+1}) + 1$ for $h = 1, \ldots, k-1$, so $d(n_1) \le d(s) + k - 1 \le 4|O| + 2$.   Therefore the operation $\mathrm{Relabel}(n)$ increases $d(n)$ to at most $d(n_1) + 1 \le 4|O| + 3$.
\end{proof}  

An immediate consequence is:

\begin{lem}
    The number of $\mathrm{Relabel}$ operations during the course of the algorithm is at most $(|I| + |O|)(4|O| + 3)$.
\end{lem}

The operation $\mathrm{Push}(n,n')$ is \emph{saturating} if, after the operation, $r_f(n,n') = 0$.

\begin{lem}
  The number of saturating pushes during the course of the algorithm is at most $(4|O| + 2) |A|$.
\end{lem}

\begin{proof}
  Fix $n, n' \in N$.  After a saturating push from $n$ to $n'$ and before another push from $n$ to $n'$, there must be a push from $n'$ to $n$, so $d(n)$ must increase by at least $2$ between the two pushes from $n$ to $n'$.  After a saturating push from $n$ to $n'$ and before a push from $n'$ to $n$, $d(n')$ must increase by at least 2.  Therefore $d(n) + d(n')$ increases by at least two between any two saturating pushes.  We have $d(n) + d(n') \ge 1$ at the time of the first saturating push, and $d(n) + d(n') \le 8|O| + 6$ at the time of the last such push, so there are at most $4|O| + 2$ such pushes.  If there are saturating pushes from $n$ to $n'$ or from $n'$ to $n$, then $(n,n') \in A$ or $(n',n) \in A$ (or both).  
\end{proof}

\begin{lem}
  The number of nonsaturating pushes during the course of the algorithm is not greater than $(4|O| + 3)(|I| + |O| + |A|) - |A|$.
\end{lem}

\begin{proof}
  Let $\Phi = \sum_{\text{$n$ is active}} d(n)$.  Note that when the algorithm is initialized we have $\Phi = 0$, and at the end of the algorithm $\Phi = 0$ because there are no active nodes.  The total increase of $\Phi$ due to relabelling operations is at most $(|I| + |O|)(4|O| + 3)$.  A saturating push may increase or decrease $\Phi$, but it never increases $\Phi$ by more than $4|O| + 3$.  Therefore the total net increase of $\Phi$ due to relabelling and saturated pushes is not more than $$(4|O| + 3)(|I| + |O|) +  |A|(4|O| + 2)) = (4|O| + 3)(|I| + |O| + |A|) - |A|.$$  Each nonsaturating push from $n$ to $n'$ decreases $\Phi$ by at least $1$ because it makes $n$ inactive and $d(n) = d(n') + 1 \ge 1$. 
\end{proof}

We have now shown that the procedure is in fact an algorithm, insofar as it necessarily halts in finite time.  Earlier results now imply that is does compute a maximum flow, because it halts at a preflow that is in fact a flow, because it has no active nodes, and there is no augmenting path.

The total number of push operations is at most $(4|O| + 3)(|I| + |O| + 2|A|) - 2|A|$, which dominates the number of relabels.  We have $|A| \le |I| \cdot |O|$, so we obtain $\cO(|O|^2|I|)$ as the order of magnitude of the complexity bound.  Since $|N| = |I| + |O|$ and $|O| < |I|$ the bound of \cite{ChMa05} reduces to $\cO(|I|^{5/2}|O|^{1/2})$.

The basic operations may be performed in any order, and Goldberg and Tarjan derive their bound from a particular version of the generic algorithm that gives an improved bound, as they and \cite{ChMa05} prove.  Their intuitive motivation is to minimize the number of nonsaturating pushes. 
Their procedure has two data structures.  The first is an augmented version of the graph $G$ defined above, organized as a list of records for the various nodes, where the record for each node $n$ is a list of records for its neighbors, in arbitrary but fixed order, and a specification of a particular neighbor, which is called $n$'s \emph{current neighbor}.  They initialize $G$ by letting the current neighbor of $n$ be the first neighbor in $n$'s list.  The record of $n$ for each of its neighbors $n'$ specifies $c(n,n')$, $c(n',n)$, and $f(n,n')$.  

The main loop of the procedure invokes the \emph{push/relabel operation} at an active node $n$ with current neighbor $n'$.  That is: (a) it pushes excess from $n$ to $n'$ if that is possible; (b) if pushing excess from $n$ to  $n'$ is not possible and $n'$ is not the last element of $n$'s list, then it resets $n'$ to the next in $n$'s list; (c) if pushing excess from $n$ to $n'$ is not possible and $n'$ is the last element of $n$'s list, then it does $\mathrm{Relabel}(n)$ and resets the current neighbor to the first in $n$'s list.  By looping over the possible $n$, this approach insures a cycling through all possibilities, which means that opportunities are not ignored for long.

It is quite clear that the bounds derived in Goldberg and Tarjan's argument are very pessimistic, so we should expect that the practical performance of the algorithm will be much better than, and not closely related to, the theoretical worst case.  Our intuition is that the algorithm will be optimized practically by removing impediments to the flow of excess as quickly as possible, which means doing relabel operations whenever possible, and by having excess flow from the top down, so doing push operations at the highest possible value of $d$.  Since the push and relabel operations are not very expensive, we do not want to spend a lot of effort looking for the optimal such operations, which suggests maintaining well organized lists of operations that are possible.

\section{Bland's Rule}

The material in this section is related to what came before only insofar as these are notes to insure understanding of the algorithms I am implementing.  

We are studying the simplex algorithm of linear programming.  We first try to describe the situation geometrically, so suppose that our starting point is a linear program in an unusual form: $\max c^Tx + d$ subject to $Ax \le b$, where $A$ is an $m \times n$ matrix, $b \in \Re^m$, $c \in \Re^n$, and $d \in \Re$.   (We do not require that $x \ge 0$.) Let $P = \{\, x \in \Re^n : Ax \le b \,\}$.  We assume that $P$ is nonempty and bounded, hence a polytope.  We also assume that its affine hull is all of $\Re^n$, so $m > n$.  Geometrically, the simplex method moves from one vertex of $P$ to an adjacent vertex with a higher value of the objective function.  

For each vertex $v$ of $P$ there $i_1, \ldots, i_n$ such that the system of equations $A_{i_h} v = b_{i_h}$ for $h = 1, \ldots, n$ has $v$ 
as the unique solution.  In the nondegenerate case there is a unique such system, the edges of $P$ leading away from $v$ are each given by replacing one of these with 
the inequality $A_{i_h} x \le b_{i_h}$, and if none of these edges improve the objective function, then $v$ is a solution of the problem.  

In the degenerate case there are multiple such systems.  If there is an improving edge leading away from $P$, then there is some such system from which the edge is obtained by replacing one of the equations with the corresponding inequality.  The problem is to find such a system.

\norev


Our starting point is a linear program in an unusual form: $\max c^Tx + d$ subject to $Ax = b$ and $x \ge 0$, where $A$ is an $m \times n$ matrix, $m \le n$, $b \in \Re^m$, $c \in \Re^n$, and $d \in \Re$.  To begin with, at least, we assume that this data has a particularly simple form, namely that there are indices $j_1, \ldots, j_m \in \{1, \ldots, n\}$ such that for each $h$, the $j_h$ column of $A$ is the $h^{\text{th}}$ standard unit basis vector.  We also assume that $c_{j_h} = 0$ for all $h = 1, \ldots, m$, and that $b \ge 0$.   Let $x$ be the point with coordinates $x_{j_h} = b_h$ for $h = 1, \ldots, m$ and all other coordinates $0$.  This point satisfies $Ax = b$, and geometrically we imagine that we are ``at'' $b$ in the system of coordinates given by $j_1, \ldots, j_m$.
If $c \le 0$, then $x$ is a solution of the problem, and its value is $d$.

Suppose that $c_j > 0$.  We would like to increase $x_j$.  If $a_{ij} \le 0$ for all $i$, then $x_j$ and $c^Tx$ could be increased without bound, which is not the case of interest.    Therefore assume that there is at least one $i$ such that $a_{ij} > 0$.  The maximum that $x_j$ can be increased is $\min_{i : a_{ij} > 0} b_i/a_{ij}$.  

\bibliographystyle{agsm}
\bibliography{pa_ref}

%%%------------------------------------------------------------------------------------
%%%------------------------------------------------------------------------------------
\end{document}
