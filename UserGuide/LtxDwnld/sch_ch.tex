\documentclass[12pt, A4paper]{article}
\usepackage{amsmath}
\usepackage{amstext}
\usepackage{amssymb}
\usepackage{amsthm}
\makeatletter
\usepackage{graphicx,epsf}
\usepackage{times,float}
\usepackage{enumerate}
\usepackage[round,comma]{natbib}
\usepackage[colorlinks=true,citecolor=blue]{hyperref}
\usepackage{bm}
\usepackage{multirow}
%\usepackage{blkarray}
\usepackage{rotating}
\usepackage{setspace}

%\setlength{\textwidth}{6.4in} \setlength{\textheight}{8.5in}
%\setlength{\topmargin}{-.2in} \setlength{\oddsidemargin}{.1in}

%\renewcommand{\baselinestretch}{1.15}
\onehalfspacing

\usepackage{geometry}
\geometry{margin=1.25in}

\theoremstyle{definition}
\newtheorem{thm}{Theorem}
\newtheorem*{thm*}{Theorem}
\newtheorem{prop}{Proposition}
\newtheorem{cor}{Corollary}
\newtheorem{lem}{Lemma}
\newtheorem*{lem*}{Lemma}
\newtheorem{claim}{Claim}
\newtheorem{rem}{Remark}
\newtheorem{ex}{Example}
\newtheorem{fact}{Fact}
\newtheorem*{fact*}{Fact}
\newtheorem{remark}{Remark}


\newcommand{\rR}{\mathrel{R}}
\newcommand{\rP}{\mathrel{P}}
\newcommand{\real}{\mathbb{R}}
\newcommand{\norev}{\medskip \centerline{\textbf{No Revisions Below}} \medskip}
\renewcommand{\Re}{\mathbb{R}}
\newcommand{\In}{\mathbb{Z}}

\newcommand{\bq}{\mathbf{q}}

\newcommand{\cE}{\mathcal{E}}
\newcommand{\cG}{\mathcal{G}}
\newcommand{\cH}{\mathcal{H}}
\newcommand{\cI}{\mathcal{I}}
\newcommand{\cJ}{\mathcal{J}}
\newcommand{\cK}{\mathcal{K}}
\newcommand{\cL}{\mathcal{L}}
\newcommand{\cM}{\mathcal{M}}
\newcommand{\cO}{\mathcal{O}}
\newcommand{\cP}{\mathcal{P}}
\newcommand{\cQ}{\mathcal{Q}}
\newcommand{\cV}{\mathcal{V}}
\newcommand{\cX}{\mathcal{X}}

\newcommand{\dr}{{\dot r}}
\newcommand{\dq}{{\dot q}}
\newcommand{\dg}{{\dot g}}
\newcommand{\ddp}{{\dot p}}

\newcommand{\ho}{{\hat o}}
\newcommand{\hatt}{{\hat t}}

\newcommand{\hA}{{\hat A}}
\newcommand{\hO}{{\hat O}}

\newcommand{\halpha}{{\hat \alpha}}

\newcommand{\ta}{{\tilde a}}
\newcommand{\te}{{\tilde e}}
\newcommand{\tm}{{\tilde m}}
\newcommand{\tn}{{\tilde n}}
\newcommand{\tildet}{{\tilde t}}

\newcommand{\tB}{{\tilde B}}
\newcommand{\tP}{{\tilde P}}

\newcommand{\bare}{\overline{e}}
\newcommand{\bark}{{\overline k}}
\newcommand{\barp}{\overline{p}}
\newcommand{\bart}{{\overline t}}
\newcommand{\barv}{{\overline v}}
\newcommand{\bartheta}{{\overline \theta}}

\newcommand{\varep}{\varepsilon}

\newcommand{\bone}{\mathbf{1}}

\usepackage{titlesec}
\titlespacing*{\section}{0pt}{2ex plus 1ex minus 2ex}{2 ex plus .2ex minus 2ex}
\titlespacing*{\subsection}{0pt}{2ex plus 1ex minus 2ex}{2 ex plus .2ex minus 2ex}

\begin{document}

\title{An Efficient, Computationally Tractable School Choice Mechanism}

\author{Andrew McLennan\footnote{School of Economics, University of
    Queensland, {\tt a.mclennan@economics.uq.edu.au}} \and  Shino
Takayama\footnote{School of Economics, University of
  Queensland, {\tt s.takayama1@uq.edu.au}} \and Yuki Tamura\footnote{Center for Behavioral Institutional Design, NYU Abu Dhabi; {\tt yuki.tamura@nyu.edu}}}

\date{\today}

\maketitle

\begin{abstract}
We provide a generalization of Hall’s marriage theorem. Using this, we describe an implementation of the Generalized Constrained Probabilistic Serial (GCPS) mechanism, proposed by \cite{balbuzanov22jet}, to problems with feasibility constraints. If the number of agents or the number of objects is not large, our algorithm is computationally tractable. In particular, it can be applicable to school choice problems with a moderate number of schools. In our context, the GCPS mechanism is efficient and has no justified envy. In addition, this mechanism is effectively strategy proof when the number of agents becomes large.

\smallskip
\noindent \textbf{Keywords:} School Choice, Object Allocation, Efficiency, Fairness, Strategy Proofness, Probabilistic Serial Mechanism, Hall's Marriage Theorem.
%We introduce a generalized version of Hall's marriage theorem, providing a new algorithm for implementing the Generalized Constrained Probabilistic Serial (GCPS) mechanism by \cite{balbuzanov22jet}. This extends \cite{bm01}'s probabilistic serial mechanism to various constraint problems. When the agent or object numbers are not too large, our algorithm is computationally feasible, making it applicable to school choice problems with moderate school quantities. In our context, the GCPS mechanism demonstrates strong efficiency without justified envy. Furthermore, we present a variation of \cite{km10jet} result, highlighting the mechanism's strategy-proof nature in scenarios with a large number of agents competing for each object.
\end{abstract}

\medskip

In a seminal paper \cite{as03aer} propose the application, to school choice, of two mechanisms based on matching theory.  The \emph{student proposes deferred acceptance} (DA) mechanism was originally proposed by \cite{GaSh62}, and it has been widely adopted for school choice and similar problems around the world.  The \emph{top trading cycles} (TTC) mechanism was originated by David Gale, as described by \cite{ShSc74}, and although it has some superior theoretical properties, it has found less practical acceptance.  This paper argues that the \emph{generalized constrained probabilistic serial} (GCPS) mechanism of \cite{balbuzanov22jet}, which is a generalization of  the probabilistic serial (PS) mechanism of \cite{bm01} (henceforth BM) is also viable and attractive as a school choice mechanism.

Both DA and TTC require that the schools have strict rankings of the students that are called \emph{priorities}.  At the outset in DA each student applies to her favorite school.  Each school with more applicants than its capacity rejects the lowest priority applicants beyond the number it can serve.  In each subsequent round each student who was rejected in the preceeding round applies to her favorite school among those that have not rejected her, and each school retains the highest priority applicants, up to its capacity, among those who have applied in all rounds, and rejects all others.  The process continues in the same manner until there is a round with no rejections.  

In TTC each student points to her favorite school and each school points to its highest priority student. The resulting directed graph has at least one cycle, each student in a cycle is assigned to the school she points to, and she is removed from the mechanism, along with the seat she claimed in her school.  This process is then repeated with the remaining students and seats, and it continues in this manner until all students have been assigned.

In practice (e.g.~\cite{Pat17}) transparency and straightforward incentives are required in order for school choice mechanisms to be accepted by parents.  Proving that DA and TTC are strategy proof, and that TTC assignments are ex post efficient from the point of view of the students, may be suitable exercises for graduate students, but these results are far from intuitive for the untrained.  When different schools have different priorities, the role of priorities in TTC is particularly hard to grasp.  (See \cite{LeLo20}.)  

The \emph{Boston} (or immediate acceptance) mechanism begins by assigning as many students to their favorite (according to the submitted rankings) schools as possible.  It then assigns as many of the remaining students to their second favorite schools as possible, and it continues similarly, in the $k^{\text{th}}$ round assigning as many of the remaining students to their  $k^{\text{th}}$ favorite schools as possible.  Since it is possible that a student can (for example) greatly increase her chance of being accepted at her second favorite school if she ranks it as her favorite, the Boston mechanism is not strategy proof, and in fact it is strategically tricky, with high stakes.  Nevertheless it continues to be widely used because it is conceptually simple.  

The GCPS mechanism has a similar conceptual simplicity: during the unit interval of time, each student is assigned probability of receiving a seat in her favorite school, among those she is eligible for, at unit speed, until some capacity constraint is encountered, at which point each student who can no longer be assigned probability in her favorite school switches to receiving probability in the favorite school among those that still have some available capacity.  The process continues similarly, with each student at each time receiving probability of receiving a seat in her favorite school among those that are still available, so that at time $1$ each student has a probability distribution over schools.  As we explain in Online Appendix \ref{sec:Implementability}, it is possible to compute a random deterministic assignment with a distribution that realizes these probabilities.

Strictly speaking, the GCPS mechanism is not strategy proof, but we will argue that it is effectively strategy proof in an easily understood sense.  In order to manipulate by submitting a false ranking, there must first be a period of time during which the student is receiving probability of a school that is worse than the one she would have been consuming if she had told the truth, followed by a period of time during which she is receiving probability in a school that is better than the one she would have been consuming, but if many students are competing for desirable schools, the extension of the latter period due to the manipulation is necessarily brief.

DA produces assignments that are \emph{stable}: there is no student-school pair such that the student prefers that school to the one she has been matched with and the student has a higher priority at that school than some other student that has been assigned to that school.  Consequently the assignments produced by DA are ex post efficient, at least if the schools' priorities actually reflect society's values.  However, if the priorities are generated randomly, simply in order to fulfill the requirements of the mechanism, then assignments can be inefficient.  (In the simplest instance, if Bob prefers Carol School to Alice School, while Ted prefers Alice School, the mechanism may nevertheless assign Bob to Alice School and Ted to Carol School if the schools' priorities ``prefer'' that outcome.) In a study of New York City data \cite{apr09aer} found that the inefficiencies arising in this way are quantitatively significant.

The outcomes produced by TTC are ex post efficient relative to the students' preferences.  Specifically, the students leaving the mechanism in the first round are receiving their favorite schools, the students leaving the mechanism in the second round are receiving their favorite schools among those that remain, so their assignments cannot be improved without disturbing the assignments from the first round, and so forth. 

The assignment probabilities produced by GCPS are \emph{$sd$-efficient}: there are no other assignment probabilities that give each student a probability distribution over schools that first order stochastically dominates the one given by the GCPS assignment, with strict domination for some students.  Consequently any probability distribution over deterministic assignments that realizes the GCPS probabilities assigns positive probability only to ex post efficient assignments.    An important observation of BM is that, in the context of object allocation, random priority\footnote{In random priority for object allocation the agents are ordered randomly, the first agent chooses her favorite object, the second agent chooses her favorite of the remaining objects, and so forth.} may produce assignment probabilities that are not $sd$-efficient, even though it produces ex post efficient assignments.

In practice almost all school choice mechanisms limit the number of schools that a student can rank, and in very large districts 
such restrictions seem unavoidable.  If the GCPS mechanism is required to finish in a single round (other possibilities are described below) it makes sense to assign each student a \emph{safe school} which is guaranteed to accept the student if she is not admitted to a school she prefers to it, and which she may be required to attend if other schools do not admit her.  
We assume that for each school, the number of students for whom that school is the safe school is not greater than the school's capacity.  The GCPS mechanism requires that there is a probabilistic assignment that assigns all students to schools they are eligible for and that does not exceed any school's capacity, and our assumption insures that the assignment of each student to her safe school is such an assignment.

Mechanisms that would be strategy proof without restrictions on the number of schools that can be ranked become manipulable when such restrictions are imposed.  \cite{hk09jet} study the Nash equilibria of matching based mechanisms with such limitations.  \cite{chk10aer} is an experimental study of the effects of constraining the number of schools that can be ranked, for DA and TTC; a main finding is that constraints have a large negative effect on manipulability, and reduce efficiency and stability while increasing segregation.

In the GCPS mechanism with safe schools each student submits only a ranking of those schools she weakly prefers to her safe school.  
Safe schools are also possible with DA: if each student has the highest possible priority at her safe school, and each school has enough capacity for all the students for which it is the safe school, then a student will never be rejected by her safe school, and only needs to rank schools she prefers to it.
Both for GCPS and DA, strategy proofness is largely restored by having safe schools if most students prefer at most a small number of schools to the safe school. 
Of course having safe schools that students are likely to find  desirable is consistent with the main goal of school choice, which is to assign students to schools they would like to attend. Some systems (e.g., the state of Victoria in Australia) have \emph{neighborhood priority} in which each student's safe school is the one whose district contains her residence.

In the New York City High School Match as of 2006 \citep{Pat06} each student submitted a ranking of up to 12 schools.  Of the roughly 100,000 participants, over 8,000 were unmatched after the main round, in the sense that they were not offered a seat by any school they ranked.  These students submitted new rank ordered lists for the supplementary round, in which schools with unfilled capacity participated.  Students who did not receive a seat in the supplementary round were assigned administratively.  We do not know the particular considerations that motivated this design.  (One possibility is that neighborhood priority would have impeded a goal of school desegregation since there was a high degree of de facto residential segregation.) 

The important point for us is that GCPS can also be employed in multiround systems: each student's safe school in the first round is participation in the second round, for each student in the second round the safe school is participation in the third round or administrative assignment, and so forth.  In the remainder we assume a single round because this setting is simple, but rich enough to encompass the relevant technical issues.

The priorities in school choice mechanisms often express societal values.  
For example, in China \citep{WaZh20} the student's score on a standardized exam is taken to be her priority, presumably reflecting a policy objective of providing the most highly demanded resources, and the widest range of options, to the most talented students.  
Priorities may be affected by gender, minority status, and residential location.
In effect, DA computes a vector of priority cutoffs for the schools, and each student is matched with the most preferred school among those whose threshhold she exceeds.   (This is also the case for TTC if all schools have the same priorities.)  

One of the advantages of the GCPS mechanism is that the schools' priorities need not be strict.  At the other extreme, the GCPS mechanism makes it possible to have dichotomous priorities: each school gives equal consideration to all students who are qualified, perhaps by virtue of gender for single sex schools or test score cutoffs for selective schools.  Welfare analysis is simplified and clarified because one may consider only the priorities that express societal values, whereas DA and TTC may mix such priorities with arbitrary tie breaking.

When each school has finitely many priority classes, an ideal outcome for any particular school is that it has a cutoff priority class, students with lower priority are not admitted, students with higher priority have no probability of being required to attend a school they like less, and the school's capacity is exactly utilized.  To achieve this with GCPS one can run the GCPS mechanism repeatedly while adjusting the cutoff priority classes, and the numbers of seats allocated to the cutoff classes, until the desired balance is achieved, simultaneously at all schools, as described in Section \ref{sec:Nonbinary}.  For example, in a system with two selective schools, one can adjust both schools' test score cutoffs, and the numbers of seats assigned to the respective cutoff scores, until balance at both schools is achieved. 

Two technical innovations underlie the computational feasibility of the GCPS mechanism.   To facilitate the discussion we quickly review some basic results (without proofs) and terminology. 

A \emph{polytope} $Q$ may be defined to be the convex hull of a finite set of points, or as an intersection of finitely many closed half spaces that happens to be bounded.  To avoid technical detail our discussion in this paragraph assumes that $Q$ is full dimensional, in the sense that its affine hull is the entire Euclidean space of which it is a subset. Among the finite systems of weak linear inequalities that may be used to define $Q$, there is a unique (up to rescaling of inequalities by multiplication by positive scalars) such system that is minimal, and that is contained in any other such system.  Its elements are the \emph{facet inequalities} of $Q$.  For each facet inequality the corresponding \emph{facet} is the subset of $Q$ on which the facet inequality holds with equality.  A subset of $Q$ is a \emph{face} if it is $Q$ itself, the null set, or the intersection of some set of facets.  A polytope $Q$ is the convex hull of a finite set of points, and among the finite sets whose convex hulls are $Q$, there is a unique such set that is minimal in the sense that it is contained in any other such set, whose elements are the \emph{vertices} of $Q$.  The vertices of $Q$ may also be described as its extreme points, where an \emph{extreme point} of $Q$ is a point that cannot be expressed as a convex combination of other points of $Q$.

In \cite{balbuzanov22jet} the set of feasible allocations is a given polytope $Q$ in the nonnegative orthant of the space of matrices of assignment probabilities. (\cite{EcMiZh21} follow this approach in their study of pseudo-market equilibria with constraints.) Let $R$ be the intersection of the nonnegative orthant with the sum of $Q$ and the nonpositive orthant.  That is, a point in the nonnegative orthant is in  $R$ if and only if it lies below some point of $Q$.  

The GCPS allocation process is a piecewise linear function $p \colon [0,1] \to R$.  It begins with $p(0)$ equal to the origin and increases each student's probability of receiving her favorite school, among those she is allowed to consume, until one of the facet inequalities of $R$ is encountered.  
A key result (Balbuzanov's Proposition 1) is that the facet inequalities of $R$ (other than the nonnegativity conditions) require that weighted sums of probabilities, with nonnegative weights, not exceed certain quantities.  When the process encounters one or more facet inequalities  each student's set of allowed objects is updated by disallowing further consumption of probabilities that would result in one of these facet inequalities being violated.   The process then continues, with each student increasing the probability of receiving her favorite allowed school until additional facet inequalities of $R$ are encountered, and again the students' sets of allowed schools are updated. (For the problems we study each student's set of allowed schools is always nonempty.) Eventually the process arrives at a point  $p(1) \in Q$ that is, by definition, the GCPS allocation.

A computational implementation of the GCPS mechanism must have a way of detecting when the allocation process encounters a facet of $R$.  Our first main result is a generalization of Hall's marriage theorem that gives a set of inequalities, in closed form, that contains the facet inequalities of $R$.   An implementation that scans over all inequalities has been implemented, and works well if the number of schools is not too large, say not more than 25.  Furthermore, there are reasons to hope that it might be tractable for much larger problems.

Our second main innovation is a computational procedure that drastically speeds up the computation of the GCPS allocation, especially for large problems. In addition to computing $p$, it also computes a piecewise linear path $\barp \colon [0,1] \to Q$ such that $p(t) \le \barp(t)$ for all $t$.  Having chosen a trajectory for $\barp$, $p$ continues along the trajectory given by the students' favorite allowed schools, and $\barp$ continues along the chosen trajectory, until the time at which some resource constraint is encountered by $p$, or continuing further would result either in $\barp$ leaving $Q$ or $p$ no longer lying below $\barp$.  At that time a combinatoric calculation (described in Section \ref{sec:Procedure}) of bounded complexity gives either a new trajectory for $\barp$ or a facet inequality of $R$ that is satisfied by $p$ and $\barp$ at that time.

We briefly describe the structure of the remainder.  The next section reviews related literature.  Section \ref{sec:GenHall} states and proves our generalization of Hall's marriage theorem.  During the allocation process there can be a \emph{critical pair} consisting of a set $J$ of agents and a set $P$ of objects such that the agents in $J$ must be assigned all of the remaining capacity of the objects in $P$.  Section \ref{sec:Critical} studies such pairs.  Section \ref{sec:Procedure} describes the algorithm for computing the GCPS allocation.  Section \ref{sec:Nonbinary} describes an interative adjustment procedure in which the GCPS allocation is computed repeatedly in order to optimize the allocation in relation to nondichotomous priorities.

Section \ref{sec:Efficiency} shows that GCPS allocations are $sd$-efficient, and also efficient in relation to other orderings of the set of probability measures on objects derived from an ordinal preference that correspond to the limits of extreme risk loving and risk averse cardinal preferences.
Section \ref{sec:Fairness} considers the fairness properties of GCPS allocations.  Section \ref{sec:StrategyProof} argues that although the GCPS mechanism is not fully strategy proof, it is very difficult to manipulate in its application to school choice, and we present two theoretical results in this direction.  Section \ref{sec:Conclusion} provides some concluding remarks.  

Online Appendix \ref{sec:Implementability} describes a special case of an algorithm of \cite{bckm13aer} that passes from a matrix of assignment probabilities to a random deterministic assignment whose distribution realizes the given probabilities.  Online Appendix \ref{sec:GCPSSchools} gives a brief informal description of the software package \texttt{GCPS Schools}, which implements the algorithms described in Section \ref{sec:Procedure} and Online Appendix \ref{sec:Implementability}. \emph{GCPS Schools} also contains an application that generates sample school choice problems.  Online Appendix \ref{app:Eating} contains the proofs of the results of Section \ref{sec:StrategyProof}.

\section{Background and Related Literature}

The literature on school choice is now vast; \citet{aa22nber} is a recent survey. In this section  we survey some of the literature that is most closely related our work. 

In response to the inefficiencies observed by \cite{apr09aer}, a rather extensive literature (\citealp{ee08aer,kesten10qje,TaYu14,ku15te}) studies how the outcome of DA might be adjusted ex post, possibly by substituting new priorities.  In fact there are theoretical barriers to improving efficiency by manipulating the breaking of ties in the schools' rankings.  \cite{GaSh62} show that DA yields the best outcome for each student that can be achieved in any allocation without justified envy for the given priorities.   Improving on results of \cite{kesten06jet} and \cite{ee08aer}, Theorem 1 of \cite{apr09aer} asserts that, for any member of a large class of tie breaking rules, there is no mechanism that is both strategy proof for that tie breaking rule and gives outcomes that weakly Pareto dominate those produced by DA. 

Although it is not widely used, TTC continues to be a topic of research.  Some papers \citep{Mor15,HaKe18,Gri23} proposed modified versions of the mechanism.  \cite{LeLo20} analyze it in terms of cutoffs for the schools.

The GCPS mechanism may be applied to domains other than school choice.  For example, motivated by matching of medical residents with hospitals in Japan and similar problems, \cite{KaKo15,KaKo17} study mechanisms in which regional caps on the number of residencies are implemented by imposing caps on the number of residencies at individual hospitals in the region.   This can lead to a hospital rejecting applicants as a result of the hospital's cap even though other hospitals in the region have unfilled vacancies.  They propose a more flexible version of DA in which some hospitals are allowed to exceed their caps if the total number of doctors matched to the region is below the region's cap.  Similar effects can be achieved by running the GCPS mechanism repeatedly while adjusting the caps of individual hospitals.  

One way to implement affirmative action objectives has been suggested by \cite{as03aer}.  For example, a school may be divided into three subschools, one with 30\% of the seats that is reserved for minority students, one with 30\% of the seats that is reserved for majority students, and one with 40\% of the seats that accepts all students. ``Hard'' upper and lower bounds for the percentages of students of different types are extensively used in practice, but \cite{Koj12} and \cite{HaYeYi13} point out that they lead to conflicts with other objectives, and \cite{EHYY14} suggest implementing affirmative action goals using soft bounds.  Such an approach can be implemented, at least informally, by running the GCPS algorithm multiple times while adjusting the parameters to better reconcile competing objectives. 

We now describe the PS mechanism of BM and subsequent generalizations.  BM study the problem of assigning a different object from a finite set to each of finitely many agents, based on their reported strict ordinal preferences.
BM provide an intuitive description of the PS mechanism in which each object is regarded as a perfectly divisible cake of unit size.  At each moment in the unit interval of time each agent consumes, at unit speed, probability of her favorite cake, among those that have not yet been fully consumed.  Provided that there are at least as many objects as agents, at time 1 each agent has a probability distribution over the objects, and for each object the sum of the assignment probabilities is not greater than one.    Among the most important theoretical results are that the PS mechanism is $sd$-efficient but not strategy proof. 

Extensions of BM's cake eating procedure have been proposed in (at least) six other papers.  Using the method of network flows (see Section \ref{sec:GenHall}) \cite{KaSe06} extend the PS mechanism to profiles of preferences with indifferences.  Their mechanism has both the PS mechanism for strict preferences and the mechanism proposed by \cite{bm04} for matching problems with dichotomous preferences as special cases. \cite{bog15} provides a welfarist characterization of it.

\cite{kojima09mss} studies perhaps the simplest extension of BM in which agents receive multiple objects.  Each agent receives $r \ge 2$ objects, and the number of objects is $r$ times the number of agents.  The mechanism is shown to be $sd$-efficient and envy-free, but not weakly strategy proof, as we explain in more detail in Section \ref{sec:StrategyProof}.

\cite{yilmaz10geb} studies house allocation problems with existing tenants, which are object allocation problems in which some objects have owners who can insist on not receiving a worse object.  He proposes the special case of the mechanism studied here for that problem, and in particular he recognizes the relationship between Hall's marriage theorem, its generalization by \cite{Gal57}, and the set of feasible allocations.  His algorithm is generalized by \cite{AtSe11} to problems in which agents have fractional endowments.  \cite{yilmaz09geb} uses the methods of \cite{KaSe06} to extend the mechanism to the domain of preferences with indifferences.

\cite{bckm13aer} study problems in which there are constraints that require that certain sums of probabilities are bounded, either below, in which case the constraint is a \emph{floor constraint}, or above, in which case it is a \emph{ceiling constraint}.  For a problem with only ceiling constraints in which  there is a ``null object'' (e.g., being unemployed, unhoused, or unschooled) that is available in infinite supply, and which is not involved in any constraint, they propose a \emph{generalized probabilistic serial} (GPS) mechanism.  As in BM, at each moment in $[0,1]$ each agent increases her probability of her favorite available object.  When a ceiling constraint binds with equality, the sets of available objects are revised by disallowing further consumption of probabilities that would violate a constraint.  Since the null object is always available, each agent's set of available objects is always nonempty.  Thus at time 1 each agent has total probability one, and the GPS assignment is defined as the probability shares that have been consumed by each agent at time 1.

As we have already described, \cite{balbuzanov22jet} generalizes the \cite{bckm13aer} mechanism by allowing the set of feasible allocations to be an arbitrary polytope $Q$ in the nonnegative orthant of the space of matrices of assignment probabilities. 

\section{A Generalized Hall's Marriage Theorem} \label{sec:GenHall}

In this section we introduce the formal framework, state and prove the generalization of Hall's theorem, and provide useful characterizations of $Q$ and $R$.

A \emph{communal endowment economy} (CEE) is a quintuple $E = (I,O,r,q,g)$ in which $I$ is a nonempty finite set of \emph{agents}, $O$ is a nonempty finite set of \emph{objects}, $r \in \Re_+^I$, $q \in \Re_+^O$, and $g \in \Re_+^{I \times O}$.  We say that $r_i$ is
$i$'s \emph{requirement}, $q_o$ is the \emph{quota} of $o$, and $g_{io}$ is \emph{$i$'s $o$-max}.  In comparison with most models of random assignment, the matrix $g$ is the main novelty, and we will see that it may represent several things and be used in various ways. 

Several types of CEE occur in our discussion. We say that $E$ is \emph{integral} if $r \in \In_+^I$, $q \in \In_+^O$, and $g \in \In_+^{I \times O}$.  A \emph{Gale supply-demand CEE}  is a CEE $E$ such that $g_{io} \in \{0,r_i\}$ for all $i \in I$ and $o \in O$. We say that $E$ is a \emph{school choice CEE} if $r_i = 1$.   In this case elements of $I$ are \emph{students} and elements of $O$ are \emph{schools}, and each student must receive a seat in some school. In an integral school choice CEE each school has an integral number of seats, and for each student $i$ and school $o$, $g_{io} = 1$ if $i$ is eligible to attend $o$ and weakly prefers it to her safe school, and otherwise $g_{io} = 0$.  (In Section \ref{sec:Nonbinary} we will describe a method that uses fractional $g_{io}$ to ``fine tune'' the GCPS allocation.)
A \emph{Hall marriage problem} is a CEE such that for all $i$ and $o$, $r_i = 1$, $q_o = 1$, and $g_{io} \in \{0,1\}$.   In this case elements of $I$ are \emph{boys} and elements of $O$ are \emph{girls}.
Intuitively a Hall marriage problem is a bipartite graph with an edge connecting boy $i$ to girl $o$ if $i$ and $o$ are compatible.  

An \emph{allocation} for $I$ and $O$ is a
matrix $p \in \Re_+^{I \times O}$.  Such a $p$ is \emph{integral} if $p \in \In_+^{I \times O}$. In general, for any matrix $p \in \Re^{I \times O}$ and any $i \in I$ and $o \in O$, $p_i = (p_{io})_{o \in O}$ and $p_o = (_{io})_{i \in I}$ are the corresponding row and column.  A \emph{partial allocation} for $E$ is an allocation $p$ such that $\sum_o p_o \le r$, $\sum_i p_i \le q$, and $p_{io} \le g_{io}$ for all $i$ and $o$. A \emph{feasible allocation} is a partial allocation $m$ such that $\sum_o m_o = r$. 
A partial allocation $p$ is \emph{possible} if there is a feasible allocation $m$ such that $p \le m$.
Let $Q$ be the set of feasible allocations, and let $R$ be the set of possible partial allocations.

For $J \subset I$ and $P \subset O$ let $J^c = I \setminus J$ and $P^c = O \setminus P$ be the complements.  We say that $E$ satisfies the \emph{generalized marriage condition}
(GMC) if,  for every $J \subset I$ and $P \subset O$,
$$\sum_{i \in J} r_i \le \sum_{i \in J} \sum_{o \in P^c} g_{io} + \sum_{o \in P} q_o.$$  We will refer to this relation as the \emph{GMC inequality} for $(J,P)$.  Note that the GMC inequality for  $(\{i\},\emptyset)$ is  $r_i \le \sum_o g_{io}$,  and the GMC inequality for  $(I,O)$ is  $\sum_i r_i \le \sum_o q_o$.  The GMC is obviously necessary for the existence of a feasible allocation.
Our first main result is:

\begin{thm} \label{th:MultiHall}
  The CEE $E$ has a feasible allocation if and only if it satisfies the GMC.
\end{thm}

\noindent
The \cite{Gal57} supply-demand theorem\footnote{
Although this result is attributed to \cite{Gal57} by \cite{yilmaz10geb}, and perhaps others, this exact formulation does not appear in Gale's paper.  The paper does consider slightly more complicated problems, and it is easy to see that this result can be obtained from Gale's methods in the same manner.
} is the special case of this for a Gale supply-demand CEE.

If $E$ is a Hall marriage problem, the  set of \emph{neighbors} of boy $i$ is $N_g(i) = \{\, o \in O : g_{io} = 1
\,\}$, and for $J \subset I$ we set $N_g(J) = \bigcup_{i \in J}
N_g(i)$.  We say that $E$ satisfies the \emph{marriage condition} if $|J| \le |N_g(J)|$ for all $J \subset I$.  
The GMC inequality for $J$ and $P = N_g(J)$ gives this inequality.  Conversely, for a
given $J \subset I$, the contribution of $o \in N_g(J)$ to the right
hand side of the GMC inequality is minimized if $o \in P$, and the
contribution of $o \in N_g(J)^c$ is minimized if $o \in
P^c$, so $|J| \le |N_g(J)|$ for all $J$ implies that the GMC is
satisfied.  Therefore Theorem \ref{th:MultiHall} implies that $E$ has a feasible allocation if and only if the marriage condition is satisfied.

For a Hall marriage problem an integral feasible allocation is called a \emph{matching}.  (Each of the boys has a different partner.)  Hall's marriage theorem asserts that a Hall marriage problem has a matching if
and only if it satisfies the marriage condition.  
To pass from a feasible allocation to a matching one can repeatedly adjust the allocation along paths of fractional allocations that alternate between boys and girls, and either form a loop or pass from one incompletely allocated girl to another.  A more precise and general version of this argument is given in  Section \ref{sec:Implementability}.

For $i \in I$ let $$\alpha_i = \{\, o \in O : g_{io} > 0 \,\}$$ be the set of objects that are \emph{possible} for $i$, and for $P \subset O$ let
$J_P = \{\, i \in I : \alpha_i \subset P \,\}$ be the set of agents who cannot be allocated objects outside of $P$.
If $E$ is an integral school choice CEE, then for any $P \subset O$, $J_P$ minimizes the difference between the right hand side and the left hand side of the GMC inequality.  Therefore $E$ satisfies the GMC if and only if, for each $P \subset O$, $$|J_P| \le \sum_{o \in P} q_o.$$

Our proof of Theorem \ref{th:MultiHall} is a simple application of the method of network flows.  (\cite{AhMaOr93} provides a general introduction and overview.)  Let $(N,A)$ be a directed graph ($N$ is a finite set of \emph{nodes} and $A \subset N \times N$ is a set of \emph{arcs}) with distinct distinguished nodes $s$ and $t$, called the \emph{source} and \emph{sink} respectively.  For the sake of simplicity and clarity of intuition (the formal analysis can be more general) we assume that $(n, s), (t,n), (n,n) \notin A$ for all $n \in N$.

A \emph{flow} is a function $f \colon N \times N \to \Re$ such that:
\begin{enumerate}
  \item[(a)] for all $n$ and $n'$, if $(n,n') \notin A$, then $f(n,n') \le 0$ ; 
  \item[(b)] for all $n$ and $n'$,  $f(n,n') = - f(n',n)$; 
  \item[(c)] $\sum_{n' \in N} f(n',n) = 0$ for all $n \in N \setminus \{s,t\}$. 
\end{enumerate}
If neither $(n,n')$ nor $(n',n)$ is in $A$, then (a) and (b) imply that $f(n,n') = 0$.  
In conjunction with the other requirements, (c) can be understood as saying that for each $n$ other than $s$ and $t$, the total flow into $n$ is equal to the total flow out.  Note that (a) and (b) imply that $f(s,n), f(n,t) \ge 0$ for all $n \in N$.  Applying (b), then (c), gives
$$0 = \sum_{n \in N}\sum_{n' \in N} f(n',n) = \sum_{n' \in N} f(n',s) + \sum_{n' \in N} f(n',t),$$
so we may define \emph{value} of $f$ to be
$$|f| = \sum_{n \in N} f(s,n) = \sum_{n \in N} f(n,t).$$

A \emph{capacity} is a function $c \colon N \times N \to [0,\infty]$ such that $c(n,n') = 0$ whenever $(n,n') \notin A$.  
A flow $f$ is \emph{bounded} by a capacity $c$ if $f(n,n') \le c(n,n')$ for all $(n,n')$.  A \emph{maximum flow} for $c$ is a flow $f$ that is maximal for $|f|$ among those flows bounded by $c$.  

A \emph{cut} is a set $S \subset N$ such that $s \in S$ and $t \in S^c$ where $S^c = N \setminus S$ is the complement.  For a capacity $c$, the \emph{capacity} of $S$ is
$$c(S) = \sum_{(n,n') \in S \times S^c} c(n,n').$$  The value $|f|$ is the net flow from $S$ to $S^c$, hence the total flow from $S$ to $S^c$ minus the total flow from $S^c$ to $S$, so $|f| \le c(S)$ when $f$ is bounded by $c$, and thus the maximum value of flows bounded by $c$ is not greater than the minimum capacity of a cut for $c$.  The max-flow min-cut theorem \citep{FoFu56} asserts that these two quantities are equal. 
  
A consequence of this is that if $f$ is a maximal flow for $c$, then a cut $S$ has minimal capacity if and only if $$|f| = \sum_{(n,n') \in S \times S^c} f(n,n') = c(S).$$  Since $c$ bounds $f$, this holds if and only if $f(n,n') = c(n,n')$ for all $(n,n') \in S \times S^c$.  
It is well known \citep{FoFu56,Sha61,Ore62} that the set of minimal cuts is a lattice in the sense that if $S_1$ and $S_2$ are minimal cuts, then so are $S_1 \cup S_2$ and $S_1 \cap S_2$\footnote{If $n \in S_1 \cup S_2$ and $n' \in (S_1 \cup S_2)^c = S_1^c \cap S_2^c$ 
(or $n \in S_1 \cap S_2$ and $n' \in S_1^c \cup S_2^c$) 
then either $n \in S_1$ and $n' \in S_1^c$ or $n \in S_2$ and $n' \in S_2^c$, and in either case $f(n,n') = c(n,n')$.}.

For the given CEE $E$ we define a particular directed graph $(N_E,A_E)$ in which
$N_E = \{s\} \cup I \cup  O \cup \{t\}$ and
$$A_E = \{\, a_i : i \in I \,\} \cup \{\, a_{io} : i \in I, o \in O \,\} \cup \{\, a_o : o \in O \,\}$$
where, for $i \in I$ and $o \in O$, $a_i = (s,i)$, $a_{io} = (i,o)$, and $a_o = (o,t)$.
Let $c_E$ be the capacity in which 
$c_E(a_i) = r_i$, $c_E(a_{io}) = g_{io}$, and $c_E(a_o) = q_o$.  
If $p$ is an allocation, there is a unique flow $f_p$ such that $f_p(a_{io}) = p_{io}$ for all $i$ and $o$ that has $f_p(a_i) = \sum_o p_{io}$ for all $i$ and $f_p(a_o) = \sum_i p_{io}$ for all $o$. Evidently $p$ is a partial allocation if and only if $f_p$ is bounded by $c_E$, and it is a feasible allocation if and only if, in addition, $f_p(a_i) = r_i$ for all $i$, which is the case if and only if $|f_p| = \sum_i r_i$.  Conversely, if $f$ is a flow bounded by $c_E$ with $|f| = \sum_i r_i$ and thus $f(a_i) = r_i$ for all $i$, then setting $m_{io} = f(a_{io})$ gives a feasible allocation $m$.

Thus there is a feasible allocation if and only if the maximum value of a flow bounded by $c_E$ is $\sum_i r_i$.  
The max flow-min cut theorem implies that this is the case if and only if the minimum capacity of a cut for $c_E$ is $\sum_i r_i$.  
Since $c_E(\{s\}) = \sum_i r_i$, there is a feasible allocation if and only $c_E(S) \ge \sum_i r_i$ for all cuts $S$.  

For $J \subset I$ and $P \subset O$ let $S_{(J,P)} = \{s\} \cup J \cup P$.  This is a cut, and if $S$ is a cut, then $S = S_{(J,P)}$ where  $J = S \cap I$ and $P = S \cap O$.  An arc can go from a node in $S_{(J,P)}$ to a node in $S_{(J,P)}^c$ by going from $s$ to a node in $J^c$, by going from a node in $J$ to a node in $P^c$, and by going from a node in $P$ to $t$, so
$$c_E(S_{(J,P)}) = \sum_{i \in J^c} r_i + \sum_{i \in J} \sum_{o \in P^c} g_{io} + \sum_{o \in P} q_o.$$
Thus there is a feasible allocation if and only if $\sum_i r_i \le c_E(S_{(J,P)})$ for all $J \subset I$ and $P \subset O$, and subtracting $\sum_{i \in J^c} r_i$ from both sides reveals that this inequality is equivalent to the GMC inequality for $J$ and $P$.

Hall's marriage theorem, the Gale supply-demand theorem, and the max-flow min-cut theorem are three members of a large and important class of results in combinatorial matching theory that are equivalent in the informal sense that relatively simple arguments (described in detail by \cite{Rei78,Rei85}) allow one to pass from any member of the class to any other.  As yet another member of this class, Theorem \ref{th:MultiHall} does not provide distinctly novel mathematical information.  Its primary significance here, and perhaps more generally, is that the test it provides is in closed form.

 We now give useful characterizations of the set $Q$ of feasible allocations and the set $R$ of possible allocations.
If $p$ is a partial allocation, let $$E - p = (I,O,r',q',g')$$ be the derived CEE in which $r_i' = r_i - \sum_o p_{io},$ $q'_o = q_o - \sum_i p_{io},$ and $g'_{io} = g_{io} - p_{io}.$  If $p$ is a partial allocation, $m$ is an allocation, and $p \le m$, then $m$ is a feasible allocation for $E$ if and only if $m - p$ is a feasible allocation for $E - p$.  Thus a partial allocation $p$ is possible if and only if $E - p$ has a feasible allocation, which of course is the case if and only if $E - p$ satisfies the GMC.  Substituting the definitions above into the GMC inequality for $E - p$ and $(J,P)$, then simplifying, gives
\begin{equation} \label{eq:prop1}
\sum_{i \in J^c}\sum_{o \in P} p_{io} \le -\sum_i r_i + \sum_{i \in J}\sum_{o \in P^c} g_{io} + \sum_{o \in P} q_o.
\end{equation}

\begin{prop}
  An allocation $p$ such that $p_{io} \le g_{io}$ and $\sum_o p_{io} \le r_i$ for all $i$ and $o$ is possible if and only if \eqref{eq:prop1} holds for all $J \subset I$ and $P \subset O$.
\end{prop}

Now consider $m \in \Re^{I \times O}_+$ such that $m_{io} \le g_{io}$ and $\sum_o m_{io} = r_i$ for all $i$ and $o$, and let $E_m = (I,O,r,q,m)$ be the CEE obtained by replacing $g$ with $m$.  If $m$ is a feasible allocation for $E$, then it is a feasible allocation for $E_m$, so $E_m$ satisfies the GMC.  Conversely, if $E_m$ satisfies the GMC, then it has a feasible allocation $m'$, but since $m' \le m$ and $r_i = \sum_o m_{io}' = \sum_o m_{io}$ for all $i$, the only possibility is $m' = m$, so $m$ is a feasible allocation for $E_m$ and thus also for $E$.  Thus $m$ is a feasible allocation if and only if $E_m$ satisfies the GMC.  Since $r_i = \sum_{o \in P} m_{io} + \sum_{o \in P^c} m_{io}$ for each $i$, we arrive at:

\begin{prop} \label{prop:mFeasible}
  $Q$ is the set of $m \in \Re^{I \times O}_+$ such that $m_{io} \le g_{io}$ and $\sum_o m_{io} = r_i$ for all $i$ and $o$ and, for all $J \subset I$ and $P \subset O$, $\sum_{i \in J} \sum_{o \in P} m_{io} \le \sum_{o \in P} q_o.$
\end{prop}

\section{Critical Pairs} \label{sec:Critical}

In this section we work with a given CEE $E$ that satisfies the GMC.
For $J \subset I$ and $P \subset O$ we say that the pair $(J,P)$ is \emph{critical} for $E$ if $(J,P) \ne (\emptyset,\emptyset)$ and it satisfies the GMC inequality for $(J,P)$ with equality: $$\sum_{i \in J} r_i = \sum_{i \in J} \sum_{o
\in P^c} g_{io} + \sum_{o \in P} q_o.$$  We refer to this condition as the \emph{GMC equality} for $(J,P)$. Our goal in this section is to understand the relationship between critical pairs and feasible allocations, and how the various critical pairs for $E$ are related.  

Evidently, if $(J,P)$  is critical for $E$, then any feasible allocation $m$ gives the agents in $J$ all of the endowment of objects in $P$ and also as much of the objects in $P^c$ as $g$ allows.  Conversely, if $m$ is a feasible allocation such that $\sum_{i \in J} m_{io} = q_o$ for all $o \in P$ and $m_{io} = g_{io}$ for all $i \in J$ and $o \in P^c$, then $$\sum_{i \in J} r_i = \sum_{i \in J} \sum_o m_{io} = \sum_{i \in J} \sum_{o \in P^c} m_{io} + \sum_{o \in P} \sum_{i \in J} m_{io} = \sum_{i \in J} \sum_{o \in P^c} g_{io} + \sum_{o \in P} q_o.$$

\begin{lem} \label{lem:critical}
  For $J \subset I$ and $P \subset O$ the following are equivalent:
  \begin{enumerate}
    \item[(a)] $(J,P)$ is critical for $E$; 
    \item[(b)] There is a feasible allocation $m$ such that  $\sum_{i \in J} m_{io} = q_o$ for all $o \in P$ and $m_{io} = g_{io}$ for all $i \in J$ and $o \in P^c$;
    \item[(b)] For every feasible allocation $m$, $\sum_{i \in J} m_{io} = q_o$ for all $o \in P$ and $m_{io} = g_{io}$ for all $i \in J$ and $o \in P^c$.
  \end{enumerate}
\end{lem}
\noindent In particular, if $(J,\emptyset)$ is critical, then then every feasible allocation $m$ has $m_{io} = g_{io}$ for all $i \in J$ and $o \in O$, and if $(\emptyset,P)$ is critical, then $q_o = 0$ for all $o \in P$.  (The latter situation will arise during the allocation process as objects' quotas are exhausted.)


If $(J,P)$ is critical for $E$, let 
$$E_{(J,P)} = (J,O,r|_J,q',g|_{J \times O}) \quad \text{and} \quad E^{(J,P)} = (J^c, P^c, r|_{J^c},q'',g|_{J^c \times P^c})$$ where $q'_o = q_o$ if $o \in P$, $q'_o = \sum_{i \in J} g_{io}$ if $o \in P^c$, and $q'' \colon P^c \to \Re_+$ is the function $q''_o = q_o - \sum_{i \in J} g_{io}$. Any feasible allocation for $E$ is the sum of a feasible  allocation for $E_{(J,P)}$ and a feasible allocation  for  $E^{(J,P)}$, so $E_{(J,P)}$ and  $E^{(J,P)}$ satisfy the GMC.  Conversely, any sum of a feasible allocation for $E_{(J,P)}$ and a feasible allocation for $E^{(J,P)}$ is a feasible allocation for $E$. Thus a critical pair splits the given allocation problem into two smaller problems of the same type.  This is very important because it allows our algorithms to be recursive.

We say that $E$ is \emph{critical} if $(I,O)$ itself is a critical pair, which is the case if and only if $\sum_i r_i = \sum_o q_o$, so that any feasible allocation consumes all of the available resources. 
If $(J,P)$ is a critical pair for $E$, then $E_{(J,P)}$ is critical, and $E^{(J,P)}$ is critical if and only if $E$ is critical.

We say that $E$ is \emph{simple} if there are no critical pairs $(J,P)$ with $(J,P) \ne (I,O)$.  A critical pair $(J,P)$ is \emph{minimal} if there is no critical pair $(J',P')$ with $J' \subset J$, $P' \subset P$, and $(J',P') \ne (J,P)$.  The next result (whose proof follows easily from the discussion above and is therefore left as an exercise) implies that if $(J,P)$ is a minimal critical pair for $E$, then $E_{(J,P)}$ is simple.

\begin{lem} \label{lemma:MinimalSimple} 
  If $(J,P)$ is critical for $E$, $J' \subset J$, and $P' \subset P$, then $(J',P')$ is critical for $E$ if and only if it is critical for $E_{(J,P)}$.
\end{lem}

% \begin{proof} % [Proof of Lemma \ref{lemma:MinimalSimple}]
%   Any feasible allocation for $E$  gives the agents in $J$ all of the endowment of objects in $P$ and also as much of the objects in $P^c$ 
%   as $g$ allows, and its restriction to $J \times O$ is a feasible allocation for $E_{(J,P)}$, so if $(J',P')$ is critical for $E_{(J,P)}$, 
%   then its restriction to $J' \times O$ gives the agents in $J'$ all of the endowment of objects in $P'$ and as much of the objects 
%   in ${P'}^c$ as $g$ allows.  Thus $(J',P')$ is critical for $E$ if it is critical for $E_{(J,P)}$.
  
%   Any feasible allocation for $E$ is the sum of a feasible allocation for $E_{(J,P)}$ and a feasible allocation for $E^{(J,P)}$, 
%   and any feasible allocation for $E_{(J,P)}$ can be added to this feasible allocation for $E^{(J,P)}$ to obtain a feasible 
%   allocation for $E$.  Therefore if $(J',P')$ is critical for $E$, then any feasible allocation for $E_{(J,P)}$  gives the 
%   agents in $J'$ all of the endowment of objects in $P'$ and as much of the objects in ${P'}^c$ as $g$ allows.  
%  Thus $(J',P')$ is critical for $E_{(J,P)}$ if it is critical for $E$.
% \end{proof}

A pair $(J,P)$ is critical if and only if $\sum_i r_i = c_E(S_{(J,P)})$, i.e., $S_{(J,P)}$ is a minimal cut for $c_E$.  For any pairs $(J,P)$ and $(J',P')$ the definition of $S_{(J,P)}$ easily implies that $S_{(J,P)} \cup S_{(J',P')} = S_{(J \cup J',P \cup P')}$ and $S_{(J,P)} \cap S_{(J',P')} = S_{(J \cap J',P \cap P')}.$ Since the set of minimal cuts for $c_E$ is a lattice we have:

\begin{prop}
The set of critical pairs for $E$ is a lattice in the sense that if  $(J,P)$ and $(J',P')$
are critical pairs, then so are $(J \cup J',P \cup P')$ and $(J \cap J',P \cap P')$.  
\end{prop}

If $m$ is a feasible allocation for $E$, for $J \subset I$ and $P \subset O$ we say that the pair $(J,P)$ is \emph{$m$-critical} for $E$ if $\sum_{i \in J} \sum_{o \in P} m_{io} = \sum_{o \in P} q_o.$ Proposition \ref{prop:mFeasible} implies that $(J,P)$ is $m$-critical for $E$ if and only if $(J,P)$ is critical for $E_m$.  Thus:

\begin{cor} \label{cor:mCritical}
If $m$ is a feasible allocation for $E$, the set of $m$-critical pairs for $E$ is a lattice.  
\end{cor}

If $(J,P)$ is a critical pair, then any feasible allocation $m$ has $m_{io} = 0$ for all $i \in J^c$ and $o \in P$, and in this sense $g_{io} > 0$ is illusory. We say that $E$ is \emph{tight} if $g_{io} = 0$ for all critical pairs $(J,P)$ and all $i \in J^c$ and $o \in P$.   

If $(J,P)$ is a critical pair for $E$, the \emph{$(J,P)$-tightening of $E$} is $E' = (I,O,q,r,g')$ where $g'_{io} = 0$ if $i \in J^c$ and $o \in P$, and otherwise $g'_{io} = g_{io}$.  Since $E$ satisfies the GMC, it has a feasible allocation $m$, which necessarily has $m_{io} = 0$ for all $i \in J^c$ and $o \in P$, so it is a feasible allocation for $E'$, and consequently $E'$ satisfies the GMC.  
A \emph{tightening sequence} for $E$ is a sequence $(J_1,P_1), \ldots, (J_\ell,P_\ell)$ for which there is a sequence $E_0 = E, E_1, \ldots, E_\ell$ of CEE's such that for each $j = 1, \ldots, \ell$, $(J_j,P_j)$ is a critical pair for $E_{j-1}$ and $E_j$ is the $(J_j,P_j)$-tightening of $E_{j-1}$.  By induction each $E_j$ satisfies the GMC.  

The following result is obvious:

\begin{lem} \label{lem:criticalpair}
    If $E = (I,O,r,q,g)$ satisfies the GMC,  $(J,P)$ is a critical pair for $E$, $g' \le g$, $E' = (I,O,r,q,g')$, and $E'$ satisfies the GMC, then $(J,P)$ is a critical pair for $E'$.
\end{lem}

In view of the last result, if $(J_1,P_1), \ldots, (J_\ell,P_\ell)$ and $(J_1',P_1'), \ldots, (J_{\ell'}',P_{\ell'}')$ are tightening sequences, then so is $(J_1,P_1), \ldots, (J_\ell,P_\ell),(J_1',P_1'), \ldots, (J_{\ell'}',P_{\ell'}')$.  Therefore starting with $E$ and repeatedly tightening with respect to critical pairs, including pairs that become critical as a result of the tightening, until no further tightening is possible, leads to a tight CEE that is independent of the order of tightening, that we call the \emph{tightening of $E$}. 



% When $E$ is a tight school choice CEE we can say a bit more.  We say that $P$ is a \emph{critical set of schools} for $E$ 
% if $(J_P,P)$ is a critical pair for $E$.  

% \begin{lem} \label{lem:tight}
% If $E$ is a tight school choice CEE that satisfies the GMC and $P$ and $P'$ are critical sets of schools for $E$ with  $P \subset P'$, 
% then $(J_{P'} \setminus J_P, P' \setminus P)$ is critical, and $J_{P'} \setminus J_P = J_{P' \setminus P}$, so $P' \setminus P$ 
% is a critical set of schools for $E$.
% \end{lem}

% Thus, when $E$ is a tight school choice CEE that satisfies the GMC, every critical set of schools is a union  of minimal critical sets of schools.
% Recall (Lemma \ref{lemma:MinimalSimple}) that if $(J,P)$ is a critical pair, then $E_{(J,P)}$ is critical and simple.    We have the following decomposition result.

% \begin{prop}
%  If $E$ is a tight school choice CEE that satisfies the GMC, then there is a unique partition  $P_0, P_1, \ldots, P_k$ of $O$ such that $P_1, \ldots, P_k$ are the minimal critical sets of schools and $E_1 = E_{(J_{P_1},P_1)}, \ldots, E_k = E_{(J_{P_k},P_k)}$ are simple and critical.  In addition:
 % \begin{enumerate}
%    \item[(a)] If $E$ is critical, then $P_0$ is a minimal critical set of schools for $E$, and  $E_0 = E_{(J_{P_0},P_0)}$ is simple and critical.
%    \item[(b)] If $E$ is not critical, let $E_0 = (J_0,P_0,r|_{J_0},q',g|_{J_0 \times P_0})$ where $q' \colon P_0 \to \Re_+$ is the function $q'_o = q_o - \sum_{i \in I \setminus J_0} g_{io}$.  Then $E_0$ is simple and not critical.
% \end{enumerate}
% In either case $E_0, \ldots, E_k$ is called the \emph{simple decomposition} of $E$.
% \end{prop}

\section{The Allocation Procedure} \label{sec:Procedure}

We now describe how the GCPS mechanism can be computed.  We work with a fixed CEE $E = (I,O,r,q,g)$ that satisfies the GMC and a profile $\succ \; = (\succ_i)_{i \in I}$ of strict preferences\footnote{The mechanism actually depends only on the preferences of each agent $i$ over her set $\alpha_i$ of possible objects, but it is simplest to treat each $\succ_i$ as a complete ordering of $O$.} over $O$.
Let $T = \max_i r_i$.  The allocation procedure is a piecewise linear function $p \colon [0,T] \to R$ with $p(0) = 0$, $p(t) \in R \setminus Q$ for all $t < T$, and $p(T) \in Q$.  The \emph{GCPS allocation} is $$GCPS(E,\succ) = p(T).$$ 

At each moment the trajectory of $p$ increases, at unit speed, each agent's assignment of her favorite object, among those that are still available to her, while leaving other allocations fixed.  This direction is adjusted when an agent attains her requirement, when an agent $i$'s assignment of an object $o$ reaches $g_{io}$, and  when $p$ arrives at one of the facets of $R$.  If $t^*$ is the first time such that $p(t^*)$ is in a facet of $R$, so that for some pair $(J,P)$ the GMC equality holds, then
the residual CEE $E - p(t^*)$ is not simple and $(J,P)$ is a minimal critical pair for it.  The GCPS allocation has a recursive definition: for $t \in (t^*,T]$, $p(t)$ is, by definition, the sum of $p(t^*)$ and the results of applying the allocation procedure to $(E - p(t^*))_{(J,P)}$ and $(E - p(t^*))^{(J,P)}$ on the interval $[t^*,t]$.  

The main computational challenge is to detect  when $p$ arrives at one of the facets of $R$.  

One possible implementation first passes to the description of $Q$ as a convex hull of vertices.  The vertices of $R$ are all the points obtained from vertices of $Q$ by changing some of the components to zero, and one may then pass from this set of vertices to the description of $R$ as an intersection of finitely many half spaces.  The computational problem of passing from the description of a polytope as a convex hull of vertices to its description as an intersection of half spaces, and the reverse computation, are well studied, and efficient softwares for these tasks are available.  (See Section 3 of \cite{balbuzanov22jet}.)  However, even if the number of bounding inequalities of $Q$ and the number of bounding inequalities of $R$ are small, large data structures can arise at intermediate stages of the computation.  For example, for the problem of assigning $n$ objects to $n$ agents the numbers of facet inequalities of $Q$ and $R$ are constant multiples of $n$, but $Q$ has $n!$ vertices.

Theorem \ref{th:MultiHall} improves on this by showing that the facet inequalities of $R$ are a subset of the set of GMC inequalities.  For a given $P \subset O$ it is easy to find the $J \subset I$ that minimizes the difference between the two sides of the GMC for $(J,P)$.  Thus there is a computational burden that is roughly proportional to the number $2^{|P|}$ of subsets of $P$.  An algorithm using this approach has been implemented, and works reasonably well for moderate (roughly $|P| \le 50$) numbers of schools.

The procedure we describe now is much more efficient, especially for large problems. During the computation it computes an auxilary piecewise linear function $\barp \colon [0,T] \to Q$ such that $p(t) \le \barp(t)$ for all $t$.   We assume that $\barp(0)$ is given.  Possibly $\barp(0)$ is the assignment of safe schools, or it may be the output of an algorithm that computes a maximal flow for the network $(N_E,A_E)$.

The combined function $(p,\barp)$ is piecewise linear, and $[0,T]$ is a finite union of intervals $[t_0,t_1], [t_1, t_2], \ldots, [t_{K-1}, t_K]$, where $t_0 = 0$ and $t_K = T$, such that on each interval $[t_k,t_{k+1}]$ the derivative of $(p,\barp)$ is constant.
Suppose that we have already computed $p(t_k)$ and $\barp(t_k)$.  For each agent $i$ we compute the set $\alpha_i(t_k)$ of objects that are still possible for $i$, and we determine her $\succ_i$-favorite element $e_i^k$.  Let $\theta^k \in \In^{I \times O}$ be the matrix such that $\theta^k_{io} = 1$ if $o = e_i^k$, and otherwise $\theta^k_{io} = 0$.

There are now two possibilities.  The first is that there is some $t' > t_k$ such that $p(t_k) + \theta^k(t - t_k) \in R$ for all $t \in [t_k,t']$.  In this case we will find a $\theta \in \In^{I \times O}$ such that for some $t' > t_k$ and all $t \in [t_k,t']$, $\barp(t_k) + \theta(t - t_k) \in Q$ and $$p(t_k) + \theta^k(t - t_k) \le \barp(t_k) + \theta(t - t_k). \eqno{(*)}$$  Now $t_{k+1}$ is the first time after  $t_k$ such that one of the following holds: a) $t_{k+1} = r_i$ for some $i$; b) 
$p_{ie_i^k}(t_{k+1}) = g_{ie_i^k}$ for some $i$; c) $\barp(t_k) + \theta(t - t_k) \notin Q$ for $t > t_{k+1}$; d) ($*$) does not hold  for $t > t_{k+1}$.
For $t \in [t_k,t_{k+1}]$ we have determined that $p(t) = p(t_k) + \theta^k(t - t_k)$, and we set $\barp(t) = \barp(t_k) + \theta(t - t_k)$, then repeat the process.

The second possibility is that it is not possible to continue $p$, as described above, without leaving $R$, because $p(t_k)$ satisfies the GMC equality of some pair $(J,P)$.  In this case we find such a pair, then descend recursively to the computation of the GCPS allocations of $(E - p(t_k))_{(J,P)}$ and $(E - p(t_k))^{(J,P)}$.  We now describe an algorithm that determines which of these possibilities hold, finding a satisfactory $\theta$ in the first case and a suitable $(J,P)$ in the second case.

Suppose that there is $\theta \in \In^{I \times O}$ such that for there exists a $t' > t_k$ such that for all $t \in [t_k,t']$, ($*$) holds and $\barp(t_k) + \theta(t - t_k) \in Q$.  Together these conditions imply that $p(t_k) + \theta^k(t - t_k) \in R$, so the first possibility above holds, and we can use $\theta$ to define the continuation of $\barp$.  The algorithm may be thought of as a search for such a $\theta$.

% To simplify notation, henceforth we write $p$ in place of $p(t_k)$ and $\barp$ in place of $\barp(t_k)$.  
For a given $\theta$, a $t' > 0$ as above exists if and only if $\theta$ satisfies the following conditions:
\begin{enumerate} 
  \item[(a)] For each $i$ and $o$:
    \begin{enumerate}
      \item[(i)] If $o \notin \alpha_i$, then $\theta_{io} = 0$.
      \item[(ii)] If $\barp_{io}(t_k) = p_{io}(t_k)$, then $\theta_{io} \ge 0$, and if, in addition, $o = e^k_i$, then $\theta_{io} \ge 1$.
      \item[(iii)] If $\barp_{io}(t_k) = g_{io}$, then $\theta_{io} \le 0$.
    \end{enumerate}
  \item[(b)] For each $i$, $\sum_o \theta_{io} = 0$.
  \item[(c)] For each $o$, if $\sum_i \barp_{io}(t_k) = q_o$, then $\sum_i \theta_{io} \le 0$.
\end{enumerate}


Our search for a suitable $\theta$ begins by defining an initial $\theta \in \In^{I \times O}$ as follows.  For each $i$,  if $\barp_{ie^k_i}(t_k) > p_{ie^k_i}(t_k)$, then we set 
$\theta_{io} = 0$ for all $o$.  If $\barp_{ie^k_i} = p_{ie^k_i}(t_k)$, then we set
$\theta_{ie^k_i} = 1$, we set $\theta_{io_i} = -1$ for some $o_i \ne e^k_i$ such that $\barp_{io_i}(t_k) > p_{ie^k_i}(t_k)$, and we set $\theta_{io} = 0$ for all other $o$.  Evidently  $\theta$ satisfies (a) and (b).

Let $$\tP = \{\, o : \text{$\sum_i \barp_{io}(t_k) = q_{o}$ and $\sum_i \theta_{io} > 0$} \,\}.$$  If $\sum_{o \in \tP} \sum_i \theta_{io} \le 0$, then (c) holds.  Suppose that this is not the case.  We now describe a construction that may or may not be possible.  When it is possible, it passes from $\theta$ to a $\theta' \in \In^{I \times O}$ satisfying (a) and (b) such that if $\tP' = \{\, o : \text{$\sum_i \barp_{io}(t_k) = q_{o_h}$ and $\sum_i \theta'_{io} > 0$} \,\}$, then $$\sum_{o \in \tP'} \sum_i \theta'_{io} = \sum_{o \in \tP} \sum_i \theta_{io} - 1.$$
Repeating this construction will eventually produce an element of $\In^{I \times O}$ satisfying (a)--(c) unless, at some point, the construction becomes impossible.

Choose $o_0 \in \tP$, and let $P_0 = \{o_0\}$.  We define sets $J_1,P_1, J_2, P_2, \ldots$ inductively.  If $P_{h-1}$ is given, let $J_h = \bigcup_{o \in P_{h-1}} J_h(o)$ where 
$$J_h(o) = \{\, i : \text{$o \in \alpha_i$ and if $\barp_{io}(t_k) = p_{io}(t_k)$, then  $\theta_{io} > 0$ and $\theta_{io} > 1$ if $o = e^\succ_i$} \,\}.$$
If $J_h$ is given, let $P_h = \bigcup_{i \in J_h} P_h(i)$ where
$$P_h(i) = \{\, o \in \alpha_i : \text{$\theta_{io} < 0$ if $\barp_{io}(t_k) = g_{io}$} \,\}.$$

Suppose that for some $h$ there is an $o_h \in P_h \setminus P_{h-1}$ such that either $\sum_i \barp_{io_h} < q_{o_h}$ or $\sum_i \theta_{io_h} < 0$. We can find a $i_h \in J_h$ such that $o_h \in J_h(i_h)$,  then find an $o_{h-1} \in P_{h-1}$ such that $i_h \in J_h(o_{h-1})$, and so forth. Define $\theta'$ by setting $$\theta'_{i_go_{g-1}} = \theta_{i_go_{g-1}} - 1 \quad \text{and} \quad \theta'_{i_go_g} = \theta_{i_go_g} + 1$$ for $g = 1, \ldots, h$ and $\theta'_{io} = \theta_{io}$ for all other $(i,o)$.  

It is easy to see that $\theta'$ satisfies (a), because $\theta'_{io}$ differs from $\theta_{io}$ only when the difference is permitted, according to (i)--(iii).  For each $i$, $\sum_o \theta'_{io} = \sum_o \theta_{io}$, either by construction if $i = i_g$ for some $g$, or because $\theta'_{io} = \theta_{io}$ for all $o$.  Since $\theta$ satisfies (b), $\theta'$ also satisfies (b).

For $o \notin \{o_0, \ldots, o_h\}$ we have $\theta'_{io} = \theta_{io}$ for all $i$ and thus $\sum_i \theta'_{io} = \sum_i \theta_{io}$.  For each  $g = 1, \ldots, h-1$ we have $\sum_i \theta'_{io} = \sum_i \theta_{io}$ by construction.  Clearly $\sum_i \theta'_{io_0} = \sum_i \theta_{io_0} - 1$ and $\sum_i \theta'_{io_h} = \sum_i \theta_{io_h} + 1$.  Since either $\sum_i \barp_{io_h} < q_{o_h}$ or $\sum_i \theta'_{io_h} \le 0$, $\theta'$ has all desired properties.

It is impossible to construct $\theta'$ in this manner if there is no $h$ and $o_h \in P_h \setminus P_{h-1}$ such that $\sum_i \barp_{io_h} < q_{o_h}$ or $\sum_i \theta_{io_h} < 0$.  
Supposing that this is the case, let $J = \bigcup_h J_h$ and $P = \bigcup_h P_h$.  
We have $\sum_i \barp_{io}(t_k) = q_o$ for all $o \in P$.  If $o \in P$ and $i \notin J$, then $\barp_{io}(t_k) = p_{io}(t_k)$.  If $i \in J$ and $o \notin P$, then  $\barp_{io}(t_k) = g_{io}$ if $o \in \alpha_i \setminus P$, and $\barp_{io}(t_k) = g_{io} = 0$ if $o \notin \alpha_i$. Thus $\barp(t_k) - p(t_k)$ is a feasible allocation for $E - p(t_k)$ that gives all of the resources in $P$ to agents in $J$, and it gives $g_{io} - p_{io}(t_k)$ to $i \in J$ whenever $o \in O \setminus P$, so, by Lemma \ref{lem:critical}, $(J,P)$ is a critical pair for $E - p(t_k)$.  

Summarizing, the algorithm repeatedly extends $p$ and $\barp$ to intervals such as $[t_k,t_{k+1}]$ until $p(t_k)$ satisfies the GMC equality for a pair $(J,P)$, at which point it descends recursively to computation of the GCPS allocations of $(E - p(t_k))_{(J,P)}$ and $(E - p(t_k))^{(J,P)}$. If $p(t_k)$ does not satisfy such a GMC inequality, it finds a $\theta$ satisfying (a)--(c) by beginning with a $\theta$ that satisfies (a) and (b) and repeatedly adjusting it until it also satisfies (c).

The theoretical and practical complexity of the algorithm depends on the following factors:
\begin{enumerate}
  \item[(a)] How many linear segments can the piecewise linear function $(p,\barp)$ have, and how many does it typically have?
  \item[(b)] How many adjustments $\theta \to \theta'$ can be required to achieve a $\theta$ satisfying (a)--(c), and how many are typically required?
  \item[(c)] How long can the paths $o_o, i_1, o_1, \ldots, i_h, o_h$ be, and how long are they typically?
\end{enumerate}
The algorithm has been implemented for school choice problems in the software package \emph{GCPS Schools}, as described in Appendix \ref{sec:GCPSSchools}.  A computational experiment described there illuminates these issues.

It seems to be quite difficult to get a theoretical upper bound on the number of linear segments of $(p,\barp)$, and even to prove that the algorithm has polynomial time worst case complexity.  Empirically, the number of linear segments is roughly proportional to the number of students.

The construction of the initial $\theta$ implies that the number of students is an upper bound on the number of adjustments of $\theta$.  In practice the number of adjustments seems to vary between 5\% and 20\% of this number.

The number of schools is a crude  upper bound on the length of a path arising in the adjustment of $\theta$, and it is not obvious how one might develop a tighter theoretical bound.  In practice paths are quite short, with an average length between one and two.

Overall the implementation of the algorithm seems to work quite well.  Extrapolating from currently available data, it seems reasonable to hope that it will be applicable to the largest school choice problems, with running times measured in hours or days.

\section{Nonbinary Priorities} \label{sec:Nonbinary}

At least in their basic forms, DA and TTC require that each school has a priority that is strict.  One of the advantages of the GCPS mechanism is that it allows for priorities that are coarse.  In the most extreme case priorities are dichotomous: either a student $i$ is eligible to attend a school $o$, and weakly prefers it to her safe school, in which case $g_{io} = 1$, or she is ineligible or dislikes it, in which case $g_{io} = 0$.  Many schools systems use more complex priorities that assign points based on grades, test scores, minority status, the student's proximity to the school, and perhaps other factors. DA typically refines these priorities by replacing them with (perhaps randomly generated) strict refinements, and it then implicitly computes a priority cutoff for each school.  In this section we briefly consider how GCPS might be used to achieve a similar effect.

We assume that each school's priority has finitely many categories, and that these are linearly ordered.  We assume that the system as a whole has enough capacity to serve all students, so there may be some schools whose capacity is not fully utilized, and these schools do not exclude any eligible student who wishes to attend one of them.  For a school $o$ that would be overdemanded if it did not restrict eligibility, an ideal situation is that there is a cutoff category, students with lower priority are not assigned any probability of being assigned a seat in $o$, students with higher priority are not assigned any probability of receiving seats in schools that are worse for them than $o$, and the school's capacity is fully utilized.  

Suppose that for each school $o$ we choose a cutoff category.  For each student $i$ we set $g_{io} = 1$ if $i$ is eligible to attend $o$, she weakly prefers it to her safe school, and her priority at $o$ is above the cutoff priority.  We set $g_{io} = 0$ for students $i$ who are ineligible or unwilling to attend $o$, or who have a priority at $o$ below the cutoff category. For all the $i$ in the cutoff category who are eligible and willing to attend $o$ we set $g_{io}$ to a certain number in $[0,1]$.
After computing the GCPS allocation we can adjust the cutoff category of each school $o$, and the numbers $g_{io}$ for the students in these categories, and then compute the GCPS allocation for the adjust parameters, hoping for a better balance of supply and demand.  

This scenario raises several questions that are familiar in other contexts.  Is it possible to find cutoff categories, and settings of $g_{io}$ for each school's cutoff category, that balance supply and demand simultaneously for all schools with fully utilized capacity?  Is there a unique such vector of settings?  (Since any such settings should be approximable by the DA outcome of a similar problem, results of \cite{AzLe16} suggest that this is likely.) If so, do natural ``naive'' iterative adjustment processes necessarily converge to it?  If the theoretical answers are negative or indeterminate, in practice 
are such naive adjustment processes reliable nevertheless?  At present we do not have clear-cut answers.

Section \ref{app:Eating} of the Online Appendices describes the indirect effects of changing a single $g_{io}$ in a ``generic'' setting in which the map from $g_{io}$ to the allocation is differentiable.  (The effect of a large change is the integral of these differential effects over finitely many intervals of generic values of $g_{io}$.)  This analysis, and in particular Lemma \ref{lemma:IndirectEffect}, show that the total indirect effects are bounded in a certain sense, and as the allocation process continues they spread out across the different assignment probabilities.  

Consider a naive adjustment process in which, at each iteration, for each school $o$ the cutoff category and the $g_{io}$ of the cutoff category are adjusted in a way that would restore balance at that school.  Such a process will converge to simultaneous balance at all schools unless the aggregate indirect effects are large enough to take the process further away from balance, even though the direct effects are in the right direction.  Among other things, in order for the aggregate indirect effects to be large, the effects on any one school from the other schools' adjustments must mostly have the same sign.  Thus it seems likely that naive adjustment processes will be reliable in practice.  

Policy issues related to priorities are extremely complex.  On the one hand it seems desirable to provide the best resources to those who can extract the greatest benefit, so it makes sense to give high priority to students with good grades or test scores, but doing so could exacerbate inherited inequality.  Giving students points for proximity to a school may facilitate allowing each student to attend a nearby school, but preventing a student from attending a distant school if she is willing to incur the travel costs has a paternalistic aspect. The literature on peer effects in education \citep{EpRo11,Sac11} is extensive, finding significant effects with causal pathways that are not yet well understood.  In particular, \cite{BuSa13} find that low achieving students derive substantial benefits from having high achieving peers, and \cite{ViNe07} find that classroom heterogeneity can lead to higher test scores. 
One could easily list additional issues.  

Balancing various concerns in practice requires information concerning what would actually happen under various policies.  Because our mechanism allows priorities to be coarse rather than strict, we widen the range of alternatives that can be considered, and we more clearly distinguish between priorities that reflect actual societal values and those that merely fulfill the requirements of a mechanism.
Our computational methods potentially allow easy computation of counterfactual outcomes resulting from applying various alternatives to historical data.


\section{Efficiency} \label{sec:Efficiency}

In this section we work with a CEE $E = (I,O,r,q,g)$ that satisfies the GMC and a fixed profile of preferences $\succ$.  Our objective is to show that the GCPS mechanism applied to $E$ and $\succ$ yields an allocation that is efficient in strong senses. However, we should first of all mention that mechanisms that are \emph{ordinal} (that is, based on the agents' reports of ordinal preferences) and nondictatorial often allow allocations that are inefficient relative to cardinal utility functions consistent with the ordinal preferences \citep{FeNi08,Mir09,AbChYa11,Tro12,AbChYa15}.

For $i \in I$, an \emph{allocation for $i$} is a vector $m_i \in \Re^O_+$ such that $m_i \le g_i$ and 
$\sum_o m_{io} = r_i$.  The \emph{stochastic dominance relation} ${sd}(\succ_i)$ on allocations for $i$ derived from $\succ_i$ is defined by $m_i' \, {sd}(\succ_i) \, m_i$ if $\sum_{p \succeq_i o} m_{ip}' \ge \sum_{p \succeq_i o} m_{ip}$ for all $o \in O$.  Usually in applications of this concept $m_i$ is a probability distribution on $O$, but the concept makes perfect sense in our more general context, and standard arguments generalize straightforwardly to show that $m_i' \, {sd}(\succ_i) \, m_i$ if and only if $\sum_o m_{io}' u_i(o) \ge \sum_o m_{io} u_i(o)$ for any cardinal utility function $u_i : O \to \Re$ such that for all $o,o' \in O$, $u_i(o) \ge u_i(o')$ if and only if $o \succeq_i o'$.

Two other well-studied extensions of a given preference to preferences over lotteries relate to lexicographic preferences (\citealp{cho16geb}; \citealp{sv15wp}; \citealp{cd16}; \citealp{ss14jme}; \citealp{cho18scw}). The first extension, which is called the \emph{downward lexicographic} extension ($dl$-extension) compares two $i$-allocations as follows. One of the $i$-allocations is preferred if it assigns a higher amount of the most preferred object than the other. If the two $i$-allocations assign the same amount of the most preferred object, then the one that is preferred is the one that assigns the greater amount of the second most preferred object. If the two amounts are equal again, then the $i$-allocation that assigns a greater amount of the third most preferred object is preferred, and so on.
The second extension, which is called the \emph{upward lexicographic} extension ($ul$-extension) is a dual of the $dl$-extension. It lexicographically minimizes amounts of less preferred objects, starting from the least preferred object\footnote{ 
Formally, the \emph{downward lexicographic relation} ${dl}(\succ_i)$ on allocations for $i$ derived from $\succ_i$ is defined by specifying that $m_i' \, {dl}(\succ_i) \, m_i$ if  either $m_i' = m_i$ or there is an $o \in O$ such that $\sum_{p \succeq_i o'} m_{ip}' = \sum_{p \succeq_i o'} m_{ip}$ for all $o' \in O$ such that $o' \succ_i o$ and $\sum_{p \succeq_i o} m_{ip}' > \sum_{p \succeq_i o} m_{ip}$.   The \emph{upward lexicographic relation} ${ul}(\succ_i)$ is defined by specifying that $m_i' \, {ul}(\succ_i) \, m_i$ if  either $m_i' = m_i$ or there is an $o \in O$ such that $\sum_{o' \succeq_i p} m_{ip}' = \sum_{o' \succeq_i p} m_{ip}$ for all $o' \in O$ such that $o \succ_i o'$ and $\sum_{o \succeq_i p} m_{ip}' < \sum_{o \succeq_i p} m_{ip}$.
}.  The $dl$- and $ul$-extensions yield preferences that represent the limits of standard vNM utility functions with extreme risk loving and risk aversion, respectively.

For $e \in \{sd,dl,ul\}$, a feasible allocation $m'$ \emph{$e$-dominates} another feasible allocation $m$ if $m_i' \, e(\succ_i) \, m_i$ for all $i$ and there is some $i$ such that $m_i' \ne m_i$.  A feasible allocation $m$ is \emph{$e$-efficient} if there is no feasible allocation that $e$-dominates it.  This section's main result is:

\begin{thm}\label{th:axiom_topdown}
For $e \in \{sd,dl,ul\}$, the GCPS allocation for $E$ and $\succ$ is \emph{$e$-efficient}.  
\end{thm}

The following result is essentially due to \cite{cd16}.  (Lemma 3 of BM is a precursor.)  We provide no proof because it is easy to see that their proof works, essentially without any modification, in our more general setting.

\begin{lem}\label{lem:cyclic} 
  $sd$-efficiency, $dl$-efficiency, and $ul$-efficiency are equivalent. 
\end{lem}

The following is a special case of the proof of Proposition 3 of \cite{balbuzanov22jet}.  

\begin{proof}[Proof of Theorem \ref{th:axiom_topdown}] 
Let $m = p(1)$ be the GCPS allocation.  By the last result it suffices to show that $m$ is $sd$-efficient.  Aiming at a contradiction, suppose that $m' \in Q$, $m' \ne m$, and $m' \, sd(\succ_i) \, m_i$ for all $i$.  Let $t^*$ be the first time such that there is a $J \subset I$ and $P \subset O$ such that 
$p(t^*)$ satisfies the GMC equality for $J$ and $P$ and there is $i_0 \in J^c$ and $o_0 \in P$ such that $m'_{i_0o_0} \ne m_{i_0o_0}$.  Since $m'$ satisfies the GMC inequality for $J$ and $P$ we have
$$\sum_{i \in J^c}\sum_{o \in P} m'_{io} \le \sum_{i \in J^c}\sum_{o \in P} m_{io}.$$
Without loss of generality we may assume that $m'_{i_0o_0} < m_{i_0o_0}$ because otherwise we could replace $i_0$ and $o_0$ with other elements of $J^c$ and $P$.  Since $m' \, sd(\succ_i) \, m_i$ and $m' \ne m$ there is an object $o'$ such that $o' \succ_{i_o} o_0$ and $m'_{i_0o'} > m_{i_0o'}$.  Since $p_{i_0o_0}(t^*) = m_{i_0o_0} > 0$, there must have been a time prior to $t^*$ at which $o'$ became unavailable to $i_0$, but this contradicts the definition of $t^*$.  The proof is complete.
\end{proof}

\section{Fairness for School Choice} \label{sec:Fairness}

We now briefly consider fairness properties of the GCPS mechanism applied to a CEE $E = (I,O,r,q,g)$ that satisfies the GMC and a profile of preferences $\succ$.  It is obvious from the definition that the mechanism satisfies  anonymity (the outcomes do not depend on the ordering of the agents, or their ``names") and equal treatment of equals (the GCPS gives the same allocations to $i$ and $j$ if $r_i = r_j$, $g_i = g_j$, and $\succ_i \; = \; \succ_j$).

The other fairness property considered by BM is envy-freeness.  They show that the PS mechanism is envy-free in the strong sense that if $m_i$ and $m_j$ are the allocations of the PS mechanism for $\succ$, then $m_i \, sd(\succ_i) \, m_j$.  It is not reasonable to expect this if the two agents have different opportunities, and in recognition of this \cite{as03aer} introduced a notion of no justified envy.  This concept takes on different meanings depending on the setting (for a recent discussion see \cite{RoRoSh20}).  We follow \cite{yilmaz10geb} in the context of school choice.  If $E$ is a school choice CEE with $g \in \{0,1\}^{I \times O}$, we say that $m \in Q$ \emph{has no justified envy} if, for all $i,j \in I$, if $\alpha_i \subset \alpha_j$ and $o_i \succ_i o_j$ for all $o_i \in \alpha_i$ and $o_j \in \alpha_j \setminus \alpha_i$, then $m_i \, sd(\succ_i) \, m_j$.  Intuitively, $i$'s envy of $j$ is not justified if $i$ is not eligible to attend a desirable element of $\alpha_j$, or if $j$ can demand a seat in some element of $\alpha_i$ because less desirable elements of $\alpha_i$ are not in $\alpha_j$.

\begin{prop} \label{prop:jenvy}
  If $E$ is a school choice CEE with $g \in \{0,1\}^{I \times O}$, then $GCPS(\succ)$ has no justified envy.
\end{prop}

\begin{proof} % [Proof of Proposition \ref{prop:jenvy}]
  Suppose that $i,j \in I$, $\alpha_i \subset \alpha_j$, and $o_i \succ_i o_j$ for all $o_i \in \alpha_i$ and $o_j \in \alpha_j \setminus \alpha_i$.
  At each time during the allocation process, if there is a critical set of schools $P$ such that the only remaining schools that $i$ can consume are contained in $P$, but $j$ has some remaining school outside of $P$, then $j$ is consuming some element of $\alpha_j \setminus \alpha_i$.  On the other hand, if at that time, for every critical $P$, either all the remaining schools for $j$ are contained in $P$ or there is some remaining school for $i$ that is outside $P$, then the set of remaining schools for $j$ that are contained in $\alpha_i$ is the set of remaining schools for $i$.  In either case $i$ is consuming a school that she weakly prefers to the school that $j$ is consuming.  Since this is true at all times during the allocation process, $GCPS_i(E,\succ) \, sd(\succ_i) \, GCPS_j(E,\succ)$.
\end{proof}


\section{Strategy Proofness for School Choice} \label{sec:StrategyProof}

With the exception of some of the material towards the end of Section \ref{sec:Procedure}, and the result in the last section, up to this point our results have concerned the general GCPS mechanism.  In this section we examine the extent to which the GCPS mechanism, applied specifically to school choice with safe schools, is resistant to manipulation.  We fix an integral school choice CEE $E$ that satisfies the GMC and a profile $\succ$ of preferences over $O$.  For each student $j$ the safe school is the $\succ_j$-worst element of $\alpha_j$.  We also fix a particular student $i \in I$ whose possible deviations from truthful reporting we will study.

BM (p.~310) show that there is no probabilistic allocation mechanism for object allocation that is ex post efficient, strategy proof, and envy free.  Theorem 4 of \citet{yilmaz10geb} states that for house allocation problems with existing tenants, there is no mechanism that is individually rational, strategy proof, and has no justified envy.
Their settings are special cases of ours, so we cannot hope that the GCPS mechanism is fully strategy proof: there necessarily exist situations in which, for some vNM utility function consistent with the true ordinal preferences, expected utility can be increased by reporting a false preference. Nevertheless we will argue that for school choice the failures of strategy proofness of the GCPS mechanism are minor when each school has many students, and do not significantly impair its usefulness.

There are three different ways a student might try to manipulate: a) reporting that some of the schools that are actually worse than the safe school are better than it;  b) reporting that some of the schools that are better than the safe school are worse; c) reordering of the schools that are better than the safe school.
A manipulation attempt of type a) will be called an \emph{augmentation}; following \cite{RoRo99ecma}, a manipulation attempt of type b) will be called a \emph{truncation}; a  manipulation attempt of type c) will be called a \emph{reordering}.  We discuss these in turn. 

Manipulation by augmentation is unambiguously unsuccessful:

\begin{thm} \label{th:Augmentation}
  Let $\alpha_i' = \alpha_i \cup \{o^*\}$, where $o^*$ is an element of $O \setminus \alpha_i$, and let $\succ_i'$ be a preference over $O$ that has $\alpha_i'$ as the set of schools weakly preferred to the safe school, and that  agrees with $\succ_i$ on $\alpha_i$.  Let $\succ' \; = (\succ_i',\succ_{-i})$.  Then 
$$GCPS_i(\succ) \, sd(\succ_i) \, GCPS_i(\succ').$$
\end{thm}

\noindent As a matter of logic, this result does not rule out the possibility that other forms of manipulation might become successful, or more successful, if supplemented with an augmentation, but this possibility seems to have slight practical importance.

\cite{yilmaz10geb} (Example 5) presents the following example of an unambiguously successful truncation manipulation for a house allocation problem with existing tenants\footnote{Theorem 3 of \citet{cho18scw} asserts that the PS mechanism is $dl$-strategy proof, which means that manipulation never results in a $dl$-better allocation.  This example shows that Cho's result does not extend to house allocation problems with existing tenants.}.  There are three homeowners and three houses, with  $1$ endowed with $a$, $2$ endowed with $b$, and $3$ endowed with $c$, and preferences $b \succ_1 c \succ_1 a$, $a \succ_2 b$, and $b \succ_3 a \succ_3 c$.  In the GCPS process $P = \{a\}$ becomes critical at time $\tfrac12$, with $J_P = \{2\}$, so the GCPS allocation gives $\tfrac12 b + \tfrac12 c$ to $1$,  $a$ to $2$, and  $\tfrac12 b + \tfrac12 c$ to $3$.  If $1$ reports the preference  $b \succ_1' a$  (i.e., $b \succ_1' a \succ_1' c$) then $P = \{a,b\}$ is critical at time $0$, and the allocation gives $b$ to $1$, $a$ to $2$, and $c$ to $3$.  As Yilmaz points out, this manipulation continues to be possible in any of the problems obtained by replacing each homeowner-current house pair with any number of copies.  For example, if there are five copies of agent $1$, three copies of agent $2$, and two copies of agent $3$, and one of the copies of agent $1$ reports  $b \succ_1' a$, then $P = \{a,b\}$ is critical at time $\tfrac23$, and the allocation gives $b$ to the deviator, $a$ to the agents of type $2$, and $\tfrac23 b + \tfrac13 c$ to nondeviant agents of type $1$ and agents of type $3$.

\cite{ab19res} introduce a notion of asymptotic strategy proofness in large economies: a mechanism is \emph{strategy proof in the large} if, along any sequence of increasingly large economies in which each agent's belief concerning the types of the other agents is given by i.i.d.~draws from a fixed distribution over the set of possible types, the maximal gains from manipulation vanish in the limit.  They show that if a mechanism is envy-free, then it is strategy proof in the large.  The intuition is that the gain from reporting an incorrect type is the sum of the gain from getting that type's allocation, which is nonpositive by envy-freeness, plus the gain from changing the overall allocation, which diminishes as the agent's importance in the economy shrinks. 
Because the example above is robust with respect to the numbers of the three types, it shows that the GCPS mechanism is not strategy proof in the large, but neverthless the intuition still has at least an informal applicability, which is reflected in Theorem \ref{th:StrategyProof} below.

In general, in order for a truncation manipulation to succeed, the manipulation must induce a critical set of schools $P$ at some time during the allocation process that includes the manipulating student's safe school and a school or schools that that student wishes to consume, and that does not include the school or schools that the student falsely reports to be worse than her safe school, so that students who can consume those schools are outside of $J_P$ and thus prevented from consuming the schools the manipulator desires.  The manipulator's safe school must be in high enough demand that the manipulator does not end up simply consuming more of that school, and the schools that the manipulator desires must also be in high demand, since otherwise the manipulator can get what she wants simply by ranking her favorite of those schools highly, in which case the manipulation does not gain anything.  

This set of requirements is rather lengthy and specific, but this type of manipulation does not seem to have a knife-edge quality, and one can easily imagine successful truncation manipulations in quite complex settings.  In addition, it does not seem that the student needs highly specific information in order to know that the manipulation is likely to succeed or at least do no harm.  In particular, if the student believes that her safe school is so popular that there is no chance that a truncation manipulation attempt will result in her consuming more of her safe school, then such an attempt can (roughly) weakly dominate truthful revelation.

To what extent does this type of manipulation impair the usefulness of the GCPS mechanism for school choice?  As we explain below, reordering manipulations are quite difficult, so a student whose safe school is not highly popular is (with minor exceptions) incentivized to reveal truthfully, independent of the extent to which others are attempting truncation manipulations.  Thus the Nash equilibria of the mechanism are not drastically different from the naive understanding of it obtained by assuming truthful revelation.  The outcome of the mechanism when there are successful truncation manipulations is $sd$-efficient for the true preferences.  Indeed, truncation manipulations seem to be mainly an annoyance from the point of view of fairness: a student with a highly desired safe school may have an opportunity (possibly at some risk) to amplify her good fortune at others' expense.

In the remainder of this section we consider reordering manipulations.  Proposition 1 of BM asserts (among other things) that the PS mechanism is weakly strategy proof: if reporting a false preference gives an allocation that is weakly  $sd$-preferred to the allocation resulting from truthful revelation, then the two allocations are the same. \cite{kojima09mss} shows, by means of the following example, that weak strategy proofness does not extend to the allocation of $r \ge 2$ objects per agent.  Let there by two agents $1$ and $2$ and four objects $a$, $b$, $c$, and $d$, so $r = 2$.  Let the true preferences be $a \succ_1 b \succ_1 c \succ_1 d$ and $b \succ_2 c \succ_2 a \succ_2 d$.  If the agents report these preferences, then the PS mechanism gives $(1,0,\tfrac12,\tfrac12)$ to agent 1 and $(0,1,\tfrac12,\tfrac12)$ to agent 2.  On the other hand, if agent 1 reports $\succ_1'$, where $b \succ_1' a \succ_1' c \succ_1' d$, and agent 2 reports $\succ_2$, then  the PS mechanism gives $(1,\tfrac12,0,\tfrac12)$ to agent 1 and $(0,\tfrac12,1,\tfrac12)$ to agent 2.  Thus, when agent 1 reports the truth she receives the probability distribution  over pairs $\tfrac12 (a,c) + \tfrac12 (a,d)$, while misrepresenting yields $\tfrac12 (a,b) + \tfrac12 (a,d)$, which stochastically dominates (in an obvious sense) the allocation resulting from truthful reporting. 

A key idea of BM's proof of their Proposition 1 is that once other agents begin eating an object, they continue until that object is exhausted.  Consequently, in order to obtain the same amount of her favorite object as in the allocation resulting from truthful revelation, an agent must report that it is her favorite.  In order to get the maximal amount of her best among the objects that are still available after her favorite has been fully allocated, she cannot report a preference that ranks it below some other object that is still available, and so forth inductively.  In the example above, consumption of a school by other students can cease before the school is fully allocated, which allows student $1$ to advantageously defer its consumption.

Since students do not consume multiple seats, one might hope that the GCPS mechanism is weakly strategy proof for school choice, but the following example shows that this is not the case.  There are five schools with $q_a = q_b = q_c = q_d = 1$ and $q_e \ge 4$.  There are eight students, with preferences $a \succ_1 b \succ_1 c \succ_1 d$, $a \succ_2 e$,  $a \succ_3 e$,  $b \succ_4 e$,   $b \succ_5 e$,   $c \succ_6 e$,  $d \succ_7 e$, and  $c \succ_8 d$.  (For each student the lowest ranked school is the safe school.)  Up until time $\tfrac13$ each student consumes her favorite school.  At time $\tfrac13$ school $a$ is exhausted, and the set $\{b,c,d\}$ also becomes critical, with remaining capacity $\tfrac13 b + \tfrac13 c + \tfrac23 d$ that is just sufficient to serve the needs of students $1$ and $8$, who cannot attend $e$.  If student $1$ reports truthfully she receives $\tfrac13 a + \tfrac13 b + \tfrac13 d$ because student $8$ consumes what remains of school $c$ between time $\tfrac13$ and time $\tfrac23$.  If instead she reports that her preference is $a \succ_1' c \succ_1' b \succ_1' d$, then she and student $8$ divide what is left of school $c$ between time $\tfrac13$ and time $\tfrac12$, so she receives $\tfrac13 a + \tfrac13 b + \tfrac16 c + \tfrac16 d$.  In this example consumption of school $b$ by other students ceases before the school is fully allocated, which allows student $1$ to defer its consumption.

We now consider a different example illustrating how strategy proofness can fail.  Suppose that $a$ and $b$ are the agent's first and second favorite object, with $q_a = q_b = 1$, and there are $A-1$ other people who have $a$ as their favorite and $B - 1$ other people who have $b$ as their favorite, where  $1 < A < B$.  Further, assume that no one outside the set of agents  who have $a$ as their favorite will ever consume any $a$ and no one outside the set of agents  who have $b$ as their favorite will ever consume any $b$.  If the agent reports the truth she will receive $\tfrac{1}{A}$ units of $a$ and none of $b$.  If she reports that $b$ is her favorite and $a$ is her second favorite, then she will consume $b$ between time $0$ and time $\tfrac{1}{B}$ while $\tfrac{A-1}{B}$ units of $A$ are being consumed by others, and then she will
receive $\tfrac{1}{A}(1 - \tfrac{A-1}{B})$ units of $a$, so her total consumption of $a$ and $b$ will be $\tfrac{1}{A}(1 + \tfrac{1}{B})$.  This can be an improvement if the utility difference between $a$ and the agent's third favorite object is more than $A$ times the utility difference between $a$ and $b$.

This example suggests that, in general, the benefit of manipulatively consuming an inferior object (to change the later availability schedule of other objects) will be small in comparison with the amount of manipulation if there are many agents competing for the objects. In the special case of our general model in which $E$ is integral and $g_{io} = r_i$ (in effect) for all $i$ and $o$,  \cite{km10jet} (henceforth KM) establish an exact result along these lines: for a given utility function $u_i$ consistent with a preference $\succ_i$, if $\min_o q_o/r_i$ is sufficiently large, then agent $i$ will not be able to increase the expected utility from the probabilistic serial mechanism by reporting a different preference $\succ_i'$.

Following KM we present a result that shows that for a given vNM utility function, if, for each student, the number of students with the same preferences and opportunities is large, relative to the ratio of the largest utility difference to the smallest utility difference, then truthful reporting is a weakly dominant strategy.

\begin{thm} \label{th:StrategyProof}
Let $E$ be an integral school choice CEE that satisfies the GMC, and let $\succ \; = (\succ_i)_{i \in I}$ be a preference profile.  Let $\succ_i'$ be an alternative preferences for some $i \in I$,  and let $\succ' = (\succ_i',\succ_{I \setminus \{i\}})$.  Let $u_i \colon A \to \Re$ be a cardinal utility function consistent with $\succ_i$, and let $d_i = \min_{o \, \succ_i \, p} u_i(o) - u_i(p)$ and $D_i = \max_{o \, \succ_i \, p} u_i(o) - u_i(p).$  Let
$N_0 =  |\{\, j \in I : \text{$\alpha_j = \alpha_i$ and $\succ_j \; = \; \succ_i$} \,\}|$.
If $\big(1 + \frac{d_i}{D_i}\big)^{1/|O|} \ge  \frac{N_0 + 1}{N_0},$ then $u_i(GCPS(\succ)) \ge u_i(GCPS(\succ'))$.
\end{thm}

The proof of this result has two phases.  In the first it is shown that the effect of the manipulation by a student $i$ on the overall allocation is bounded by the amount that $i$'s own consumption differs between the eating schedule induced by the true preference and eating schedule induced by the reported preference.  

The second phase bounds the benefits of the periods of time during which the student can eat from a school that would not be available if the student reported her true preference.  The additional amount of the school that the student consumes during such a period is the amount that is available at the beginning of the period divided by the number of students eating from this school.  In the KM setting the set of agents eating an object type is weakly increasing while the object type is available, so having a large number of objects of each type implies that if an object type is fully consumed, then the final rate of consumption is high.  In our setting the number of agents eating from a school can decrease when sets of schools become critical, so we need an additional assumption to insure that the number of agents competing for each school is high.  The simplest way to insure this is to assume that there are many students with the same opportunities and preferences as the manipulator.

We regard Theorem \ref{th:StrategyProof}, and especially its proof, as illustrative of the difficulties of reordering manipulation, rather than as a complete explanation of them.  In particular, a key point is that in order to manipulate successfully, the manipulator must believe that during the time when a school is available due to the manipulation, there will be scant competition.  While this is certainly not the case under our assumption, there are many other reasons that competition for the school might be expected.   It will be evident that noticing an opportunity to manipulate typically requires much more information than a student is likely to possess. 

We have seen that manipulation by augmentation is impossible.  Manipulation by truncation is sometimes possible, and less frequently entails little risk, but it does little to change the incentives of other students, so (in contrast with the Boston mechanism) it does not lead to Nash equilibria that are drastically different from truthful revelation.  Manipulation by reordering has large costs and low rewards when there are many agents for each object, which is typically the case for school choice.  On the whole, failures of strategy proofness are minor and do little to impair the practical applicability of the GCPS mechanism to school choice.

\section{Concluding Remarks} \label{sec:Conclusion}

We have provided a school choice mechanism that is a specialization of the GCPS mechanism of \cite{balbuzanov22jet}, which is in turn a generalization of the PS mechanism of BM.  This mechanism guarantees each student a seat in a school that is at least as desirable as any of the schools she is legally entitled to attend.  When there are many students for each school, it is effectively strategy proof.  It is $sd$-efficient, which (as BM stress) is a stronger condition than ex post efficiency.  In contrast, bilateral matching mechanisms based on randomly generated priorities for the schools are (at least in their most basic forms) not even ex post efficient.  It is implementable: the assignment probabilities it generates can be obtained from a randomization over pure assignments.  It satisfies anonymity, equal treatment of equals, and a natural generalization of the envy-freeness condition satisfied by the PS mechanism.
Using a novel generalization of Hall's marriage theorem, we have described a computational implementation of this mechanism that seems to be tractable even for quite large school choice problems. 

A possibility we intend to explore in subsequent research is that instead of consuming probability of desirable objects, the agents may discard probability of undesirable objects.  In the case of $n$ agents and $n$ objects, each agent is endowed with one unit of each object, and at each time during the interval $[0,n-1]$ she discards probability of the least desirable object that she has not fully discarded for which discarding is still allowed.  Discarding of an object is disallowed when the agents' total remaining endowment of it is one, but it may also be disallowed for some agents in the event that the process reaches a facet of $R$.  The characterization of the PS mechanism given by \cite{bh12} implies that the discarding mechanism is certainly different from the probabilistic serial mechanism, but otherwise its properties await investigation.  It seems appropriate for problems, perhaps such as chore assignment, in which the agents' main concern is to avoid the objects that are most noxious for them.

A possibility stressed by BM, \cite{cho18scw}, and \cite{balbuzanov22jet} (perhaps among others) is that the mechanism can be varied by making the eating speeds depend on various things.  This seems unmotivated in school choice, but in other domains it may be quite interesting.    In particular, in chore assignment some agents may be unqualified to receive certain objects, and one may recognize this by taking away their endowments of such objects at the outset, but this seems unfair insofar it amounts to giving them a head start.  Giving such agents slower discarding speeds is one way this issue could be addressed. 

Although we have emphasized the school choice application, we expect that the underlying idea of our procedure, the application of the GCPS mechanism to a CEE, is potentially of interest in many other domains, with many variations.  

\bibliographystyle{agsm}
\bibliography{pa_ref}

\pagebreak
\begin{center}
\Large \textbf{For Online Publication}
\end{center}


\begin{appendix}

% \section{Proofs of Results in Sections \ref{sec:Critical}--\ref{sec:Fairness}}


% \begin{proof} [Proof of Lemma \ref{lem:tight}]
%  An immediate consequence of the definition of $J_P$ is that $J_P \subset J_{P'}$.  The GMC inequality 
% for $(J_{P'} \setminus J_P,P' \setminus P)$ holds by assumption, so criticality of this pair amounts to the opposite inequality:
% $$\sum_{i \in J_{P'} \setminus J_P} r_i \ge \sum_{i \in J_{P'} \setminus J_P} \sum_{o \in (P' \setminus P)^c} g_{io} + 
% \sum_{o \in P' \setminus P} q_o$$
% If we subtract the GMC equation for $(J',P')$ from this, then add the GMC equation for $(J,P)$, we find that it boils 
% down to $$\sum_{i \in J_{P'}} \sum_{o \in {P'}^c} g_{io} - \sum_{i \in J_P} \sum_{o \in P^c} g_{io} 
% \ge \sum_{i \in J_{P'} \setminus J_P} \sum_{o \in (P' \setminus P)^c} g_{io}.$$  
% Recognizing that $J_{P'} = J_P \cup (J_{P'} \setminus J_P)$, $P^c =( P' \setminus P) \cup {P'}^c$, and 
% $(P' \setminus P)^c = {P'}^c \cup P$, we can further reduce this inequality to 
% $-\sum_{i \in J_P} \sum_{o \in P' \setminus P} g_{io} \ge \sum_{i \in J_{P'} \setminus J_P} \sum_{o \in P} g_{io}$.  
% The definition of $J_P$ gives $g_{io} = 0$ for $i \in J$ and $o \in P^c$.
% For $i \in J_{P'} \setminus J_P$ tightness gives $g_{io} = 0$ for $o \in P$.
  
% We have now shown that $(J_{P'} \setminus J_P, P' \setminus P)$ is critical, so $J_{P' \setminus P} \subset J_{P'} \setminus J_P$.  
% If the containment is strict there is some $i \in J_{P'} \setminus J_P$ who has $g_{io} = 1$ for some $o \in (P' \setminus P)^c$, 
% and the definition of $J_{P'}$ implies that $o \in P'$, so $o \in P$, but this is impossible because $E$ is tight.
% \end{proof}

\section{Implementability} \label{sec:Implementability}

In this Appendix we consider the problem of passing from a matrix of assignment probabilities to a random deterministic assignment whose distribution realizes the given probabilities.
Let $E = (I,O,r,q,g)$ be an integral school choice CEE that satisfies the GMC, let $Q$ be its set of feasible allocations, and let $m$ be an element of $Q$.    \cite{bckm13aer} say that such an $m$ is \emph{implementable} if the assignment probabilities are those resulting from some probability distribution over deterministic assignments\footnote{Recently \cite{AkNi20} expanded the scope of this concept by studying a notion of approximate implementation that is appropriate when some constraints need not be satisfied exactly.}.  Recalling that the vertices of $Q$ are its extreme points, we see that in order for every element of $Q$ to be implementable, each of its vertices must be a deterministic assignment, which is to say that its entries are elements of $\{0,1\}$.  Conversely, since $Q$ is the set of convex combinations of its vertices, if each vertex is a deterministic assignment, then every element of $Q$ is implementable.

As we explain in detail below, Theorem 1 of  Budish et el.~has the following result as a special case.

\begin{thm} \label{th:Implementability}
  Each vertex of $Q$ is integral.
\end{thm}

The Birkhoff-von Neumann theorem asserts that if $|I| = |O|$, then the set of bistochastic matrices with entries indexed by $I \times O$ is the convex hull of the set of bistochastic matrices with entries in $\{0,1\}$.   Evidently the Birkhoff-von Neumann theorem is a special case of Theorem \ref{th:Implementability}. 

We quickly review the related concepts and results of \cite{bckm13aer}.   A \emph{constraint set} is a nonempty subset of $I \times O$, and a \emph{constraint structure} $\cH$ is a set of constraint sets.  A vector of quotas $\bq = (q_S,q^S)_{S \in \cH}$ is \emph{integral} if $q_S,q^S \in \In$ for all $S$.
An allocation $m$ is feasible under $\bq$ if $q_S \le \sum_{_{io} \in S} m_{io} \le q^S$ for all $S \in \cH$.  Let $\cM_\bq$ be the set of feasible allocations for $\bq$. If $\cH$ contains all singletons, then $\cM_\bq$ is bounded, hence a polytope. The constraint structure $\cH$ is \emph{universally implementable} if, whenever $\bq$ is integral, each vertex of $\cM_\bq$ is integral.  A constraint structure is a \emph{hierarchy} if, for all $S, S' \in \cH$, we have $S \subset S'$ or $S' \subset S$ or $S \cap S' = \emptyset$, and $\cH$ is a \emph{bihierarchy} if there are hierarchies $\cH_1$ and $\cH_2$ such that $\cH_1 \cup \cH_2 = \cH$ and $\cH_1 \cap \cH_2 = \emptyset$.  Theorem 1 of \cite{bckm13aer} (which is also a generalization of the Birkhoff-von Neumann theorem) asserts that if $\cH$ is a bihierarchy, then it is universally implementable.  (Their Theorem 2 is a partial converse, giving conditions under which if $\cH$ is universally implementable, then it is a bihierarchy.)

Let $\cH = \cH^1 \cup \cH^2 \cup \cH^3$ where 
$\cH^1 = \{\, \{i\} \times O : i \in I \,\},$ $\cH^2 = \{\, \{(i,o)\} : (i,o) \in I \times O \,\},$ and $\cH^3 = \{\, I \times \{o\} : o \in O \,\}.$  We can show that $\cH$ is a bihierarchy either by setting $\cH_1 = \cH^1 \cup \cH^2$ and $\cH_2 = \cH^3$ or by setting $\cH_1 = \cH^1$ and $\cH_2 = \cH^2 \cup \cH^3$, so our Theorem \ref{th:Implementability} follows from their Theorem 1.   

The practical implementation of a random allocation depends not only on the existence of a representation of it as a convex combination of pure allocations, but also on a practical algorithm for generating a random pure allocation with a probability distribution that averages to the given allocation.  To this end we describe the argument in  Appendix B of the Online Appendices of Budish et al., which they attribute to Tomomi Matsui and Akihisa Tamura, as it applies to our setting.

We work with the network $(N_E,A_E)$ of Section \ref{sec:GenHall}.  If $m \in Q$, the \emph{nonintegrality set} of $m$ is $C(m) \cup D(m) \subset A$ where
$C(m) = \{\, (i,o) \in I \times O : m_{io} \notin \In \,\}$ and $D(m) = \{\, (o,t) \in O \times \{t\} : \sum_i m_{io} \notin \In \,\}.$  The next result implies that points of $Q$ that are not integral are not extreme points of $Q$, hence not vertices, so Theorem \ref{th:Implementability} follows.

Recall that the \emph{floor} of a real number $x$ is the largest integer that is not greater than $x$, and the \emph{ceiling} of $x$ is the smallest integer that is not less than $x$.  
When $x$ is an integer, it is both the floor and ceiling of itself. 

\begin{prop} \label{th:ConvexComb}
  If $E$ is integral, $m \in Q$, and the nonintegrality set of $m$ is nonempty, then there are $m^0, m^1 \in Q \setminus \{m\}$ such that $m$ is a convex combination of $m^0$ and $m^1$, and for both $h = 0,1$:
  \begin{enumerate}
     \item[(a)] For each $i$ and $o$, $m^h_{io}$ is between the floor and the ceiling of $m_{io}$.
     \item[(b)] For each $o$, $\sum_i m^h_{io}$ is between the floor and the ceiling of $\sum_i m_{io}$.
     \item[(c)] The nonintegrality set of $m^h$ is a proper subset of the nonintegrality set of $m$.
  \end{enumerate}
\end{prop}

\begin{proof} % [Proof of Proposition \ref{th:ConvexComb}]
  For each $i$, if there is an $o \in O$ such that $(i,o) \in C(m)$, then (since $\sum_o m_{io} = r_i \in \In$) there are at least two such $o$.  For each $o$, if there is exactly one $i$ such that $(i,o) \in C(m)$, then $\sum_i m_{io} \notin \In$ and consequently $(o,t) \in D(m)$.  Since $\sum_o \sum_i m_{io} = \sum_i r_i \in \In$, there cannot be exactly one $o$ such that $(o,t) \in D(m)$.  
  
  An \emph{allowed cycle} is a sequence $n_1, \ldots, n_h$ of $h > 2$ distinct nodes in $I \cup O \cup \{t\}$ such that for all $g = 1, \ldots, h$ either $(n_g,n_{g+1}) \in C(m) \cup D(m)$ or $(n_{g+1},n_g) \in C(m) \cup D(m)$. (The indices are integers mod $h$.)  By hypothesis there are $n_1$ and $n_2$ such that $(n_1,n_2) \in C(m)$.  If we have already chosen distinct $n_1, \ldots, n_g$ satisfying the required condition, then there is $n_{g+1} \ne n_{g-1}$ such that either $(n_g,n_{g+1}) \in C(m) \cup D(m)$ or $(n_{g+1},n_g) \in C(m) \cup D(m)$.  Since $N$ is finite, continuing in this fashion leads eventually to $n_{g+1} \in \{n_1,\ldots,n_{g-2}\}$, so this process eventually constructs an allowed cycle.

  Let $n_1, \ldots, n_h$ be an allowed cycle. For each $i$ and $o$, if $(i,o) = (n_g,n_{g+1})$ ($(i,o) = (n_{g+1},n_g)$) for some $g$, then we say that $(i,o)$ is a  \emph{forward} (\emph{backward}) \emph{arc}.  For $\gamma \in \Re$ let $m^\gamma \in \Re^{I \times O}$ be the matrix with components
  $$m_{io}^\gamma = \begin{cases}
  m_{io} + \gamma, & \text{$(i,o)$ is a forward arc}, \\
  m_{io} - \gamma, & \text{$(i,o)$ is a backward arc}, \\
  m_{io}, & \text{otherwise}. \\
  \end{cases}$$
  Let $\alpha$ be the smallest positive number such that one of the following occurs:
  \begin{enumerate}
    \item[(a)] $m^\alpha_{io} \in \In$ for some $(i,o) \in C(m)$.
    \item[(b)] $\sum_i  m^\alpha_{io} \in \In$ for some $(o,t) \in D(m)$.  
  \end{enumerate}
  Let $\beta$ be the smallest positive number such that $m^{-\beta}$ satisfies one of these conditions.   Let $m^0 = m^\alpha$ and $m^1 = m^{-\beta}$, so that  $m = \tfrac{\beta}{\alpha + \beta}m^0 + \tfrac{\alpha}{\alpha + \beta}m^1$.  
  
  For each $i$ and $g$ such that $n_g = i$, $(i,n_{g-1})$ is a backward arc and $(i,n_{g+1})$ is a forward arc, so $\sum_o m_{io}^\gamma = \sum_o m_{io} = r_i$ for all $\gamma$.  
  Since $E$ is integral, it follows that $m^0$ and $m^1$ satisfy all the constraints defining $Q$.  It is now easy to see that $m^0$ and $m^1$ satisfy (a)--(c) of the statement.
\end{proof}  

To generate a random integral allocation whose expectation is the given $m$ we repeatedly execute the computation described in this argument, passing to $m^0$ with probability $\tfrac{\beta}{\alpha + \beta}$ and  passing to $m^1$ with probability $\tfrac{\alpha}{\alpha + \beta}$.  The number of times this must be done is bounded by the number of elements of the nonintegrality set of $m$, and the running time of each step is bounded by a the maximum size of a cycle, which is also  the number of elements of the nonintegrality set at that step.  Thus this algorithm's worst case complexity is bounded by a constant times the square of the number of elements of the nonintegrality set of $m$, which is at most $\sum_i |\alpha_i|$ in our intended application.

\section{GCPS Schools} \label{sec:GCPSSchools}

For the application to school choice a version of the algorithm described in Section \ref{sec:Procedure} has been encoded, using the \texttt{C} programming language, as an executable \texttt{gcps}, in the software package \texttt{GCPS Schools}\footnote{To obtain the software, open the url \texttt{https://github.com/Coup3z-pixel/SchoolOfChoice/} in a web browser.  Clicking on the file \texttt{gcps\_schools.tar} opens a page for that file.  Clicking the \texttt{raw} button on the line for the file downloads the file to your browser.  After placing the file in a suitable directory, in a Unix command line terminal at that directory give the command \texttt{tar xvf gcps\_schools.tar}.  In the directory \texttt{gcps\_schools} created by that command the document \emph{GCPS\_Schools\_User\_Guide.pdf} has further instuctions.}.  \texttt{GCPS Schools} also contains two other executables, \texttt{make\_ex} and \texttt{purify}.  The second of these is a straightforward implementation of the algorithm of \cite{bckm13aer} described in Section \ref{sec:Implementability}, which passes from the output of \texttt{gcps} to a random deterministic allocation whose distribution induces the assignment probabilities computed by \texttt{gcps}.  

The executable \texttt{make\_ex} generates random school choice problems of the sort that might occur in large school districts.  The schools and students are spaced uniformly around a circle.  Each student's safe school is the school that is closest to her home.  Each school has a random \emph{valence}, which is normally distributed, for each student-school pair there is a normally distributed \emph{idiosyncratic shock}, and the student's utility for a seat in the school is the sum of these two quantities minus the distance between her home and the school.  The schools that the student is eligible for are those that provide at least as much utility as the safe school, and the student's preference over such schools is the one induced by these utilities.

Table 1 reports the result in which \texttt{gcps} was applied to two series of examples produced by \texttt{make\_ex}.  In the first series the number of schools is fixed at 10 while the number of seats per school increases from 10 to 100, while in the second series the number of seats per school is fixed at 10 while the number of schools increases from 10 to 100.  In all examples there are nine students for every ten seats, so in both series the number of students increases from 90 to 900.

In order to explain Table 1 we review the main features of the algorithm.  The path $p$ of the GCPS allocation process, and the path $\barp$ of the feasible allocation such that $p(t) \le \barp(t)$ for all $t$, are both piecewise linear, so the combined function $(p,\barp)$ is also piecewise linear.  Having arrived at the values of these functions at a particular point in time, and having determined directions for $p$ and $\barp$, the algorithm computes the amount of time that these directions can be followed before some constraint becomes binding, and it computes the parameters of the residual problem that results from following these directions for that amount of time.  The arithmetic burden of one of these computations is proportional to the number of student-school pairs such that the school is possible for the student, i.e, the number of students times the average number of schools that are possible for a student.  In both series of examples the number of linear pieces (the variable \emph{Segments} in the table) is approximated, quite roughly, by the number of students divided by four.  Thus the overall burden of computing the times at which segments end, and the new values of the parameters at the endpoints, seems to be proportional to the square of the number of students.

\bigskip

\centerline{
\begin{tabular}{| r r r r r r |}
\hline
Schools & Seats/sch & Segments & Splits & Pivots & h-sum \\
\hline
10  &  10    &     25    &    4  &     261 &    396  \\
10  &  20    &     49    &    5  &    1519 &   1965  \\
10  &  30    &     97    &    6  &    3736 &   5037  \\
10  &  40    &    109    &    6  &    6276 &   9500  \\
10  &  50    &    145    &    6  &   11788 &  15718  \\
10  &  60    &    183    &    6  &   13162 &  19021  \\
10  &  70    &    233    &    6  &   22617 &  33373  \\
10  &  80    &    285    &    6  &   35721 &  58331  \\
10  &  90    &    236    &    6  &   36191 &  54515  \\
10  &  100   &    255    &    6  &   42536 &  64683  \\
    &        &           &       &         &         \\
10  &  10    &     25    &    4  &     261 &    396  \\
20  &  10    &     42    &    9  &     555 &    582  \\
30  &  10    &    111    &   19  &    2433 &   3551  \\
40  &  10    &    102    &   27  &    2805 &   4110  \\
50  &  10    &    148    &   31  &    4456 &   6370  \\
60  &  10    &    181    &   37  &    7540 &  11027  \\
70  &  10    &    215    &   44  &    9985 &  15915  \\
80  &  10    &    254    &   51  &   12153 &  19032  \\
90  &  10    &    249    &   52  &   14612 &  22736  \\
100  &  10   &    250    &   57  &   16035 &  23975  \\
\hline
\end{tabular} 
}
\medskip
\centerline{
Table 1: numbers of events during \texttt{gcps} computations
}

\bigskip
After following the trajectories of $p$ and $\barp$ to the end of a segment, the algorithm either computes a new trajectory for $\barp$ that allows the trajectory of $p$ to continue, or it finds a critical pair $(P,J)$.  If it finds such a pair, and $P$, $J$, $P^c$, and $J^c$ are all nonempty, then it descends recursively to the derived problems for $(J,P)$ and $(J^c,P^c)$, and we say that the process \emph{splits}.  The combined complexity of the two subproblems is less than the complexity of the problem from which they are derived, so such events do not give rise to complexity concerns.  The number of such events (throughout the recursive descent) is not greater than the number of schools minus one.  From Table 1 we see that the actual number of such events is roughly half the number of schools.

In the computation  of  a critical pair $(P,J)$ or a new direction for $\barp$ that allows $p$ to continue in the same direction, the computation starts with an $|I| \times |O|$ matrix of integers $\theta$.  This matrix is modified repeatedly until it reaches a satisfactory state, and we say that each individual modification is a \emph{pivot}.  In Section \ref{sec:Procedure} we explained that each pivot decreases a certain quantity by one, the matrix is satisfactory when this quantity is nonpositive, and the initial value of this quantity cannot be greater than the number of students, so the number of pivots cannot exceed the number of students.  The fifth column of Table 1 reports the total number of pivots.  In the first sequence of experiments the number of pivots per segment increases from roughly 10 to roughly 160, and in the second sequence of experiment  this number increases from roughly 10 to roughly 64.  If the number of pivots per segment does not grow more rapidly that the number of students, and the number of segments is roughly proportional to the number of students, then the computational burden of computing pivots is roughly proportional to the square of the number of students times the average cost of a pivot. 

A pivot that does not result in a critical pair decreases $\theta_{i_1o_0},  \theta_{i_2o_1}, \ldots, \theta_{i_ho_{h-1}}$ by one and increases  $\theta_{i_1o_1}, \theta_{i_2o_2}, \ldots,  \theta_{i_ho_h}$ by one, for some sequences $o_0, o_1, \ldots, o_h$ of schools and $i_1, \ldots, i_h$ of students.  The quantity \emph{h-sum} is the sum, across all pivots, of the integer $h$.  In all cases it is between one and two times the number of pivots, so the average cost of a pivot does not seem to grow with the size of the problem.

Thus both the cost of computing endpoints of segments, and the cost of pivoting, seem to be at worst proportional to the square of the number of students.  The example with 100 schools and 10 seats and 9 students per school has a running time of 1.6 seconds, which suggests that even for the world's largest school choice problem (New York City, with over 500 schools and 100,000 students) the algorithm should run to completion within a few hours.

As we mentioned at the end of Appendix \ref{sec:Implementability}, the running time of the BCKM algorithm described there is bounded by a constant times the square of $\sum_i |\alpha_i|$, and thus by a constant times $|I|^2$ for any given bound on $|\alpha_i|$.  We have not done systematic experiments on  \texttt{purify}, but in our experience it runs very quickly, as one would expect if the cycles that it finds in the graph of nonintegral entries of the matrix $m$ are typically short.  This intuition, and our experience, strongly suggest that the running time of \texttt{purify} will not be a factor that limits the applicability of the GCPS mechanism.

\section{Eating Function Analysis} \label{app:Eating}

In this appendix we prove Theorems \ref{th:Augmentation} and \ref{th:StrategyProof}.  Both proofs are based on a detailed analysis of the consequences of manipulation in terms of its effect on the continuous time eating process of BM, as generalized here.  At this point we prepare both of the proofs by studying general aspects of this analysis.  

We fix a  CEE $E = (I,O,r,q,g)$ that satisfies the GMC and a profile $\succ \; = (\succ_j)_{j \in I}$ of preferences over $O$. 
For each $j$ we extend $\succ_j$ to a strict preference over $O \cup \{\emptyset\}$ by specifying that $o \succ_j \emptyset$ for all $o \in O$.  
Let $T = \max_i r_i$.

For $j \in I$ and $t \in [0,T]$, an \emph{eating schedule} on $[0,t)$ is a function $e_j \colon [0,t] \to O \cup \{\emptyset\}$ or  $e_j \colon [0,t) \to O \cup \{\emptyset\}$
that is  piecewise constant (i.e., changes objects finitely many times)  and right continuous: for any $t' \in [0,t)$ there is an $\varep > 0$ such that $e_j(t'') = e_j(t')$ for all $t'' \in [t', t'+\varep)$.  
For such an $e_j$, $o \in O$, and $t' \in [0,t]$ let
$$p_{jo}(e_j,t') = \int_0^{t'} \bone_{e_j(s) = o} \, ds.$$
An \emph{eating function} on $[0,t)$ is a vector $e = (e_j)_{j \in I}$ of eating schedules on $[0,t)$.
For $t' \in [0,t]$ let $p(e,t') \in \Re_+^{I \times O}$ be the allocation with components $p_{jo}(e_j,t')$. 

For $J \subset I$, $P \subset O$, and $t' \in [0,t]$ let 
$$s_{(J,P)}(e,t') = \sum_{o \in P} q_o - \sum_{i \in J} r_i + \sum_{i \in J} \sum_{o \in P^c} g_{io} - \sum_{i \in J^c} \sum_{o \in P} p_{io}(e,t').$$  For each $j \in I$, $o \in O$, $J \subset I$, and $P \subset O$ let $$\tau_{jo}(e) = \sup \{\, t' : p_{jo}(e,t') < g_{jo} \,\} \quad \text{and} \quad \tau_{(J,P)}(e) = \sup \{\, t' : s_{(J,P)}(e,t') > 0 \,\}.$$
For $j \in I$ and $t' \in [0,t]$, if $\sum_o p_{jo}(e,t') = r_j$ let $\alpha_j(e,t') = \{\emptyset\}$, and otherwise let
$$\alpha_j(e_j,t') = \alpha_j  \setminus \Big(\{\, o : p_{jo}(e_j,t') = g_{io} \,\} \cup \bigcup_{J \subset I, P \subset O \, : \, \text{$j \in J^c$ and $s_{(J,P)}(e,t') \le 0$}} P \Big).$$   Note that $\alpha_j(\cdot,t')$ is right continuous.
If $\alpha_j(e,t') \ne \emptyset$, let $e^{\succ_j}_j(e,t')$ be its $\succ_j$-best element  We say that $e_j$ is \emph{myopic} for $e$ if,  for all $t' \in [0,t)$, $\alpha_j(e,t') \ne \emptyset$ and
$e_j(t') = e^{\succ_j}_j(e,t')$.

We now fix a particular $i \in I$.

\begin{lem} \label{lemma:ExtendEatBound}
  For any $t \in [0,T]$ and any eating schedule $e_i$ on $[0,t)$ for $i$, if $e_i(0) \in \alpha_i$, then there is a unique $\bart \in [0,t]$ and  unique eating schedules $e_j$ on $[0,\bart)$ for $j \ne i$ such that if $e_{-i} = (e_j)_{j \ne i}$ and $e = (e|_{[0,\bart)}, e_{-i})$, then:
  \begin{enumerate}
    \item[(a)] for all $t' \in [0,\bart)$, $e_i(t') \in \alpha_i(e,t')$;
    \item[(b)] for each $j \ne i$, $e_j$ is myopic for $e$;
    \item[(c)] either $\bart = t$, or $e_i(\bart) \notin \alpha_i(e,\bart)$, or there is some $j \in I$ such that $\alpha_j(e,\bart) = \emptyset$.
  \end{enumerate}
\end{lem}

\begin{proof}
  For sufficiently small $\varep > 0$, if, for each $j \ne i$, $e_j$ is the constant function on $[0,\varep)$ with value $e^{\succ_j}_j(0)$, $e_{-i} = (e_j)_{j \ne i}$ and $e = (e|_{[0,\bart)}, e_{-i})$, then for all $t' \in [0,\bart)$, $e_i(t') \in \alpha_i(e,t')$, and for each $j \ne i$, $e_j$ is myopic for $e$.  Therefore we can fix $\bart \in [0,t]$ and a vector of eating schedules $e_{-i}$ on $[0,\bart)$ for $j \ne i$ such that if $e = (e|_{[0,\bart)}, e_{-i})$, then for each $j \ne i$, $e_j$ is myopic for $e$, and for all $t' \in [0,\bart)$, $e_i(t') \in \alpha_i(e,t') \cap \alpha_i(e',t')$.

  Suppose that $e'_{-i}$ is also a vector of eating schedules on $[0,\bart)$ for $j \ne i$ such that if $e' = (e|_{[0,\bart)}, e'_{-i})$, then for each $j \ne i$, $e'_j$ is myopic for $e'$, and for all $t' \in [0,\bart)$, $e_i(t') \in \alpha_i(e',t')$.  For each $j \ne i$ we have $e_j(0) = e_j^{\succ_j}(e,0) = e_j^{\succ_j}(e',0) = e'_j(0)$, so $e_{-i}$ and $e'_{-i}$ agree on the degenerate interval $[0,0]$.  If $\hatt \in [0,\bart)$ and $e_{-i}$ and $e'_{-i}$ agree on $[0,\hatt]$, then $\alpha(e,\hatt) = \alpha_j(e',\hatt)$ for all $j \in I$, so for some $\varep > 0$ they agree on $[0,\hatt + \varep)$.  Therefore, for the given $\bart$, the vector $e_{-i}$ is unique.

  If $\bart < t$, $e_i(\bart) \in \alpha_i(e,\bart)$, and $\alpha_j(e,\bart) \ne \emptyset$ for all $j \ne i$, then for some $\varep > 0$ we can extend $e_{-i}$ to $[0,\bart + \varep)$ by setting $e_j(t') = e^{\succ_j}_j(e,\bart)$ for all $t' \in [\bart,\bart+\varep)$, and each extended $e_j$ will be myopic for the extended $e$.  It follows that there is a unique maximal $\bart$, which satisfies (c).
\end{proof}

In the circumstance described in the last result we let $\bart(e_i)$, denote the maximal $\bart$, and we say that the vector $e_{-i}$ of eating schedules on $[0,\bart(e_i))$ is \emph{induced} by $e_i$.  An eating schedule $e_i \colon [0,t] \to O$ for $i$ is \emph{feasible} if $\bart(e_i) = T$.

Arguments similar to those used to prove Lemma \ref{lemma:ExtendEatBound} imply the next result, so we do not provide a proof.

\begin{lem} \label{lemma:ExtendEat}
  There is a unique eating function $e$ such that for each $j$, $e_j$ is myopic on $[0,\infty)$.
\end{lem}

We now fix a particular object $o^*$.  
We study how the allocation changes as  the parameter $\rho = g_{io^*}$ varies.  
We assume that for each $\rho$, the GMC is satisfied by the CEE $E^\rho = (I,O,r,q,g^\rho)$ that is obtained by replacing $g_{io^*}$ with $\rho$.
For $\rho \in [0,1]$ let $e^\rho$ be the unique eating function that, for each $j$, is $\succ_j$-myopic for $E^\rho$.

If the time at which $j$ begins consuming $o$ is a piecewise linear function of $\rho$, then $\tau_{jo}(e^\rho)$ is a piecewise linear function of $\rho$ on $\{\, \rho : \tau_{jo}(e^\rho) < \infty \,\}$.  Similarly, if the times at which elements of $J^c$ start consuming elements of $P$, and the times prior to $\tau_{(J,P)}(e^\rho)$ at which they stop consuming them, are piecewise linear functions of $\rho$, then $\tau_{(J,P)}(e^\rho)$ is a piecewise linear function on $\{\, \rho : \tau_{(J,P)}(e^\rho) < \infty \,\}$.  By induction over increasing start and stop times, these are piecewise linear functions on the sets of $\rho$ on which they are finite.

We say that $\rho_0 \in (0,1)$ is \emph{generic} if, for some $\varep > 0$, there are  affine functions $$t_0, t_1, \ldots, t_K \colon \cI^{\rho_0,\varep} \to [0,1]$$ where $\cI^{\rho_0,\varep} = (\rho_0 - \varep,\rho_0 + \varep)$, such that $0 \equiv t_0 < t_1 < \cdots < t_K$ and:
\begin{enumerate}
  \item[(a)] for each $j$ there is some $k$ such that $t_k \equiv r_j$;
  \item[(b)] for each $j$ and $o$, if $\tau_{jo}(e^\rho) < \infty$ for some $\rho \in \cI^{\rho_0,\varep}$, then there is a $k$ such that $t_k = \tau_{{jo}}|_{\cI^{\rho_0,\varep}}$;
  \item[(c)]  for each $J \subset I$ and $P \subset O$, if $\tau_{(J,P)}(e^\rho) < \infty$ for some $\rho \in \cI^{\rho_0,\varep}$, then there is a $k$ such that $t_k = \tau_{(J,P)}|_{\cI^{\rho_0,\varep}}$.
\end{enumerate}
The interval $[0,1]$ of possible values of $\rho = g_{io^*}$ is partitioned into finitely many nongeneric values and finitely many open intervals, in each of which there is such a system of functions.  For each $k$ let $\sigma_k$ be the number such that 
$t_k(\rho) = t_k(\rho_0) + \sigma_k(\rho - \rho_0)$ for $\rho \in \cI^{\rho_0,\varep}$.

For each $j$ and $k = 1, \ldots, K$ there is an $o_{jk} \in O \cup \{\emptyset\}$ such that $e_j^\rho(t) = o_{jk}$ for all $\rho \in \cI^{\rho_0,\varep}$ and $t \in [t_{k-1}(\rho),t_k(\rho))$, and there is a number $\kappa_{j,k-1}$ such that
$$p_{jo_{jk}}(e^\rho,t) = p_{jo_{jk}}(e^{\rho_0},t) + \kappa_{j,k-1} (\rho - \rho_0)$$
for all $\rho \in \cI^{\rho_0,\varep}$ and $t \in [t_{k-1}(\rho),t_k(\rho)) \cap [t_{k-1}(\rho_0),t_k(\rho_0))$.  These satisfy the following conditions:
\begin{enumerate}
  \item[(a)] $\kappa_{j,0} = 0$ for all $j$;
  \item[(b)] if $o_{j,k+1} = o_{jk}$, then $\kappa_{jk} = \kappa_{j,k-1}$;
  \item[(c)] if $t_k \equiv r_j$, then $\kappa_{jk} = 0$;
  \item[(d)] if $t_k = \tau_{{jo_{jk}}}|_{\cI^{\rho_0,\varep}}$, then $\kappa_{jk} = -\sigma_k$;
  \item[(e)] if $t_k = \tau_{(J,P)}|_{\cI^{\rho_0,\varep}}$, $j \in J^c$, and $o_{jk} \in P$, then $\kappa_{jk} = -\sigma_k$.
\end{enumerate}

For each $k = 1, \ldots, K$ let $\cP_k = \{\, (J,P) : t_k = \tau_{(J,P)}|_{\cI^{\rho_0,\varep}} \,\}$.  For $(J,P) \in \cP_k$ let $L_{(J,P)} = \{\, j \in J^c : o_{jk} \in P \,\}$.  Let $k^*$ be the integer such that $t_{k^*} = \tau_{{io^*}}|_{\cI^{\rho_0,\varep}}$.  Let $k^{**}$ be the smallest integer greater than $k^*$ such that there is a $(J,P) \in \cP_k$ such that $o^* \in P$, if such an integer exists.  The numbers $\sigma_k$ satisfy the following conditions:
\begin{enumerate}
  \item[(a)] $\sigma_{k^*} = 1$;
  \item[(b)] if $t_k \equiv r_j$ for some $j$, then $\sigma_k = 0$;
  \item[(c)] if $t_k = \tau_{{jo}}|_{\cI^{\rho_0,\varep}}$ for some $j$, then $\sigma_k = -\kappa_{j,k-1}$;
  \item[(d)] if $(J,P) \in \cP_k$ and $k \ne k^{**}$ or $o^* \notin P$, then $\sigma_k = -\sum_{j \in L_{(J,P)}} \kappa_{j,k-1}/|L_{{(J,P)}}|$;
  \item[(e)] if $(J,P) \in \cP_{k^{**}}$ and $o^* \in P$, then $\sigma_{k^{**}} = -(1 + \sum_{j \in L_{(J,P)}} \kappa_{j,k-1})/|L_{{(J,P)}}|$.
\end{enumerate}

\begin{lem} \label{lemma:IndirectEffect}
  If $k^* \le k < k^{**}$, then $\kappa_{jk} \le 0$ for all $j$, and $\sum_j \kappa_{jk} = -1$.  If $k^{**} \le k$, then $\sum_j \kappa_{jk} = 0$, and if $I^-_k = \{\, j : \kappa_{jk} < 0 \,\}$ and $I^+_k = \{\, j : \kappa_{jk} > 0 \,\}$, then $\sum_{j \in I^-_k} \kappa_{jk} \ge -1$ and $\sum_{j \in I^+_k} \kappa_{jk} \le 1$.
\end{lem}

\begin{proof}
  We argue by induction on $k$.  Clearly the assertion holds when $k = k^*$, and we may suppose that it holds with $k - 1$ in place of $k$. First suppose that $k \ne k^{**}$. Since the set of critical pairs is a lattice, the set of $L_{(J,P)}$ such that $(J,P)$ is a minimal element of $\cP_k$ is a partition of $\bigcup_{(J,P) \in \cP_k} L_{(J,P)}$.  For $j$ outside this union we have $\kappa_{jk} = \kappa_{j,k-1}$.  If $(J,P)$ is a minimal element of $\cP_k$, and $k \ne k^{**}$, then $\kappa_{jk} = \sum_{j' \in L_{(J,P)}} \kappa_{j',k-1}/|L_{(J,P)}|$ is an average of numbers between $-1$ and $1$, or  between $-1$ and $0$ if $k < k^{**}$, so it lies in the same range, and $\sum_j \kappa_{jk} = \sum_j \kappa_{j,k-1}$. If $k > k^{**}$ this averaging cannot increase $-\sum_{j \in I^-_k} \kappa_{jk}$ or $\sum_{j \in I^+_k} \kappa_{jk}$, but it will decrease them if there is a minimal $(J,P) \in \cP_k$ such that $\kappa_{j,k-1}$ is positive for some $j \in L_{(J,P)}$ and negative for others.

  For $k = k^{**}$ the same style of argument, with obvious modifications, clearly leads to the desired conclusion.
\end{proof}

\subsection{The Proof of Theorem \ref{th:Augmentation}}

For the given $i$, let $o^*$ be an element of $O \setminus \alpha_i$, let $\alpha_i' = \alpha_i \cup \{o^*\}$, and let $\succ_i'$ be a preference over $O$ that has $\alpha_i'$ as the set of schools weakly preferred to the safe school, and that
agrees with $\succ_i$ on $\alpha_i$.  We wish to show that the augmentation manipulation of reporting $\succ_i'$ rather than $\succ_i$ results in an allocation for $i$ that is weakly $sd(\succ_i)$ worse. 
Our method of analysis is to study how $i$'s allocation changes as  the parameter $\rho = g_{io^*}$ varies continuously between $0$ and $1$.  For $\rho \in [0,1]$ let $e^\rho$ be the eating function of the GCPS mechanism when $g_{io^*} = \rho$.  As we saw above, for each $o \in \alpha_i$, the probability that $i$ receives a seat in a school that is at least as good as $o$ is a piecewise linear function of $\rho$. Let $(\rho_0 - \varep,\rho_0 + \varep)$ be an interval of generic values of $\rho$.  

If $o \succ' o^*$, then $i$'s total consumption of schools preferred to $o$ is unaffected by $g_{io^*}$.  For $\rho$ in this interval it may be the case that $i$ is excluded from consuming $o^*$ before $i$ has finished consuming $g_{io^*}$ units, and it is possible that $i$ does not finish consuming all $g_{io^*}$ units at time $1$.  In both these cases small variations of $g_{io^*}$ do not change the allocation. Finally, if $o^* \succ_i' o$, then Lemma \ref{lemma:IndirectEffect} implies that the total consumption of schools that are $\succ_i'$ preferred to $o$ does not increase more rapidly than $g_{io^*}$ as this variable increases, so  the total consumption of schools that are $\succ_i$ preferred to $o$ does not increase as $g_{io^*}$ increases.  Since this is the case for each of the finitely many generic intervals, it holds also for large increases, including the increase from $g_{io^*} = 0$ to $g_{io^*} = 1$.

\subsection{The Proof of Theorem \ref{th:StrategyProof}} \label{app:StrategyProof}

We now assume that $E$ is a school choice problem, so $r_j = 1$ for all $j$.  From this point forward $\emptyset$ denotes an artificial object $\emptyset$ that is not an element of $O$, and that is available in unlimited supply, and we let $\hO = O \cup \{\emptyset\}$.  Intuitively, consuming $\emptyset$ may be thought of as refraining from consumption.

We now fix a particular $i$ whose possible manipulation we study.
Below we consider eating schedules $e_i \colon [0,1] \to \hO$ for $i$, while the other agents eat only from $O$.  In the first phase of the analysis we begin with an eating schedule $e_i^0$ on $[0,1)$.  Consider an open interval $\cI^{\rho_0,\varep} = (\rho_0 - \varep,\rho_0 + \varep)$ such that $e_i^0$ is constant on this interval with value $o^* \in O$, and for each $\rho$ in this interval we let $e_i^\rho$ be the eating schedule that is constant on $(\rho_0 - \varep,\rho)$ with value $\emptyset$ and agrees with $e_i^0$ elsewhere.  Let  $e^\rho_{-i}$ be the profile of eating schedules on $[0,\bart(e_i))$ induced by $e_i^\rho$, and let $e^\rho = (e_i^\rho,e_{-i}^\rho)$.  We assume that $\cI^{\rho_0,\varep} \subset [0,\bart(e_i))$.

We assume that $\rho_0$ is generic and that $\varep$ is small enough that there are affine functions $$t_0, t_1, \ldots, t_K \colon \cI^{\rho_0,\varep} \to [0,1]$$ with $0 \equiv t_0 < t_1 < \cdots < t_K \equiv 1$ such for each $\rho \in \cI^{\rho_0,\varep}$ and each $k = 1, \ldots, K-1$, $t_k(\rho) = \tau_{jo}(e_j^\rho)$ for some $j$ and $o$ or $t_k(\rho) = \tau_{(J,P)}(e^\rho)$ for some $(J,P)$,  and for each $\rho$ and $k = 1, \ldots, K-1$ there are no $j$ and $o$ such that $t_{k-1}(\rho) < \tau_{jo}(e^\rho) < t_k(\rho)$ and no pairs $(J,P)$ such that $t_{k-1}(\rho) < \tau_{(J,P)}(e^\rho) < t_k(\rho)$.  By continuity, for each $k = 1, \ldots, K-1$ the set of pairs $(j,o)$ such that $t_k(\rho) = \tau_{jo}(e_j^\rho)$
and the set $\cP_k$ of pairs $(J,P)$ such that $\tau_{(J,P)}(e^\rho) = t_k(\rho)$ are the same for all $\rho \in \cI^{\rho_0,\varep}$.   

For $j \ne i$ and $k = 0, \ldots, K-1$ let $o_{jk} = e^{\rho_0}_j(t_{k-1}(\rho_0))$. By right continuity and $\succ_j$-maximization, $e^{\rho}_j(t) = o_{jk}$ for all $\rho$ and $t \in [t_{k-1}(\rho),t_k(\rho))$.    For $k = 1, \ldots, K$ and $(J,P) \in \cP_k$ let $$L_{k(J,P)} = \{\, j \ne i : \text{$o_{jk} \in P$ and $j \in J^c$} \,\}.$$

For each $k = 1, \ldots, K$ and $j \ne i$ let $\kappa_{j,k-1}$ be the number such that $$p_{jo_{jk}}(e_j^\rho,t) = p_{jo_{jk}}(e_j^{\rho_0},t) +  \kappa_{j,k-1} \cdot (\rho - \rho_0),$$  for all $t$ such that
  $t_{k-1}(\rho) \le t \le t_k(\rho)$.
Let $\sigma_0, \ldots, \sigma_K$ be the numbers such that 
$$t_k(\rho) = t_k(\rho_0) + \sigma_k \cdot (\rho - \rho_0)$$ for all $k$ and $\rho \in \cI^{\rho_0,\varep}$.
We have $\kappa_{jk} = -\sigma_k$ for all $j \in L_{k(J,P)}$, and $\kappa_{jk} = \kappa_{j,k-1}$ for all $j$ such that $o_{j,k+1} = o_{jk}$ or $p_{jo_{jk}}(e^\rho,t_k(\rho)) = g_{j,o_{jk}}$.  

Suppose that there is a $k_0$ such that  there is a $(J,P) \in \cP_{k_0}$ such that $o^* \in P$ and $i \in J^c$.  Since $\sum_{j \in J^c} \sum_{o \in P} p_{jo}(e_j^\rho,t_{k_0}(\rho))$ is constant, $|L_{k_0(J,P)}| \cdot \sigma_{k_0} - 1 = 0$ and thus
$$\sigma_{k_0} = 1/|L_{k(J,P)}|.$$
For all $k$ and $(J,P) \in \cP_k$ such that $o^* \notin P$ or $i \notin J^c$, since $\sum_{j \in J^c} \sum_{o \in P} p_{jo}(e_j^\rho,t_k(\rho))$ is constant,  we have $|L_{k_0(J,P)}| \cdot \sigma_{k_0} + \sum_{j \in L_{k(J,P)}} \kappa_{j,k-1} = 0$ and thus
$$\sigma_k = - \sum_{j \in L_{k(J,P)}} \kappa_{j,k-1}/|L_{k(J,P)}|.$$  

By the inductive argument in the proof of Lemma \ref{lemma:IndirectEffect}, we find that if there is no $k_0$ such that  there is a $(J,P) \in \cP_{k_0}$ such that $o^* \in P$ and $i \in J^c$, then $\kappa_{jk} = 0$ for all $j \ne i$ and $k = 0, \ldots, K$.  If there is such a $k_0$, then $\kappa_{jk} = 0$ for all $j \ne i$ and $k < k_0$, $\kappa_{jk} \le 0$ for all $j \ne i$ and $k \ge k_0$, and induction also gives $\sum_{j \ne i} \kappa_{jk} = -1$ for all $k \ge k_0$.  If $(J,P) \in \cP_k$, then 
$$\tfrac{\partial}{\partial \rho} \tau_{(J,P)}(e^\rho) = x_k \in [0,1] \quad \text{and} \quad \tfrac{\partial}{\partial \rho} s_{(J,P)}(e^\rho) = - \sum_{j \in L_{k(J,P)}} \kappa_{j,k-1} \in [0,1]$$

  
\begin{lem} \label{lemma:FeasibleNew}
  Suppose that $e_i \colon [0,1) \to O$ and $\bare_i \colon [0,1) \to \hO$ are eating schedules for $i$ such that $\{\, t : e_i(t) \ne \bare_i(t) \,\} = \bare_i^{-1}(\emptyset)$.  Let $e_{-i}$ and $\bare_{-i}$ be the profiles of eating schedules on $[0,\bart(e_i))$ and $[0,\bart(\bare_i))$ induced by $e_i$ and $\bare_i$, and let $e = (e_i,e_{-i})$ and $\bare = (\bare_i,\bare_{-i})$.  For $t \in [0,1]$ let
  $$\delta(t) = \int_0^t \bone_{\bare_i(s) \ne e_i(s)} ds.$$
  If $e$ is feasible, then $\bare$ is feasible, and for all $(J,P)$ and $t \in [0,1]$, 
  $$0 \le \tau_{(J,P)}(\bare) - \tau_{(J,P)}(e) \le \delta(\tau_{(J,P)}(\bare)) \quad \text{and} \quad 0 \le s_{(J,P)}(\bare,t) - s_{(J,P)}(e,t)  \le \delta(t).$$
\end{lem}

\begin{proof}
  In the obvious way we can create a path $\rho \mapsto e_i^\rho$ from $[0,\delta(1)]$ to the space of eating schedules, with $e_i^0 = e_i$ and $e_i^{\delta(1)} = \bare_i$, that traverses each of the finitely many intervals in $[0,1]$ along which the value of $e_i$ is some $o^* \in O$ and the value of $\bare_i$ is $\emptyset$. Due to the piecewise linear nature of the problem, each of these intervals is a finite union of closed intervals such that the hypotheses of the discussion above are satisfied on the interiors of these integrals.  
  The asserted inequalities are attained from $\sigma_k \in [0,1]$ by integrating the inequalities above, then summing over the intervals. 
  
  The 
  analysis above implies that $\bart(e^\rho) = 1$ for all $\rho$.  
  (More formally, for each $\varep > 0$ the function 
  $\rho \mapsto \bart(e^\rho)$ cannot leave the interval $(1 - \varep,1]$.)  Therefore $\bare_i$ is feasible.
\end{proof}

We now introduce a preferences $\succ_i$ and  $\succ_i'$ over $O$ for $i$, and we let $\succ = (\succ_i,\succ_{-i})$ and $\succ' = (\succ_i',\succ_{-i})$.  There are unique eating functions $e^\succ$ and  $e^{\succ'}$ that are  generated by the GCPS procedure when agents report these preferences.  Since $E$ satisfies the GMC, these eating functions are feasible.
Let $\bare_i$ be the eating schedule
$$\bare_i(t) = 
\begin{cases}
  e_i^\succ(t) & \text{if $e_i^\succ(t) = e_i^{\succ'}(t)$,} \\
  \emptyset & \text{otherwise},
\end{cases}$$
let $\bare_{-i}$ be the profile of eating schedules induced by $\bare_i$, and let $\bare = (\bare_i,\bare_{-i})$.  Lemma \ref{lemma:FeasibleNew} implies that $\bare$ is feasible.

Let $\beta(t)$, $\gamma(t)$, and $\delta(t)$ denote the sums of the lengths of time intervals, before time $t$, on which agent $i$'s consumption in the eating algorithm is $\succ_i$-preferred, $\succ_i$-less preferred, and different, respectively, when the reported preferences change from $\succ$ to $\succ'$.  Formally,
$$\beta(t) = \int_0^t \bone_{e_i^{\succ'}(s) \succ_i e_i^\succ(s)} ds \quad \text{and} \quad \gamma(t) = \int_0^t \bone_{e_i^{\succ}(s) \succ_i e_i^{\succ'}(s)} ds,$$ so that $\delta(t) = \beta(t) + \gamma(t)$.

Let
$$\{o_1, o_2, \ldots, o_{\ell}\} = \{\, o \in O : \text{$o = e_i^{\succ'}(t) \succ_i e_i^\succ(t)$ for some $t \in [0,1)$} \,\}$$
be the set of objects $o$ such that for some time $t$, $o = e_i^{\succ'}(t)$ is $\succ_i$-preferred to $e_i^\succ(t)$.  
These objects are indexed so that 
$o_1 \succ_i' o_2 \succ_i' \cdots \succ_i' o_{\ell}$.  For $l = 1, \ldots, \ell$ let
$$T_l = \inf \{\, t : o_l = e_i^{\succ'}(t) \succ_i e_i^\succ(t) \,\}$$ 
be the first time $t$ when $o_l = e_i^{\succ'}(t)$ is $\succ_i$-preferred to $e_i^\succ(t)$.  For each $l$ let 
$$T_l' = \sup \{\, t : o_l = e_i^{\succ'}(t) \,\}$$ 
Clearly, $0 < T_1 < T_1' \le T_2 < \cdots < T_{\ell - 1}' \le T_{\ell} < T_{\ell}' \le1$.  Let $T_0= 0$ and $T_{\ell + 1} = 1$.



\begin{lem} \label{lemma:KM5New}
  For all $l = 1, \ldots, \ell$, 
  $$T_l' - T_l \le \frac{\delta(T_l)}{N_0}.$$
\end{lem}

\begin{proof}
  After time $T_l$ the object $o_l$ is not available to $i$ under the eating function $e^\succ$, so there is a $P_l$ such that $o_l \in P_l$, $i \in J_{P_l}^c$, and $\tau_{P_l}(e^\succ) = T_l$. Since $s_{P_l}(T_l,e^{\succ}) = 0$, Lemma \ref{lemma:FeasibleNew} gives
  $$s_{P_l}(T_l,e^{\succ'}) = (s_{P_l}(T_l,e^{\succ'}) - s_{P_l}(T_l,\bare)) - (s_{P_l}(T_l,e^{\succ}) - s_{P_l}(T_l,\bare)) \le \delta(T_l).$$
  By assumption $i$ is one of at least $N_0$ students $j \in J_{P_l}^c$ such that $e^{\succ'}_j(t) = o_l$ for all $t \in [T_l, T_l')$, so $T_l' \le \tau_{P_l}(e^{\succ'}) \le \tau_{P_l}(e^\succ) + \delta(T_l)/N_0$.  
\end{proof}

Let $\lambda = 1 + 1/N_0$.

\begin{lem} \label{lemma:KM6}
  For all $l = 1, \ldots, \ell$,
  $T_l' - T_l \le \gamma(1)(\lambda - 1)\lambda^{l-1}$.
\end{lem}

\begin{proof}
  We prove the lemma by induction on $l$.  We have $\delta(T_1) = \gamma(T_1) \le \gamma(1) \le 1$, so Lemma \ref{lemma:KM5New} implies that $T_1' - T_1 \le \delta(T_1) /N_0 \le \gamma(1)(\lambda - 1)$. 
  Suppose that $l \ge 2$ and the induction hypothesis holds for $1, \ldots, l - 1$.  Then
  $$\delta(T_l) = \gamma(T_l) + \beta(T_l) \le \gamma(1) + \sum_{g = 1}^{l-1} \beta(T_{g+1}) - \beta(T_g) = \gamma(1) + \sum_{g = 1}^{l-1} T_g' - T_g$$
  $$\le \gamma(1)\Big(1 + (\lambda - 1)\sum_{g = 0}^{l-2}\lambda^g\Big) = \gamma(1)\lambda^{l-1}.$$
  Applying Lemma \ref{lemma:KM5New} again gives
  \begin{equation*}
  T_l' - T_l \le \delta(T_l)/N_0 \le \gamma(1)(\lambda - 1)\lambda^{l-1}. \qedhere
  \end{equation*}
\end{proof}

\begin{proof}[Proof of Theorem \ref{th:StrategyProof}.]
  We have
  $$u_i(GCPS(\succ)) - u_i(GCPS(\succ')) = \int_0^1 u_i(e_i^\succ(s)) - u_i(e_i^{\succ'}(s)) \, ds \ge d_i\gamma(1) - D_i\beta(1).$$
  Since $\beta(T_1) = 0$, adding up the inequalities from Lemma \ref{lemma:KM6} for gives
  $$\beta(1) = \sum_{g = 1}^{\ell} \beta(T_{g+1}) - \beta(T_g) = \sum_{g = 1}^{\ell} T_g' - T_g \le \gamma(1)(\lambda - 1)
  \sum_{g = 1}^{\ell}\lambda^{g-1} = \gamma(1)(\lambda^{\ell} - 1).$$ 
  Therefore
  $$u_i(GCPS(\succ)) - u_i(GCPS(\succ')) \ge \gamma(1)\big(d_i - D_i(\lambda^{\ell} - 1)\big),$$
  and since $\ell \le |O|$, this is nonnegative if
  \begin{equation*}
  \big(1 + \frac{d_i}{D_i}\big)^{1/|O|} \ge \lambda = 1 + 1/N_0. \qedhere
  \end{equation*}
\end{proof}

\end{appendix}

%%%------------------------------------------------------------------------------------
%%%------------------------------------------------------------------------------------
\end{document}

\section{Little Proof}

\begin{lem}
  For all $k$, $\sum_{j \in I} \kappa_{jk} = 0$.
\end{lem}

\begin{proof}
  Evidently $\sum_j \kappa_{jk} = 0$ for all $k \le k_2$.  We claim that $\sum_j \kappa_{jk} = \sum_j \kappa_{j,k-1}$ for all $k > k_2$, which implies (f) by induction.  By (ii) of (e) we have $\kappa_{jk} = \kappa_{j,k-1}$ if $j \in I \setminus \bigcup_{P \in \cP_k} J_{kP}$.  By (i) of (e) we have $\sum_{j \in J_{kP}} \kappa_{jk} = \sum_{j \in J_{kP}} \kappa_{j,k-1}$ for each $P \in \cP_k$.  Recalling that the set of critical pairs is a lattice, if $P, P' \in \cP_k$, then $P \cap P' \in \cP_k$ and $J_{P \cap P'} = J_P \cap J_{P'}$, hence $J_{k,P \cap P'} = J_{kP} \cap J_{kP'}$. Therefore $\sum_{j \in J_{kP} \cap J_{kP'}} \kappa_{jk} = \sum_{j \in J_{kP} \cap J_{kP'}} \kappa_{j,k-1}$ and 
  $$\sum_{j \in J_{kP} \setminus J_{kP'}} \kappa_{jk} = \sum_{j \in J_{kP} \setminus J_{kP'}} \kappa_{j,k-1}.$$  
  Therefore the set of $J \subset I$ such that $\sum_{j \in J} \kappa_{jk} = \sum_{j \in J} \kappa_{j,k-1}$ includes the algebra generated by the sets $J_{kP}$, which contains a partition of $I$, and the claim follows.
\end{proof}


\section{Proof of Theorem \ref{th:StrategyProof}} \label{app:OldStrategyProof}

Fix a school choice CEE $E = (I,O,r,q,g)$ that satisfies the GMC. It will be convenient to introduce a \emph{null object}, which we denote by $\emptyset$.  Let $\hO = O \cup \{\emptyset\}$.  
An \emph{eating function} $e = (e_j)_{j \in I}$ is a vector of eating schedules $e_j \colon [0,1] \to \hO$ that are  piecewise constant  and right continuous: for all $t < 1$ there is $\varep > 0$ such that $e_j(t') = e_j(t)$ for all $t' \in [t, t+\varep)$.
For such an $e$ and $t \in [0,1]$ let $p(e,t) \in \Re_+^{I \times O}$ be the allocation given by
$p_{jo}(e,t) = \int_0^t \bone_{e_j(s) = o} \, ds$. For $P \subset O$ recall that $s_P = \sum_{o \in P} q_o - \sum_{i \in J_P} r_i$, let 
$$\sigma_P(e,t) = s_P  -  \sum_{j \in J_P^c} \sum_{o \in P} p_{jo}(e,t),$$
and let $\tau_P(e) = \sup \{\, t \in [0,1] : \sigma_P(e,t) > 0 \,\}$.
Let
$$\bart(e) = \sup \{\, t \in [0,1] : \text{$\sigma_P(e,t) \ge 0$ for all $P \subset O$} \,\}.$$
An eating function $e$ is \emph{feasible} if $\bart(e) = 1$.

For $j \in I$ let
$$\alpha_j(e,t) = \alpha_j  \setminus \bigcup_{P \subset O \, : \, \text{$j \in J_P^c$ and $\sigma_P(e,t) \le 0$}} P$$
be the set of schools at which $j$ can eat at time $t$.   Note that $\alpha_j(\cdot,t)$ is right continuous.
We now fix a profile $\succ \; = (\succ_j)_{j \in I}$ of preferences over $\hO$ such that for each $j$, $\emptyset$ is the $\succ_j$-worst element of $\hO$.   If $\alpha_j(e,t) \ne \emptyset$, let $e^{\succ_j}_j(e,t)$ be its $\succ_j$-best element.  

We now fix a particular $i \in I$.  An eating function $e$ is \emph{$\succ_{-i}$-maximizing} if 
$e_j(t) = e^{\succ_j}_j(e,t)$ for all $j \ne i$ and all $t \le \bart(e)$. For a given $e_i$ that is piecewise constant  (changes object finitely many times) and right continuous, an $e_{-i}$ such that $e = (e_i,e_{-i})$ is $\succ_{-i}$-maximizing can be constructed inductively. Furthemore, this $e_{-i}$ is the unique profile such that  $e = (e_i,e_{-i})$ is $\succ_{-i}$-maximizing:   if $e_{-i}'$ is a second such profile and $t_0$ is the supremum of the set of $t$ such that $e_{-i}|_{[0,t]} = e_{-i}'|_{[0,t]}$, then right continuity implies that, for some $\varep > 0$, $e_{-i}$ and $e_{-i}'$ also agree on $[t_0,t_0+\varep)$, so $t_0= 1$.  We call $e_{-i}$ the \emph{profile of eating schedules induced by $e_i$}.

A (piecewise constant right continuous) eating schedule $e_i \colon [0,1] \to \hO$ for $i$ is \emph{feasible} if
$\bart(e) = 1$, where $e = (e_i,e_{-i})$  and $e_{-i}$ the profile of eating schedules induced by $e_i$.  If $e_i$ is a feasible eating schedule for $i$ and $\bare_i$ is another eating schedule for $i$ such that $\{\, t : e_i(t) \ne \bare_i(t) \,\} \subset \bare_i^{-1}(\emptyset)$, it may seem obvious that $\bare_i$ is also feasible, but proving this turns out to be quite technical.

We now consider a feasible eating schedule $e_i^0$ for $i$, a time $\rho_0 < \bart(e^0)$  where $e^0 = (e_i^0,e_{-i}^0)$  and $e_{-i}^0$ is the profile of eating schedules induced by $e_i^0$, and an $\varep > 0$ such that for some $o^* \in O$, $e_i^0$ is constant on $(\rho_0 - \varep,\rho_0)$ with value $\emptyset$ and constant on $[\rho_0,\rho_0 + \varep)$ with value $o^*$.  For $\rho \in (\rho_0 - \varep,\rho_0 + \varep)$ let $e_i^\rho$ be the eating function for $i$ that  is constant on $(\rho_0 - \varep,\rho)$ with value $\emptyset$, constant on $[\rho,e + \varep)$ with value $o^*$, and agrees with $e_i^0$ elsewhere.  Let  $e^\rho_{-i}$ be the profile of eating schedules induced by $e_i^\rho$, and let $e^\rho = (e_i^\rho,e_{-i}^\rho)$.

\begin{lem} \label{lemma:MonotoneCutoff} 
 The function $\rho \mapsto \bart(e^\rho)$ from $(\rho_0 - \varep,\rho_0 + \varep)$ to $[0,1]$ is piecewise linear.  For each $P \subset O$, $\sigma_P(e^\rho,t)$ is a piecewise linear function of those $(\rho,t)$ such that $t \le \bart(e^\rho)$, and $\tau_P(e^\rho)$ is a piecewise linear function of $\rho$.
 Suppose that, for some $\delta \in (0,\varep]$, there are affine functions $t_0, t_1, \ldots, t_K \colon (\rho_0 - \delta,\rho_0 + \delta) \to [0,1]$ with $0 \equiv t_0 < t_1 < \cdots < t_K$ such for each $\rho \in (\rho_0 - \delta,\rho_0 + \delta)$,  $t_K(\rho) = \bart(e^\rho)$ and, for each $k = 1, \ldots, K-1$, $t_k(\rho) = \tau_P(e^\rho)$ for one or more $P \subset O$ and there are no $P \subset O$ such that $t_{k-1}(\rho) < \tau_P(e^\rho) < t_k(\rho)$. Then for all $\rho \in (\rho_0 - \delta,\rho_0 + \delta)$, $P \subset O$, and  $t \in [0,1]$, the partial of $\sigma_P(e^\rho,t)$ with respect to $\rho$ is defined and lies in $[0,1]$, 
 the derivative of $\tau_P(e^\rho)$ with respect to $\rho$ is a constant that lies in $[0,1]$, and the derivative of $\bart(e^\rho)$ with respect to $\rho$ is a constant that lies in $[0,1]$.
\end{lem}

\begin{proof}
  It is clear from the definition that the functions $p_{jo}(e^\rho,t)$ are piecewise linear functions of $(\rho,t)$.  The piecewise linearity of $\bart(e^\rho)$,  $\sigma_P(e^\rho,t)$, and $\tau_P(e^\rho)$ follows from this.
  
  By continuity, for each $0 < k < K$ the set $\cP_k$ of $P \subset O$ such that $\tau_P(e^\rho) = t_k(\rho)$ is the same for all $\rho \in (\rho_0 - \delta,\rho_0 + \delta)$.  For $j \ne i$ and $k = 1, \ldots, K$ let $o_{jk} = e^{\rho_0}_j(t_{k-1}(\rho_0))$.  For $k = 1, \ldots, K$ and $P \in \cP_k$ let $J_{kP} = \{\, j \ne i : o_{jk} \in J_P^c \,\}$.
  
  We inductively define numbers $\kappa_{jk}$ for $j \in I$ and $k = 0, \ldots, K$, beginning by setting $\kappa_{i0} = 1$ and $\kappa_{j0} = 0$ for all $j \ne i$. Supposing that $k > 0$ and the numbers $\kappa_{j,k-1}$ have already been defined, there are several cases:
  \begin{enumerate}
    \item[(a)] If $\kappa_{i,k-1} = 1$ and there is no $P \in \cP_k$ such that $o^* \in P$ and $i \in J_P^c$, then $\kappa_{ik} = 1$ and $\kappa_{jk} = 0$ for all $j \ne i$.  
    \item[(b)] If $\kappa_{i,k-1} = 1$ and there is a $P \in \cP_k$ such that $o^* \in P$ and $i \in J_P^c$, then $\kappa_{jk} = 1/(|J_{kP}| + 1)$ for all $j \in J_{kP}$, and $\kappa_{jk} = 0$ for all other $j$ including $i$.  
    \item[(c)] If $\kappa_{i,k-1} = 0$, then $\kappa_{i,k} = 0$.  In addition: 
      \begin{enumerate}
        \item[(i)] If $P \in \cP_k$ and $j \in J_{kP}$, then $\kappa_{jk} = \sum_{j' \in J_{kP}} \kappa_{j',k-1}/|J_{kP}|$.
        \item[(ii)] If $j \ne i$ and there is no $P \in \cP_k$ such that $j \in J_{kP}$, then $\kappa_{jk} = \kappa_{j,k-1}$.  
      \end{enumerate}
  \end{enumerate}
  Below we will show that these numbers are well defined in spite of potential ambiguities arising from the sets $\cP_k$ having multiple members.

  To begin with consider the first $k$ such that there is a $P \in \cP_{k_0}$ such that $o^* \in P$ and $i \in J_P^c$.  For $\rho \in (\rho_0 - \delta, \rho_0 + \delta)$ we have   
  $$p_{io^*}(e^\rho,t_{k_0}(\rho)) + \sum_{j \in J_{k_0P}} p_{jo_{jk_0}}(e^\rho,t_{k_0}(\rho)) = p_{io^*}(e^{\rho_0},t_{k_0}(\rho_0)) + \sum_{j \in J_{k_0P}} p_{jo_{jk_0}}(e^{\rho_0},t_{k_0}(\rho_0)),$$  
  which reduces to
  $$-(\rho - \rho_0) + (|J_{k_0P}| + 1) \cdot (t_{k_0}(\rho) - t_{k_0}(\rho_0)) = 0$$
  and thus $t_{k_0}(\rho) = t_{k_0}(\rho_0) + (\rho - \rho_0)/(|J_{k_0P}| + 1)$.  This is true for \emph{all} $P \in \cP_{k_0}$ such that $o^* \in P$ and $i \in J_P^c$, but if $P'$ is a second such set, then $P \cap P'$ is also such a set, and we conclude that $J_P = J_{P \cap P'} = J_{P'}$, so the definition of the numbers $\kappa_{jk_0}$ is unambiguous.

  We claim that if $k > k_0$, $j \ne i$, and  $t_{k-1}(\rho) \le t \le t_k(\rho)$, then $$p_{jo_{jk}}(e^\rho,t) = p_{jo_{jk}}(e^0,t) - (\rho - \rho_0) \cdot  \kappa_{j,k-1}, \eqno{(*)}$$ and if $\rho \in (\rho_0 - \delta, \rho_0 + \delta)$, then
  $$t_k(\rho) = t_k(\rho_0) + (\rho - \rho_0) \cdot \sum_{j \in J_{kP}} \kappa_{j,k-1}/|J_{kP}|. \eqno{(**)}$$
  Clearly ($*$) holds when $k = k_0 + 1$.  If it holds for $k$ and $P \in \cP_k$, then    
  $$\sum_{j \in J_{kP}} p_{jo_{jk}}(e^\rho,t_k(\rho)) = \sum_{j \in J_{kP}} p_{jo_{jk}}(e^{\rho_0},t_k(\rho_0)),$$ 
  which reduces to
  $$(t_k(\rho) - t_k(\rho_0)) \cdot |J_{kP}| - (\rho - \rho_0) \cdot \sum_{j \in J_{kP}} \kappa_{j,k-1} = 0,$$
  and thus ($**$).  Clearly ($**$) implies ($*$) with $k+1$ in place of $k$.

  We claim that $\sum_j \kappa_{jk} = \sum_j \kappa_{j,k-1}$ for all $k > k_0$.  By definition we have $\kappa_{jk} = \kappa_{j,k-1}$ if $j \in I \setminus \bigcup_{P \in \cP_k} J_P$.  By (i) of (c) we have $\sum_{j \in J_{kP}} \kappa_{jk} = \sum_{j \in J_{kP}} \kappa_{j,k-1}$ for each $P \in \cP_k$.  Recalling that the set of critical pairs is a lattice, if $P, P' \in \cP_k$, then $P \cap P' \in \cP_k$ and $J_{P \cap P'} = J_P \cap J_{P'}$, hence $J_{k,P \cap P'} = J_{kP} \cap J_{kP'}$. Therefore $\sum_{j \in J_{kP} \cap J_{kP'}} \kappa_{jk} = \sum_{j \in J_{kP} \cap J_{kP'}} \kappa_{j,k-1}$ and 
  $$\sum_{j \in J_{kP} \setminus J_{kP'}} \kappa_{jk} = \sum_{j \in J_{kP} \setminus J_{kP'}} \kappa_{j,k-1}.$$  
  Therefore the set of $J \subset I$ such that $\sum_{j \in J} \kappa_{jk} = \sum_{j \in J} \kappa_{j,k-1}$ includes the algebra generated by the sets $J_{kP}$, which contains a partition of $I$, and the claim follows.

  The claim concerning the partial of $\sigma_P$ with respect to $\rho$ now follows from summation of ($*$) over those $j$ such that $o_{jk} \in P$.  The claim concerning the derivative of $\tau_P(e^\rho)$ with respect to $\rho$ follows from ($**$), and the claim concerning the derivative of $\bart(e^\rho)$ with respect to $\rho$ is a special case of this.
\end{proof}

\begin{lem} \label{lemma:Feasible}
  Suppose that $e_i$ and $\bare_i$ are eating schedules for $i$ such that $e_i$ is feasible and $$\{\, t : e_i(t) \ne \bare_i(t) \,\} \subset \bare_i^{-1}(\emptyset).$$  Then $\bare_i$ is feasible.  For $t \in [0,1]$ let
  $$\delta(t) = \int_0^t \bone_{\bare_i(s) \ne e_i(s)} ds.$$
  Let $e_{-i}$ and $\bare_{-i}$ be the profiles of eating schedules induced by $e_i$ and $\bare_i$, and let $e = (e_i,e_{-i})$ and $\bare = (\bare_i,\bare_{-i})$.  For all $P \subset O$ and $t \in [0,1]$, $$0 \le \tau_P(\bare) - \tau_P(e) \le \delta(\tau_P(\bare)) \quad \text{and} \quad \sigma_P(\bare,t) \ge \sigma_P(e,t) \ge \sigma_P(\bare,t) - \delta(t).$$
\end{lem}

\begin{proof}
  In the obvious way we can create a path $\rho \mapsto e_i^\rho$ from $[0,\delta(1)]$ to the space of eating schedules, with $e_i^0 = e_i$ and $e_i^{\delta(1)} = \bare_i$, that traverses each of the finitely many intervals in $[0,1]$ along which the value of $e_i$ is some $o^* \in O$ and the value of $\bare_i$ is $\emptyset$, in the manner described in the hypotheses of the last result.  There are finitely many endpoints of such intervals, and there are finitely many $\rho$ such that there are $P, P' \subset O$ such that $\tau_P(e^\rho) = \tau_{P'}(e^\rho)$ but $\tau_P(e^{\rho'}) \ne \tau_{P'}(e^{\rho'})$ for nearby $\rho'$.  Therefore the asserted inequalities can be obtained by integrating the inequalities of the last result.  The final assertion of the last result implies that $\bart(e^\rho) = 1$ for all $\rho$.  (More formally, for each $\varep > 0$ the function $\rho \mapsto \bart(e^\rho)$ cannot leave the interval $(1 - \varep,1]$.)  Therefore $\bare_i$ is feasible.
\end{proof}

We now introduce a preferences $\succ_i$ and  $\succ_i'$ over $\hO$ for $i$, and we let $\succ = (\succ_i,\succ_{-i})$ and $\succ' = (\succ_i',\succ_{-i})$.  There are unique eating functions $e^\succ$ and  $e^{\succ'}$ that are  generated by the GCPS procedure when agents report these preferences.  Since $E$ satisfies the GMC, these eating functions are feasible.
Let $\bare_i$ be the eating schedule
$$\bare_i(t) = 
\begin{cases}
  e_i^\succ(t) & \text{if $e_i^\succ(t) = e_i^{\succ'}(t)$,} \\
  \emptyset & \text{otherwise},
\end{cases}$$
let $\bare_{-i}$ be the profile of eating schedules induced by $\bare_i$, and let $\bare = (\bare_i,\bare_{-i})$.  Lemma \ref{lemma:Feasible} implies that $\bare$ is feasible.

Let $\beta(t)$, $\gamma(t)$, and $\delta(t)$ denote the sums of the lengths of time intervals, before time $t$, on which agent $i$'s consumption in the eating algorithm is $\succ_i$-preferred, $\succ_i$-less preferred, and different, respectively, when the reported preferences change from $\succ$ to $\succ'$.  Formally,
$$\beta(t) = \int_0^t \bone_{e_i^{\succ'}(s) \succ_i e_i^\succ(s)} ds \quad \text{and} \quad \gamma(t) = \int_0^t \bone_{e_i^{\succ}(s) \succ_i e_i^{\succ'}(s)} ds,$$ so that $\delta(t) = \beta(t) + \gamma(t)$.

Let
$$\{o_1, o_2, \ldots, o_{\ell}\} = \{\, o \in O : \text{$o = e_i^{\succ'}(t) \succ_i e_i^\succ(t)$ for some $t \in [0,1)$} \,\}$$
be the set of objects $o$ such that for some time $t$, $o = e_i^{\succ'}(t)$ is $\succ_i$-preferred to $e_i^\succ(t)$.  
These objects are indexed so that 
$o_1 \succ_i' o_2 \succ_i' \cdots \succ_i' o_{\ell}$.  For $l = 1, \ldots, \ell$ let
$$T_l = \inf \{\, t : o_l = e_i^{\succ'}(t) \succ_i e_i^\succ(t) \,\}$$ 
be the first time $t$ when $o_l = e_i^{\succ'}(t)$ is $\succ_i$-preferred to $e_i^\succ(t)$.  For each $l$ let 
$$T_l' = \sup \{\, t : o_l = e_i^{\succ'}(t) \,\}$$ 
Clearly, $0 < T_1 < T_1' \le T_2 < \cdots < T_{\ell - 1}' \le T_{\ell} < T_{\ell}' \le1$.  Let $T_0= 0$ and $T_{\ell + 1} = 1$.



\begin{lem} \label{lemma:KM5new}
  For all $l = 1, \ldots, \ell$, 
  $$T_l' - T_l \le \frac{\delta(T_l)}{N_0}.$$
\end{lem}

\begin{proof}
  After time $T_l$ the object $o_l$ is not available to $i$ under the eating function $e^\succ$, so there is a $P_l$ such that $o_l \in P_l$, $i \in J_{P_l}^c$, and $\tau_{P_l}(e^\succ) = T_l$. Lemma \ref{lemma:Feasible} gives
  $$\sigma_{P_l}(T_l,e^{\succ'}) = (\sigma_{P_l}(T_l,e^{\succ'}) - \sigma_{P_l}(T_l,\bare)) - (\sigma_{P_l}(T_l,e^{\succ}) - \sigma_{P_l}(T_l,\bare)) \le \delta(T_l).$$
  By assumption $i$ is one of at least $N_0$ students $j \in J_{P_l}^c$ such that $e^{\succ'}_j(t) = o_l$ for all $t \in [T_l, T_l')$, so $T_l' \le \tau_{P_l}(e^{\succ'}) \le \tau_{P_l}(e^\succ) + \delta(T_l)/N_0$.  
\end{proof}

Let $\lambda = 1 + 1/N_0$.

\begin{lem} \label{lemma:KM6}
  For all $l = 1, \ldots, \ell$,
  $T_l' - T_l \le \gamma(1)(\lambda - 1)\lambda^{l-1}$.
\end{lem}

\begin{proof}
  We prove the lemma by induction on $l$.  We have $\delta(T_1) = \gamma(T_1) \le \gamma(1) \le 1$, so Lemma \ref{lemma:KM5new} implies that $T_1' - T_1 \le \delta(T_1) /N_0 \le \gamma(1)(\lambda - 1)$. 
  Suppose that $l \ge 2$ and the induction hypothesis holds for $1, \ldots, l - 1$.  Then
  $$\delta(T_l) = \gamma(T_l) + \beta(T_l) \le \gamma(1) + \sum_{g = 1}^{l-1} \beta(T_{g+1}) - \beta(T_g) = \gamma(1) + \sum_{g = 1}^{l-1} T_g' - T_g$$
  $$\le \gamma(1)\Big(1 + (\lambda - 1)\sum_{g = 0}^{l-2}\lambda^g\Big) = \gamma(1)\lambda^{l-1}.$$
  Applying Lemma \ref{lemma:KM5new} again gives
  \begin{equation*}
  T_l' - T_l \le \delta(T_l)/N_0 \le \gamma(1)(\lambda - 1)\lambda^{l-1}. \qedhere
  \end{equation*}
\end{proof}

\begin{proof}[Proof of Theorem \ref{th:StrategyProof}.]
  We have
  $$u_i(GCPS(\succ)) - u_i(GCPS(\succ')) = \int_0^1 u_i(e_i^\succ(s)) - u_i(e_i^{\succ'}(s)) \, ds \ge d_i\gamma(1) - D_i\beta(1).$$
  Since $\beta(T_1) = 0$, adding up the inequalities from Lemma \ref{lemma:KM6} for gives
  $$\beta(1) = \sum_{g = 1}^{\ell} \beta(T_{g+1}) - \beta(T_g) = \sum_{g = 1}^{\ell} T_g' - T_g \le \gamma(1)(\lambda - 1)
  \sum_{g = 1}^{\ell}\lambda^{g-1} = \gamma(1)(\lambda^{\ell} - 1).$$ 
  Therefore
  $$u_i(GCPS(\succ)) - u_i(GCPS(\succ')) \ge \gamma(1)\big(d_i - D_i(\lambda^{\ell} - 1)\big),$$
  and since $\ell \le |O|$, this is nonnegative if
  \begin{equation*}
  \big(1 + \frac{d_i}{D_i}\big)^{1/|O|} \ge \lambda = 1 + 1/N_0. \qedhere
  \end{equation*}
\end{proof}

% \section{Proof of Lemma \ref{lem:cyclic}} \label{app:Cyclic}

% The argument below simply adapts the proof of Theorem 1 of \cite{cd16} to our setting.
 
% Suppose that $m$ is wasteful, so there is an agent $i$ and a pair $o,o'$ of objects such that $o \succ_i o'$, $m_{io} < g_{io}$, $m_{io'} > 0$, and $\sum_j m_{jo} < q_o$. For sufficiently small $\delta > 0$, setting $m'_{io} = m_{io}+\delta$,  $m'_{io'} = m_{io'} - \delta$, and $m'_{jp} = m_{jp}$ for all other $(j,p)$ gives an allocation $m'$ such that $m'_i$ $e$-dominates $m_i$ for any $e \in \{sd,dl,ul\}$, $m_i' \ne m_i$, and $m'_j = m_j$ for all $j \ne i$, so $m'$ $e$-dominates $m$. 

%Suppose that there is a cycle $o_0 \lhd_m o_1 \lhd_m \cdots \lhd_m o_h \lhd_m o_0$.
%If $i_0, \ldots, i_k$ and $m(\delta)$ are as above, then, for sufficiently small $\delta >0$, $m(\delta)$ is an allocation, $m_i(\delta)$ $e$-dominates $m_i$  for each $i = 0, \ldots, k$ and $e \in \{sd,dl,ul\}$, and $m'_j = m_j$ for all other $j$, so $m'$ $e$-dominates $m$.

% Now suppose that allocation $m$ is not wasteful and is $e$-dominated by the allocation $m'$.   Fix an agent $i_0$ such that $m'_{i_0} \ne m_{i_0}$. 
% There are two cases, depending on whether $e = dl$ or $e = ul$.  (Either argument can handle the case $e = sd$.)

%First suppose that $e = dl$.  Since $m_i'$ $ul$-dominates $m_i$ there are objects $o_0$ and $o_1$  such that $o_1 \succ_{i_1} o_0$, $m'_{i_0o_0} < m_{i_0o_0}$, and $m'_{i_0o_1} > m_{i_0o_1}$.  These conditions imply that $m_{i_0o_0} > 0$ and $m_{i_0o_1} < g_{i_0o_1}$, so $o_0 \lhd_m o_1$. 
% If $m'_{jo_1} \ge m_{jo_1}$ for all $j \ne i_0$, then $\sum_j m_{jo_1} < \sum_j m'_{jo_1} \le q_{o_1}$ and $m_{i_0o_0} > 0$, contradicting nonwastefulness of $m$.  Therefore there is some $i_1 \ne i_0$ such that $m'_{i_1o_1} < m_{i_1o_1}$.  Since $m'_{i_1}$ $dl$-dominates $m_{i_1}$, there is some $o_2$ such that $o_2 \succ_{i_1} o_1$ and $m'_{i_1o_2} > m_{i_1o_2}$.  We have $m_{i_1o_1} > 0$ and $m_{i_1o_2} < g_{i_1o_2}$, so $o_1 \lhd o_2$.
% Since $O$ is finite, repeating this argument leads eventually to a cycle $o_0 \lhd_m o_1 \lhd_m \cdots \lhd_m o_h \lhd_m o_0$.

% Now suppose that $e = ul$. Since $m_i'$ $dl$-dominates $m_i$ there are objects $o_0$ and $o_1$  such that $o_0 \succ_{i_1} o_1$, $m'_{i_0o_0} > m_{i_0o_0}$, and $m'_{i_0o_1} < m_{i_0o_1}$, so that $m_{i_0o_1} > 0$ and $o_1 \lhd o_0$. 

% Aiming at a contradiction, suppose that $m_{jo_1} \ge m'_{jo_1}$ for all $j \ne i_0$, so that $q_{o_1} \ge \sum_j m_{jo_1} > \sum_j m'_{jo_1}$.  Since $\sum_{j,p} m_{jp} = \sum_{j,p} m'_{jp} = \sum_j r_j$, there is a $p_1$ such that $\sum_j m_{jp_1} < \sum_j m'_{jp_1}$ and a $j_1$ such that $m_{j_1p_1} < m'_{j_1p_1}$. Since $m'_{j_1} \ne m_{j_1}$ and $m'_{j_1}$ $ul$-dominates $m_{j_1}$, there is a $p_2$ such that $p_1 \succ_{j_1} p_2$ and $m_{j_1,p_2} > m'_{j_1,p_2}$.  
% In particular, $m_{j_1,p_2} > 0$.  Together with $q_{p_1} \ge \sum_j m'_{jp_1} > \sum_j m_{jp_1}$, this contradicts the assumption that $m$ is not wasteful.

% Therefore there is some $i_1 \ne i_0$ such that $m_{i_1o_1} < m'_{i_1o_1}$.  Since $m'_{i_1} \ne m_{i_1}$ and $m'_{i_1}$ $ul$-dominates $m_{i_1}$ there is some $o_2$ such that $o_1 \succ_{i_1} o_2$ and $m_{i_1o_2} > m'_{i_1o_2}$.  In particular, $m_{i_1o_2} > 0$, so $o_2 \lhd o_1$.  
% Since $O$ is finite, repeating this argument leads eventually to a cycle $o_0 \lhd_m o_h \lhd_m \cdots \lhd_m o_1 \lhd_m o_0$.


\section{Proof of Theorem \ref{th:WeakStrategyProof}} \label{app:WeakStrategyProof}

Let $E = (I,O,r,q)$ be a uniform unrestricted eligibility CEE.  We assume that $E$ satisfies the GMC: $r|I| \le \sum_o q_0$.  As we stated in Section \ref{sec:StrategyProof}, we wish to prove something stronger than Theorem \ref{th:WeakStrategyProof}, namely that there is no eating strategy for an agent that results in an allocation that strictly stochastically dominates the GCPS allocation.  We now fix an agent $i$.  We need to explain the consequences of $i$ following a particular eating function.

We imagine that the process begins at time $0$, and we specify how it continues until the time $r$ when all agents have fulfilled their requirement.  We assume that agent $i$'s behavior is described by an eating function $e_i \colon [0,r] \to O$ that is piecewise constant, and that is \emph{feasible} in the sense that process described below is well defined.
For each $t \in [0,r]$ there are the following objects:
\begin{enumerate}
  \item[(a)] $p^{e_i}(t)$ is a partial allocation.
  \item[(b)] $\alpha^{e_i}(t) = \{\, o : q^{e_i}_o(t) > 0 \,\}$ is the set of available objects.
  \item[(c)] For each $j \ne i$, $e^{e_i}_j(t)$ is the $\succ_j$-best element of $\alpha^{e_i}(t)$.
  \item[(d)] $e^{e_i}_i(t) = e_i(t) \in \alpha^{e_i}_i(t)$.
\end{enumerate}
We assume that $p^{e_i}(0) = 0$.  We require that $q^{e_i}(t)$ and $p^{e_i}(t)$ are continuous and piecewise linear functions of $t$, and that when their derivatives  $\dq^{e_i}(t)$ and $\ddp^{e_i}(t)$ with respect to time  are defined, for all $j$ they satisfy:
\begin{enumerate}
  \item[(a)] $\dq^{e_i}_o(t) = -|\{\, j : e^{e_i}_j(t) = o \,\}|$.
  \item[(b)] $\ddp^{e_i}_{jo}(t) = 1$ if $e^{e_i}_j(t) = o$, and otherwise $\ddp^{e_i}_{jo}(t) = 0$.
\end{enumerate}
The allocation for $e_i$ is $p^{e_i}(r)$.

Let $\cE_i$ be the set of feasible eating functions for $i$ that change objects at most $|O|^2$ times.   This is a compact set with respect to the metric $d(e_i,e_i') = \int_{t = 0}^{r_i} 1 - \delta_{e_i(t),e_i'(t)} \, dt$ (Kronecker $\delta$) and the set of feasible elements is a closed subset.  We say that two feasible eating functions are \emph{equivalent} if they have the same total consumption of each object during any interval between two times at which objects are exhausted, in which case the two functions give the same allocation to $i$.  Clearly any feasible eating schedule is equivalent to an element of $\cE_i$.  
Let $a$ be $i$'s favorite element of $\{\, o : g_{io} = r_i \,\}$.  It is easy to see that the map from feasible elements of $\cE_i$ to $i$'s allocation of $a$ is continuous.  Therefore there is a feasible $e_i \in \cE_i$ that maximizes the overall allocation of $a$ for $i$.

Let $t_1$ be the time at which $a$ becomes unavailable to $i$ under $e_i$, either because $t_1 = r$ or $q_a^{e_i}(t_1) = 0$, so $q_a^{e_i}(t) > 0$ for all $t < t_1$.  Aiming at a contradiction, suppose that there are  times prior to $t_1$ at which $i$ eats something other than $a$, and let $t_0$ be the least upper bound of the set of such times.  Since $e_i$ switches objects finitely many times, optimality implies that $t_0 < t_1$.  Let $\delta > 0$ be the largest number such that $i$ does not eat $a$ during the interval $(t_0 - \delta,t_0)$.
Let $e_i'$ be an eating function that agrees with $e_i$ on $[0,t_0 - \delta]$ and eats only $a$ from $t_0 - \delta$ until the time $t_1'$ when it ceases to be available because  $q_a^{e_i'}(t_1') = 0$.  We have $t_1 - t_1' - \delta \ge 0$ because this is the difference between the consumption of $a$ by $i$ under $e_i$ and the consumption of $a$ by $i$ under $e_i'$.  In particular $t_1' < t_1$.

During the interval $[t_0,t_1']$ the events that change the availability of objects are exhaustion of objects.
Aiming at a contradiction, suppose that there is a  $t \in [t_0,t_1']$ such that there is some $o \in O(t) \setminus \{a\}$ such that $o$ is available under $e_i$ but not under $e_i'$, and let $t^*$ be the greatest lower bound of the set of such $t$.
For each $t \in [t_0,t^*]$, $j \ne i$, and $o \ne a$,  if $j$ was consumes $o$ at time $t$ under $e_i'$, and $o$ is available to $j$ at time $t$ under $e_i$, then $j$ is consuming $o$ at time $t$ under $e_i$ because the set of things in $O \setminus \{a\}$ that are available at time $t$ under $e_i$ is a subset of the set of things  in $O \setminus \{a\}$ that are available at time $t$ under $e_i'$.  Therefore, for $t \in [t_0,t^*]$, $j \ne i$, and $o \ne a$, the total consumption of $o$ by $j$ is weakly greater under $e_i$ than under $e_i'$ because $j$ is eating $o$ under $e_i$ at each time when it has not been fully allocated under $e_i'$ and is being eaten by $j$ under $e_i'$.  It follows that an object that is fully allocated at or before $t^*$ under $e_i'$ is also  fully allocated at or before $t^*$ under $e_i$.  Therefore, for times near $t^*$ (on both sides if $t^* < t_1'$) for every $o \in O(t) \setminus \{a\}$ such that $o$ is available under $e_i$,  $o$ is also available under $e_i'$, contradicting the definition of $t^*$.

We have shown that for all  $t \in [t_0,t_1']$, all $j \in I \setminus \{i\}$, and all $o \in O \setminus \{a\}$, if $o$ is not available to $j$ at $t$ under $e_i'$, then it is not available to $j$ at $t$ under $e_i$. It follows that  for all  $t \in [t_0,t_1']$ the set of $j$ eating $a$ at $t$ under $e_i$ is a superset of the set of $j$ eating $a$ at $t$ under $e_i'$.  In particular, the amount of $a$ that remains uneaten at $t_1'$ under $e_1$ is not more than $\delta$, and in order to have $t_1 - t_1' - \delta \ge 0$ it must be the case that the amount that remains uneaten is $\delta$ and $i$ is the only agent eating $a$ during the interval $[t_1',t_1]$.  Since all agents are allowed to eat all objects, and they all have the same requirement, once an agent starts to eat $a$, she will continue to do so until it is exhausted or time $r$, so it must be the case that no agent other than $i$ was eating $a$ at any time.
Even if all this is the case, the consumption of $a$ by $i$ under $e_i'$ is as large as the the consumption of $a$ by $i$ under $e_i$, so $e_i'$ is  also a feasible element of $\cE_i$ that maximizes the overall allocation of $a$ for $i$.  The argument can now be repeated with $e_i'$ in place of $e_i$, and after finitely many repetitions we arrive at the conclusion that there is an optimal eating function that eats $a$ from time $0$ until time $r$ or $a$ is exhausted without anyone else ever eating $a$.  Since $r < q_a$, the unique such eating plan is to eat $a$ at all times.

Let $b$ be $i$'s second favorite element of $O$.  With obvious modifications, the argument above easily extends to show that among the eating functions for $i$ that maximize the overall allocation of $a$ for $i$, those that maximize the overall allocation of $b$ for $i$ have $i$ eating $b$ at every time when $a$ is not available and $b$ is.  The intuition is the same: if $i$ faces competition for $b$, the amount of $b$ consumed is maximized by eating $b$ from the time when $a$ becomes unavailable, and if there is no competition for $b$, then the unique optimal eating plan is to eat $b$  from the time when $a$ becomes unavailable to time $r$.  Extending this inductively, if an eating function for $i$ results in an allocation for $i$ that stochastically dominates the allocation resulting from the GCPS mechanism, then it agrees with the eating function $e^{\succ_i}$ of the GCPS mechanism, and therefore the resulting allocation is the GCPS allocation, which is the assertion of Theorem \ref{th:WeakStrategyProof}.

\norev
 
This result opens up an interesting possibility.  Suppose that, pursuing a suggestion of \cite{as03aer}, for affirmative action purposes a school has been divided into three parts, one with 30\% of the seats that is reserved for minority students, one with 30\% of the seats that is reserved for majority students, and one with 40\% of the seats that accepts all students.   After the procedure has been run the results may be disappointing if, for example, one of the three schools is severely underenrolled.  Alternatively, it may happen that an elite school is either underenrolled or has admitted many less qualified students while students with superb test scores failed to get in, due to the luck of the draw.  In both of these cases one might wish to rerun the process with adjusted parameters, changing the seat assignments of the three schools or the test score cutoff of the elite school.  

In their most basic forms school choice mechanisms based on bilateral matching mechanisms require that the schools have priorities that are strict preference orderings of the students.  These may refine legally mandated priorities, but in some cases the additional preferences of the schools do not represent actual social desiderata, and may interfere with efficiency.  But it is also possible that it is socially desirable that the preferences of the highest priority students are given greatest consideration.  In the matching resulting from deferred acceptance each school has a priority cutoff such that the proposals of students with higher priority are certainly accepted and those with lower priority are certainly rejected.  To approximate this with the GCPS mechanism one may run it repeatedly, gradually increasing the priority cutoffs of schools that are overdemanded (as manifested by many students receiving assignment probabilities far below one) until there is approximate balance of supply and demand at each school.

In many mechanism design contexts running a mechanism repeatedly with adjusted parameters would be regarded as cheating, since the putative mechanism (a single run of the GCPS mechanism) has been replaced by a much more complicated iterative process.  However, from the the point of view of the final run it is still best to report one's true preference, so the only way that misreporting might possibly be beneficial is if the manipulator managed to maneuver the iterative adjustment process to a different endpoint, and of course it is implausible that anyone could be that foresightful.  Thus it seems to reasonable to regard the procedure as strategy proof even if there are the sorts of ex post adjustments we have described.
