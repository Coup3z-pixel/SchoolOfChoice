\documentclass[12pt]{article}
\usepackage{amsmath}
\usepackage{amstext}
\usepackage{amssymb}
\usepackage{amsthm}
\makeatletter
\usepackage{graphicx,epsf}
\usepackage{times,float}
\usepackage{enumerate}
\usepackage[round,comma]{natbib}
\usepackage[colorlinks=true,citecolor=blue]{hyperref}
\usepackage{bm}
\usepackage{multirow}
%\usepackage{blkarray}
\usepackage{rotating}

\setlength{\textwidth}{6.4in} \setlength{\textheight}{8.5in}
\setlength{\topmargin}{-.2in} \setlength{\oddsidemargin}{.1in}
\renewcommand{\baselinestretch}{1.3}

\theoremstyle{definition}
\newtheorem{thm}{Theorem}
\newtheorem*{thm*}{Theorem}
\newtheorem{prop}{Proposition}
\newtheorem{cor}{Corollary}
\newtheorem{lem}{Lemma}
\newtheorem*{lem*}{Lemma}
\newtheorem{claim}{Claim}
\newtheorem{rem}{Remark}
\newtheorem{ex}{Example}
\newtheorem{fact}{Fact}
\newtheorem*{fact*}{Fact}
\newtheorem{remark}{Remark}


\newcommand{\rR}{\mathrel{R}}
\newcommand{\rP}{\mathrel{P}}
\newcommand{\real}{\mathbb{R}}
\newcommand{\norev}{\medskip \centerline{\textbf{No Revisions Below}} \medskip}
\renewcommand{\Re}{\mathbb{R}}
\newcommand{\In}{\mathbb{Z}}

\newcommand{\bare}{\overline{e}}
\newcommand{\barl}{\overline{l}}

\newcommand{\bq}{\mathbf{q}}

\newcommand{\cE}{\mathcal{E}}
\newcommand{\cH}{\mathcal{H}}
\newcommand{\cM}{\mathcal{M}}
\newcommand{\cP}{\mathcal{P}}
\newcommand{\cX}{\mathcal{X}}

\newcommand{\dr}{{\dot r}}
\newcommand{\dq}{{\dot q}}
\newcommand{\dg}{{\dot g}}
\newcommand{\ddp}{{\dot p}}

\newcommand{\hA}{{\hat A}}
\newcommand{\hO}{{\hat O}}

\newcommand{\halpha}{{\hat \alpha}}

\newcommand{\ta}{{\tilde a}}
\newcommand{\te}{{\tilde e}}
\newcommand{\tn}{{\tilde n}}

\newcommand{\tB}{{\tilde B}}

\newcommand{\bark}{{\overline k}}
\newcommand{\bart}{{\overline t}}

\newcommand{\varep}{\varepsilon}

\newcommand{\bone}{\mathbf{1}}

\begin{document}

\title{GCPS Schools: A User's Guide}

\author{Andrew McLennan\footnote{School of Economics, University of
    Queensland, {\tt a.mclennan@economics.uq.edu.au}} \and  Shino
Takayama\footnote{School of Economics, University of
  Queensland, {\tt s.takayama1@uq.edu.au}} \and Yuki Tamura\footnote{Center for Behavioral Institutional Design, NYU Abu Dhabi; {\tt yuki.tamura@nyu.edu}}}

\date{\today}

\maketitle

\begin{abstract}
This document provides a brief introduction to the software package GCPS Schools.
\end{abstract}

% \pagebreak

\section{Introduction}

In our paper ``An Efficient School Choice Mechanism Based on a
Generalization of Hall's Marriage Theorem'' we describe a new
algorithm for school choice, along with its theoretical foundations.
This algorithm has been implemented in the software package \emph{GCPS
  Schools}.  This document describes this algorithm, from the point of
view of a user.  It doesn't assume that the reader has already read
our paper, but of course we are leaving out lots of relevant
information.

To begin with we describe a simple example of an input file, which the
computer expects to find in a file called \texttt{schools.scp}, which
the computer expects to find in the current directory.  (If there is
no such file the computer simply complains and quits.)  Our input
files begin with a comment between \texttt{/*} and \texttt{*/}.  This
is purely for your convenience, and the comment can be of any length,
and provide whatever information is useful to you, but it is mandatory
insofar as the computer will insist that the first two characters are
\texttt{/*} and will only start extracting information after it sees
the \texttt{*/}.  The computer divides the remainder into
``generalized white space'' (in addition to spaces, tabs, and new
lines, `\texttt{(}', `\texttt{)}', and `\texttt{,}' are treated as
white space) and ``tokens,'' which are sequences of characters without
any of the generalized white space characters.  Tokens are either
prescribed words, numbers, or student tags (a student number followed
by `\texttt{:}') and everything must be more or less exactly as shown,
modulo white space, so, for example, the first line must not be
\texttt{There are 3 students and 1 school}, but it could be
\texttt{There are 3 students and \ \ 1 schools}.

\begin{obeylines}\texttt{
/* This is a sample introductory comment. */
There are 4 students and 3 schools
The vector of quotas is (1,2,1)
The priority matrix is
     1     1     1
     1     0     1
     1     1     1
     1     1     1
The students numbers of ranked schools are (3,2,3,3)
The preferences of the students are
1:  1  2  3  
2:  1  3  
3:  1  2  3  
4:  1  2  3  
The priority thresholds of the schools are
1   1   1   
  }
\end{obeylines}

The next line gives the quotas (i.e., the capacities) of the schools,
so school 2 has two seats, and the other two schools each have one
seat.  Here we see the convenience of making `\texttt{(}',
`\texttt{)}', and `\texttt{,}' white space characters: otherwise we
would have had to write \texttt{The vector of quotas is 1 2 1}.

Our treatment of priorities is somewhat different from what is typical
in the school choice literature, where the priority is thought of as
representing the ``utility'' of the school, and is often required to
come from a strict ranking of the students.  At this stage a student's
priority at a school is either 1 if she is allowed to attend the
school, and may be assigned a seat there, and otherwise it is 0.
(We'll talk about more complicated priorities later.)  A student's
priority at a school may be 0 because she is not qualified (it is a
single sex school for boys, or her test scores are too low) or it may
be 0 because the student prefers a seat at her ``safe school'' as we
explain below, and can insist on receiving a seat at a school that is
no worse for her than that.

The next line provides information (for each student, the number of
schools for which she has priority 1) that the computer could figure
out for itself, but we prefer to confirm that whatever person or
software prepared the input knew what they were doing.  After that
come the students' preferences: for each student, that student's tag
followed by the schools she might attend, listed from best to worst.
Finally, there are the schools' minimum priorities for admission,
which in this context are all 1: a student is good enough to admit to
a school if her priority for that school is 1 and not
otherwise. (Again, we'll talk about more complex situations later.)

What the software does (primarily) is compute a matrix of assignment
probabilities.  Our particular example gives the following output:
\begin{obeylines}\texttt{
The allocation is:
\ \ \ \ \ 1:    \    2:  \      3:
1:      0.25     0.67     0.08
2:      0.25     0.00     0.75
3:      0.25     0.67     0.08
4:      0.25     0.67     0.08
}
\end{obeylines} \noindent
Note that the sum of the entries in each row is 1 and the sum of the
entries in each school's column is that school's quota.  In general
the sum of the quotas may exceed the number of students, in which case
we require that the sum of the entries in each school's column does
not exceed that school's quota. An assignment of probabilities with
these properties --- each student's total assignment is 1 and no
school is overassigned --- is a \emph{feasible allocation}.  An
important point \citep{bckm13aer} is that any feasible allocation can
be written as a convex combination of matrices with entries in
$\{0,1\}$, i.e., as a lottery over pure assignments.

Our mechanism is based on the ``simultaneous eating'' algorithm of
\cite{bm01} for probabilistic allocation of objects, as generalized by
\cite{balbuzanov22jet}.  In this problem each student consumes
probability of a seat in her favorite school (school 1) until that
resource is exhausted at time 0.25, at which point each student
switches to the next best thing.  This continues until school 2 is
also exhausted, after which all finish up by consuming probability of
a seat in school 3.

This makes good sense if the schools simply fill up one by one, as in
this example, but is that always what happens?  Actually, no.  To help
understand this we introduce a new concept, the ``safe school.''  The
idea is that each student has one school, say the closest school or
the school that a sibling attends, to which she is guaranteed
admission if she insists.  (It is a major advantage of our mechanism
that we can provide such a guarantee.)  Each student submits a ranking
of schools that she (weakly) prefers to the safe school, and her
priority is 1 at those schools and 0 at all others.

Now suppose that there are two schools, say 1 and 2, that are pretty
popular.  Some students have school 1 as their safe school, but prefer
2, and some students have school 2 as their safe school, but prefer 1.
There are also some students who have other safe schools, but prefer
either 1 or 2, or both.  As the students consume probability at their
favorite schools, there can come a time at which schools 1 and 2
together only have enough remaining capacity to serve the students who
insist on going to one of the two schools, even though school 1 has
excess capacity if it can ignore the students who have 1 as their safe
school but prefer 2 and the students who have 2 as their safe school
but prefer 1, and similarly for school 2.  At this time further
consumption of capacity at schools 1 and 2 is restricted to those
students who cannot be assigned to other schools, so further
consumption of these schools is denied to students who do not have 1
or 2 as their safe school, and also to students who have 1 or 2 as
their safe school but prefer some third school that is still
available.  For one of the latter students the least preferred of the
schools she is willing to attend that is still available becomes the
new safe school.

More generally, let $P$ be a set of schools, and let $J_P$ be the set
of students whose priorities for all schools outside of $P$ are 0.
For any $i \in J_P$, a feasible allocation must assign probability 1
to student $i$ receiving a seat in $P$, so a necessary condition for
the existence of a feasible allocation is that the total capacity of
the schools in $P$ is not less than the number of students in $J_P$.
In fact this condition is sufficient for the existence of a feasible
allocation: if, for each set of schools $P$, the total capacity of the
schools in $P$ is not less than the number of students in $J_P$, then
a feasible allocation exists.  This is not an obvious or trivial
result, and a somewhat more general version of it is one of the main
points of our paper.

We can now describe the algorithm in a bit more detail.  At each time
each student is consuming probability of a seat at the favorite school
among those that are still available to her.  This continues until the
first time there is a set of schools $P$ such that the remaining
capacity is just sufficient to meet the needs of the students in
$J_P$.  At this point the problem divides into two subproblems, one
corresponding to the sets $P$ and $J_P$ and the other corresponding to
the complements of these sets.  These problems have the same form as
the original problem, and can be treated algorithmically in the same
way, so the algorithm can descend recursively to smaller and smaller
subproblems until a feasible allocation has been fully computed.


\section{What If There Are Many Schools?}

As the algorithm was described above, at each step it looked ahead,
for each set of schools $P$, to determine the time at which it becomes
necessary to restrict further consumption of schools in $P$ to
students in $J_P$.  This is not unduly burdensome if there are a
moderate number of schools.  (For a ``toy'' example with 20 schools,
hence over one million sets of schools, the algorithm finishes in
about 10 seconds.)  But some school choice problems have several dozen
or even hundreds of schools, and will overwhelm the naive version of
the algorithm described above.  There are several things that can be
done about this.

When the software is invoked, the computer looks in the current
directory to see if there is a file called \texttt{related.mat}.  If
it is there, it reads it, expecting it to look as below.  As before
there is an initial comment between the expected initial \texttt{/*}
and the first occurrence of \texttt{*/}.  The next line declares the
number of schools, which must agree with the number of schools of the
given problem, and the line after that declares an integer called
\texttt{max\_clique} which must be between 1 and the number of
schools.  After that there are numbers, separated by generalized white
space, each of which must be either 0 or 1, that are the entries of a
square matrix called \texttt{related} with the number of rows and
columns equal to the number of schools.  The diagonal entries that
must all be 1, and the matrix must be symmetric: for all $j$ and $k$,
the $(j,k)$ and $(k,j)$ entries must agree.  If the computer does not
find \texttt{related.mat} it sets \texttt{max\_clique} equal to the
number of schools and \texttt{related} is set to the square matrix
with that number of rows and columns whose entries are all 1.

We think of \texttt{related} as encoding an undirected graph whose
nodes are the schools: there is an edge between schools $j$ and $k$ if
and only if the $(j,k)$ and $(k,j)$ entries of \texttt{related} are
both 1. A \emph{clique} of such a graph is a set of schools $P$ such
that there is an edge between any two of its elements.  In its search
for the $P$ that first has to restrict access to students in $J_P$,
the computer only considers cliques for the graph with
\texttt{max\_clique} or fewer elements.  To help this along a bit, the
computer looks for schools whose capacity will not be exceeded even if
every student who ranks it ends up receiving a seat.  Such a school is
never an element of a minimal set $P$ that needs to restrict access to
$J_P$, and for such a school the computer changes all of the entries
of the corresponding row and column of \texttt{related} to zero.

The specification of \texttt{related} and \texttt{max\_clique}
represents a guess that as the computer descends recursively to
smaller and smaller subproblems, the only sets $P$ that ever require
restricting access to $J_P$ are cliques of the graph represented by
\texttt{related} with no more than \texttt{max\_clique} elements.  Two
points should be emphasized: 1) if the computation succeeds, the
choices of \texttt{related} and \texttt{max\_clique} have no effect on
the final result; 2) when this guess is wrong, the computer will
eventually notice that some set of schools has been overallocated, and
it will quit with a message to that effect.  A natural next step would
be the modify \texttt{related} by increasing the relatedness of these
schools.

The specification of \texttt{related} is the responsibility of the
user, and is a matter of judgment and intuition.  (In the longer run
we hope that the user's judgment and intuition will be refined by
experience.)  Perhaps the main intuition is that it is very unlikely
that two schools will be in a $P$ that first requires restricting
access to $J_P$ if they are far apart.  


\section{Concluding Remarks} \label{sec:Conclusion}

We have provided a school choice mechanism that is a specialization of
the GCPS mechanism of \cite{balbuzanov22jet}, which is in turn a
generalization of the PS mechanism of BM.  This mechanism guarantees
each student a seat in a school that is at least as desirable as any
of the schools she is legally entitled to attend.  When there are many
students for each school, it is effectively strategy proof.  It is
$sd$-efficient, which (as BM stress) is a stronger condition than ex
post efficiency.  In contrast, bilateral matching mechanisms based on
randomly generated priorities for the schools are (at least in their
most basic forms) not even ex post efficient.  A result of
\cite{bckm13aer} implies that it is implementable: the assignment
probabilities it generates can be obtained from a randomization over
pure assignments.  It satisfies anonymity, equal treatment of equals,
and a natural generalization of the envy-freeness condition satisfied
by the PS mechanism.  Using a novel generalization of Hall's marriage
theorem, we have described a computational implementation of this
mechanism that is tractable even for very large school choice
problems.

Although we have emphasized the school choice application, the underlying idea of our procedure, the application of the GCPS mechanism to CEE, is potentially of interest in many other domains, with many variations.  A possibility stressed by BM, \cite{cho18scw}, and \cite{balbuzanov22jet} (perhaps among others) is that the mechanism can be varied by making the eating speeds depend on various things.  This seems unmotivated in school choice, but in other domains it may be quite interesting.

A possibility we intend to explore in subsequent research is that instead of consuming probability of desirable objects, the agents may discard probability of undesirable objects.  In the case of $n$ agents and $n$ objects, each agent is endowed with one unit of each object, and at each time during the interval $[0,n-1]$ she discards probability of the least desirable object that she has not fully discarded for which discarding is still allowed.  Discarding of an object is disallowed when the agents' total remaining endowment of it is one, but it may also be disallowed for some agents in the event that the process reaches a facet of $R$.  The characterization of the PS mechanism given by \cite{bh12} implies that the discarding mechanism is certainly different, but otherwise its properties await investigation.  It seems appropriate for problems, perhaps such as chore assignment, in which the agents' main concern is to avoid the objects that are most noxious for them.  Some agents may be unqualified to receive certain objects, and one may recognize this by taking away their endowments of such objects at the outset, but this seems unfair insofar it amounts to giving them a head start.  Giving such agents slower discarding speeds is one way this issue could be addressed.


\begin{appendix}

\section{Proof of Lemma \ref{lem:cyclic}} \label{app:Cyclic}

 The argument below simply adapts the proof of Theorem 1 of \cite{cd16} to our setting.
 
Suppose that $m$ is wasteful, so there is an agent $i$ and a pair $o,o'$ of objects such that $o \succ_i o'$, $m_{io} < g_{io}$, $m_{io'} > 0$, and $\sum_j m_{jo} < q_o$. For sufficiently small $\delta > 0$, setting $m'_{io} = m_{io}+\delta$,  $m'_{io'} = m_{io'} - \delta$, and $m'_{jp} = m_{jp}$ for all other $(j,p)$ gives an allocation $m'$ such that $m'_i$ $e$-dominates $m_i$ for any $e \in \{sd,dl,ul\}$, $m_i' \ne m_i$, and $m'_j = m_j$ for all $j \ne i$, so $m'$ $e$-dominates $m$. 

Suppose that there is a cycle $o_0 \lhd_m o_1 \lhd_m \cdots \lhd_m o_h \lhd_m o_0$.
If $i_0, \ldots, i_k$ and $m(\delta)$ are as above, then, for sufficiently small $\delta >0$, $m(\delta)$ is an allocation, $m_i(\delta)$ $e$-dominates $m_i$  for each $i = 0, \ldots, k$ and $e \in \{sd,dl,ul\}$, and $m'_j = m_j$ for all other $j$, so $m'$ $e$-dominates $m$.

Now suppose that allocation $m$ is not wasteful and is $e$-dominated by the allocation $m'$.   Fix an agent $i_0$ such that $m'_{i_0} \ne m_{i_0}$. 
There are two cases, depending on whether $e = dl$ or $e = ul$.  (Either argument can handle the case $e = sd$.)

First suppose that $e = dl$.  Since $m_i'$ $ul$-dominates $m_i$ there are objects $o_0$ and $o_1$  such that $o_1 \succ_{i_1} o_0$, $m'_{i_0o_0} < m_{i_0o_0}$, and $m'_{i_0o_1} > m_{i_0o_1}$.  These conditions imply that $m_{i_0o_0} > 0$ and $m_{i_0o_1} < g_{i_0o_1}$, so $o_0 \lhd_m o_1$. 
If $m'_{jo_1} \ge m_{jo_1}$ for all $j \ne i_0$, then $\sum_j m_{jo_1} < \sum_j m'_{jo_1} \le q_{o_1}$ and $m_{i_0o_0} > 0$, contradicting nonwastefulness of $m$.  Therefore there is some $i_1 \ne i_0$ such that $m'_{i_1o_1} < m_{i_1o_1}$.  Since $m'_{i_1}$ $dl$-dominates $m_{i_1}$, there is some $o_2$ such that $o_2 \succ_{i_1} o_1$ and $m'_{i_1o_2} > m_{i_1o_2}$.  We have $m_{i_1o_1} > 0$ and $m_{i_1o_2} < g_{i_1o_2}$, so $o_1 \lhd o_2$.
Since $O$ is finite, repeating this argument leads eventually to a cycle $o_0 \lhd_m o_1 \lhd_m \cdots \lhd_m o_h \lhd_m o_0$.

Now suppose that $e = ul$. Since $m_i'$ $dl$-dominates $m_i$ there are objects $o_0$ and $o_1$  such that $o_0 \succ_{i_1} o_1$, $m'_{i_0o_0} > m_{i_0o_0}$, and $m'_{i_0o_1} < m_{i_0o_1}$, so that $m_{i_0o_1} > 0$ and $o_1 \lhd o_0$. 

Aiming at a contradiction, suppose that $m_{jo_1} \ge m'_{jo_1}$ for all $j \ne i_0$, so that $q_{o_1} \ge \sum_j m_{jo_1} > \sum_j m'_{jo_1}$.  Since $\sum_{j,p} m_{jp} = \sum_{j,p} m'_{jp} = \sum_j r_j$, there is a $p_1$ such that $\sum_j m_{jp_1} < \sum_j m'_{jp_1}$ and a $j_1$ such that $m_{j_1p_1} < m'_{j_1p_1}$. Since $m'_{j_1} \ne m_{j_1}$ and $m'_{j_1}$ $ul$-dominates $m_{j_1}$, there is a $p_2$ such that $p_1 \succ_{j_1} p_2$ and $m_{j_1,p_2} > m'_{j_1,p_2}$.  
In particular, $m_{j_1,p_2} > 0$.  Together with $q_{p_1} \ge \sum_j m'_{jp_1} > \sum_j m_{jp_1}$, this contradicts the assumption that $m$ is not wasteful.

Therefore there is some $i_1 \ne i_0$ such that $m_{i_1o_1} < m'_{i_1o_1}$.  Since $m'_{i_1} \ne m_{i_1}$ and $m'_{i_1}$ $ul$-dominates $m_{i_1}$ there is some $o_2$ such that $o_1 \succ_{i_1} o_2$ and $m_{i_1o_2} > m'_{i_1o_2}$.  In particular, $m_{i_1o_2} > 0$, so $o_2 \lhd o_1$.  
Since $O$ is finite, repeating this argument leads eventually to a cycle $o_0 \lhd_m o_h \lhd_m \cdots \lhd_m o_1 \lhd_m o_0$.




\end{appendix}

%%%------------------------------------------------------------------------------------
%%%------------------------------------------------------------------------------------
\bibliographystyle{agsm}
\bibliography{pa_ref}
\end{document}

